\DocumentMetadata
  {
    lang=en-US,
    pdfversion=2.0,
    pdfstandard=ua-2,
    testphase={phase-III,math,title,table,firstaid},
  }
\documentclass{article}
\usepackage{ctex}
\usepackage{xpinyin}
\iftutex
\setmainfont{Libertinus Serif}
\setCJKmainfont{SimSun}
\else
\usepackage[T1]{fontenc}
\usepackage{lmodern}
\fi
\begin{document}
\xpinyin*{汉语拼音示例}

\begin{pinyinscope}
列位看官:你道此书从何而来?说起根由,虽近荒唐,细按则深有趣味。
待在下将此来历注明\footnote{虽近荒唐},方使阅者\xpinyin{了}{liao3}然不惑。
\end{pinyinscope}

\xpinyin{长}{chang2}

\xpinyin*{\xpinyin{重}{zhong4}要}

\pinyin{lv2zi}

\pinyin{nv3hai2zi}

\end{document}