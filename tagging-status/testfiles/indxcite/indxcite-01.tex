\DocumentMetadata
  {
    lang=en-US,
    pdfversion=2.0,
    pdfstandard=ua-2,
    testphase={phase-III,title,math,table,firstaid}
  }
\documentclass{article}
\usepackage{indxcite}
\newindex{aut}{adx}{and}{Author Index} % makeindex -o example.and example.adx
\begin{document}
\bibliographystyle{dcu_ic}
\citationstyle{dcu}
\citationindex{aut}
\citationformat{textit}
This is a small example demonstrating the use of the \texttt{indxcite}
package.
It is possible, with a single command, to both make a citation and
generate index entries for the authors of the cited word,
for example, \iciteasnoun{latexcomp}.
\pagebreak
Unusual effects are possible. If you want you can have the authors’ names to
appear in small caps in the index with a typewriter font used for the
page numbers thusly.
\citationformat[textsc]{texttt}
And make the citation \icite[Chapter 5]{latexcomp}.
You can then reset the format back to a more conventional setting.
\citationformat{textit}
Using a format other than roman can be a good idea so that automatically
generated index entries are distinguishible from manually generated
ones\index[aut]{Goossens, M.}.
Note that if the authors’ names are indexed with more than one format
as in this document (i.e., usually in roman but once in small caps) then
multiple index entries will be generated.
\pagebreak
\indexcite[(]{latexcomp}
If a citation is refered to over several paragraphs you may want to
index the whole range of text.
In this case you need to use three commands: one to generate the
citation\cite{latexcomp}; one to mark the begining of the text to
be indexed and one to mark the end of the text to be indexed.
\pagebreak
Then, if the indexed text runs over multiple pages, this will be
reflected in the index entry\indexcite[)]{latexcomp}.
It is a good idea to reset the citation format before the bibliography
is processed so that index entries for the bibliography can be easily
distinguished for those for the rest of the document.
\citationformat{textbf}
\begin{thebibliography}{xx}
\harvarditem[Goossens et~al.]{Goossens, Mittelbach \harvardand\
Samarin}{1993}{latexcomp}{\\{Goossens, M.}\\{Mittelbach, F.}\\{Samarin, A.}}
Goossens, M., Mittelbach, F. \harvardand\ Samarin, A. \harvardyearleft
1993\harvardyearright .
\newblock {\em The {\LaTeX{}} Companion}, Addison-Wesley, Reading,
Massachusetts.\indexcite{latexcomp}
\end{thebibliography}
\printindex[aut]
\end{document}
