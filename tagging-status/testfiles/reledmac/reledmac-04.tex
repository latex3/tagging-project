% adapted from 1-tabular.tex
% errors
\DocumentMetadata
  {
    lang=en-US,
    pdfversion=2.0,
    pdfstandard=ua-2,
    testphase={phase-III,title,math,table,firstaid}
  }
\documentclass{article}

\usepackage[colorlinks]{hyperref}

\usepackage[series={A,B},noend,noeledsec,noledgroup]{reledmac}
\firstlinenum{1}
\linenumincrement{1}%

\begin{document}

\beginnumbering
\pstart[\section{Left align}]
\begin{edtabularl}
linea prima & linea prima\ledouternote{sidenote l} \\
linea secunda & linea secunda\footnoteA{familiar footnote A.} \\
linea tertia & linea \edtext{tertia}{\Afootnote{critical note A}} \\
linea quarta & linea \edtext{quarta}{\Bfootnote{critical note B}} \\
linea quinta & linea quinta\footnoteB{familiar footnote B.} \\
linea sexta & linea sexta
\end{edtabularl}
\pend

\pstart[\section{Center align}]
\begin{edtabularc}
linea prima & linea prima\ledouternote{sidenote c} \\
linea secunda & linea secunda\footnoteA{familiar footnote A.} \\
linea tertia & linea \edtext{tertia}{\Afootnote{critical note A}} \\
linea quarta & linea \edtext{quarta}{\Bfootnote{critical note B}} \\
linea quinta & linea quinta\footnoteB{familiar footnote B.} \\
linea sexta & linea sexta
\end{edtabularc}
\pend



\pstart[\section{Right align}]
\begin{edtabularr}
linea prima & linea prima\ledouternote{sidenote r} \\
linea secunda & linea secunda\footnoteA{familiar footnote A.} \\
linea tertia & linea \edtext{tertia}{\Afootnote{critical note A}} \\
linea quarta & linea \edtext{quarta}{\Bfootnote{critical note B}} \\
linea quinta & linea quinta\footnoteB{familiar footnote B.} \\
linea sexta & linea sexta
\end{edtabularr}
\pend

\endnumbering

\end{document}
