\DocumentMetadata
  {
    lang=en-US,
    pdfversion=2.0,
    pdfstandard=ua-2,
    tagging=on
  }
\documentclass{article}

\usepackage{arabluatex}
\newfontfamily\arabicfont{Amiri}[Script=Arabic]

\title{arabluatex tagging test}

\begin{document}

Arabic, like Hebrew and
Syriac, is written and read from right to left. The letters
of the alphabet (\arb{.hurUf-u 'l-hijA'-i}, \arb{.hurUf-u
  'l-tahajjI}, \arb{al-.hurUf-u 'l-hijA'iyyaT-u}, or
\arb{.hurUf-u 'l-mu`jam-i}) are twenty-eight in number and
are all consonants, though three of them are also used as
vowels (see §~3).

\begin{arab}
'at_A .sadIquN 'il_A ju.hA ya.tlubu min-hu .himAra-hu
li-yarkaba-hu fI safraTiN qa.sIraTiN fa-qAla la-hu:
sawfa 'u`Idu-hu 'ilay-ka fI 'l-masA'-i
wa-'adfa`u la-ka 'ujraTaN. fa-qAla ju.hA:
'anA 'AsifuN jiddaN 'annI lA 'asta.tI`u 'an
'u.haqqiqa la-ka ra.gbata-ka fa-'l-.himAr-u laysa hunA
'l-yawm-a.  wa-qabla 'an yutimma ju.hA kalAma-hu bada'a
'l-.himAr-u yanhaqu fI 'i.s.tabli-hi. fa-qAla la-hu
.sadIqu-hu: 'innI 'asma`u .himAra-ka yA ju.hA
yanhaqu. fa-qAla la-hu ju.hA: .garIbuN
'amru-ka yA .sadIqI 'a-tu.saddiqu 'l-.himAr-a
wa-tuka_d_diba-nI?
\end{arab}

\end{document}