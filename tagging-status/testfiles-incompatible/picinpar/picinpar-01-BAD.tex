\DocumentMetadata
  {
    lang=en-US,
    pdfversion=2.0,
    pdfstandard=ua-2,
    tagging=on
  }
  \partokencontext=0 % TEMP FIX for https://github.com/schlcht/microtype/issues/59

\documentclass{article}
\usepackage{picinpar}

\font\yn=cmss17 scaled \magstep5
\setlength{\parindent}{0pt}

\title{picinpar tagging test}

\begin{document}

\begin{window}[0,l,{\yn V},{}]
or einigen Jahren wurde von Donald E.~Knuth im TUGboat ein kleines
Problem mit der Bitte um L"osung vorgestellt. Es handelte sich darum,
in einem Paragraphen ein Fenster zu erzeugen, in das man beliebigen Text
oder eine Zeichnung hineinsetzen kann. Prompt kamen dann in den folgenden
Ausgaben L"osungsvorschl"age: Einer von DEK pers"onlich, der andere von
Alan Hoenig. Der letztgenannte brachte die elegantere L"osung, die keine
manuellen Korrekturen mehr notwendig machte. Sein Makro verlangte lediglich
in den Parametern Informationen "uber die Breite und H"ohe der
freizulassenden Stelle im Paragraphen. Die Einz"uge und der Satz der
Fragmente des Abschnitts erfolgten automatisch.
\end{window}

\end{document}