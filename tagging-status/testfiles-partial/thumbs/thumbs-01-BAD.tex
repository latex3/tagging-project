\DocumentMetadata
  {
    lang=en-GB,
    pdfversion=2.0,
    pdfstandard=ua-2,
    tagging=on,
  }
%% This is file `thumbs-example.tex',
\documentclass[a4paper,twoside,british]{article}[2025/01/22]% v1.4n Standard LaTeX document class
\usepackage{lipsum}[2021-09-20]%  v2.7 150 paragraphs of Lorem Ipsum dummy text
%\usepackage{eurosym}[1998/08/06]% v1.1 European currency symbol ''Euro''
\usepackage[extension=pdf,%
 pdfpagelayout=TwoPageRight,pdfpagemode=UseThumbs,%
 plainpages=false,pdfpagelabels=true,%
 hyperindex=false,%
 pdflang={en},%
 pdftitle={thumbs package example},%
 pdfauthor={H.-Martin Muench},%
 pdfsubject={Example for the thumbs package},%
 pdfkeywords={LaTeX, thumbs, thumb marks},%
 pdfview=Fit,pdfstartview=Fit,%
 linktoc=all]{hyperref}[2025-07-12]% v7.01o Hypertext links for LaTeX
\usepackage[a4paper]{geometry}[2020/01/02]% v5.9 Page Geometry
 % % For testing with package crop,
 % % you can use a4 paper layout and print on a3 paper;
 % % for real production probably "frame" would be removed from the crop options:
 % \usepackage[cross,cam,frame,width=297truemm,height=420truemm,center,axes]{crop}[2017/11/19]% v1.10 Crop marks (MF)
\IfPackageLoadedTF{crop}{%
   % In contrast to the "normal" example  the offsets must be regarded for use with the crop package:
   \PassOptionsToPackage{ignorehoffset=false,ignorevoffset=false}{thumbs}%  crop
 }{\PassOptionsToPackage{ignorehoffset=true,ignorevoffset=true}{thumbs}% no crop
  }
\usepackage[linefill=dots,thumblink=rule,minheight={47pt},height={auto},width={auto},%
            distance={2mm},topthumbmargin={auto},bottomthumbmargin={auto},%
            eventxtindent={5pt},oddtxtexdent={5pt},%
            evenmarkindent={0pt},oddmarkexdent={0pt},evenprintvoffset={0pt},%
            leftindexruleplus={1em},%
            % unused options: frame, frameinnercolor
            txtcentered=false,%
            nophantomsection=false,plainthumbsoverview=false,%
            verbose=true,final=true, hidethumbs=false,%
            % unused, obsolete options:
            %  pagecolor:  instead the pagecolor package is used.
            %  evenindent: was replaced by eventxtindent.
            %  oddexdent:  was replaced by oddtxtexdent.
            ]{thumbs}[2026-01-18]% v1.1d Thumb marks and overview page(s) (HMM)
\nopagecolor% use \pagecolor{white} if \nopagecolor does not work
\IfPackageLoadedF{hyperref}{\usepackage{url}[2013/09/16]% v3.4 Verb mode for urls etc.
}

\DeclareMathAlphabet{\mathup}{OT1}{\familydefault}{m}{n}
\DeclareRobustCommand{\unit}[1]{\ensuremath{\mathord{\thinspace\mathup{#1}}}}

\xdef\thetextchangingcolour{red}
\colorlet{textchangingcolour}{\thetextchangingcolour}

\xdef\themarkchangingcolour{green}
\colorlet{markchangingcolour}{\themarkchangingcolour}

\DeclareRobustCommand{\thethumbchangingtextE}{\Huge{\textbf{E}}}
\DeclareRobustCommand{\thethumbchangingtexte}{\Huge{\textbf{e}}}
\gdef\thumbchangingtext{\thethumbchangingtextE}

\listfiles
\begin{document}
\pagenumbering{arabic}
\label{FirstPage}
\section*{Example for thumbs}
\addcontentsline{toc}{section}{Example for thumbs}
\markboth{Example for thumbs}{Example for thumbs}

This example demonstrates the most common uses of package \textsf{thumbs}, v1.1d as of 2026-01-18.
The used options were\newline
\texttt{linefill=dots}, \texttt{thumblink=rule}, \texttt{minheight=\{47pt\}},
\texttt{height=auto}, \texttt{width={auto}}, \newline
\texttt{distance=\{2mm\}}, \texttt{topthumbmargin=\{auto\}}, \texttt{bottomthumbmargin=\{auto\}}, \newline
\texttt{eventxtindent=\{5pt\}}, \texttt{oddtxtexdent=\{5pt\}},
\texttt{evenmarkindent=\{0pt\}}, \texttt{oddmarkexdent=\{0pt\}}, \newline
\texttt{evenprintvoffset=\{0pt\}},
\texttt{leftindexruleplus=\{1em\}},
\texttt{txtcentered=false}, \newline
\texttt{nophantomsection=false},
\texttt{plainthumbsoverview=false},
\texttt{verbose=true}, \texttt{final=true}, and \newline
\texttt{hidethumbs=false}. \newline

These are the default options, except
\texttt{leftindexruleplus=\{1em\}} (default: \texttt{0pt}) and  \newline
\texttt{verbose=true}.\newline

If package \textsf{crop} was loaded, options
\texttt{ignorehoffset=true} and \texttt{ignorevoffset=true} were passed to the \textsf{thumbs} package,
otherwise \texttt{ignorehoffset=false} and \texttt{ignorevoffset=false}. \newline

Options \texttt{frame} and \texttt{frameinnercolor} were not used.\newline

The obsolete options \texttt{pagecolor} (instead the \texttt{pagecolor} package is used),
\texttt{evenindent} (was replaced by \texttt{eventxtindent}), and
\texttt{oddexdent} (was replaced by \texttt{oddtxtexdent}) were not used. \newline

For more details please see the documentation!\bigskip

\textbf{Hyperlinks or not:} If the \textsf{hyperref} package is loaded,
the references in the overview page for the thumb marks are also hyperlinked
(except when option \texttt{thumblink=none} is used).\bigskip

\noindent\textbf{For testing purpose pages \pageref{greenpage}f. have been completely coloured!}\newline
\textbf{Better exclude those two pages from printing\ldots}\bigskip

Some thumb mark texts are too large for the thumb mark by intention
(especially when the paper size and therefore also the thumb mark size
is decreased). When option \texttt{width=\{autoauto\}} would be used,
the thumb mark width would be automatically increased.
Please see page~\pageref{HugeText} for details!\bigskip

For printing this example to another format of paper (e.\,g.~A4) it is necessary to add the according
option (e.\,g.~\verb|a4paper|) to the document class and recompile it! (In that case the
thumb marks column change will occur at another point, of course.)
With paper format equal to document format the document can be printed without adapting the size,
like e.\,g.~\textquotedblleft shrink to page\textquotedblright . That would add a white border around
the document and by moving the thumb marks from the edge of the paper they no longer appear
at the side of the stack of paper. It is also necessary to use a printer capable of printing up to the
border of the sheet of paper. Alternatively it is possible after the printing to cut the paper to the
right size. While performing this manually is probably quite cumbersome, printing houses use paper,
which is slightly larger than the desired format, and afterwards cut it to format.
\newpage

\addtitlethumb{Frontmatter}{0}{white}{gray}{FirstPage}

At the first page no thumb mark was used, but we want to begin with thumb marks
at the first page, therefore a
\begin{verbatim}
\addtitlethumb{Frontmatter}{0}{white}{gray}{FirstPage}
\end{verbatim}
was used at the beginning of this page. This does not place a thumb mark at that page,
but the pagenumber and link{-}target in the Table of Thumbs point to the first page.
The label \verb|FirstPage| needs to be placed manually on the first page, of course!
If manually no label can be placed on the first page, the label \texttt{pagesLTS.0}
from the \textsf{pageslts} package (needs to be loaded explicitly!) can be used.
\newpage
\tableofcontents
\newpage

To include an overview page for the thumb marks,
\begin{verbatim}
\addthumbsoverviewtocontents{section}{Thumb marks overview}%
\thumbsoverview{Table of Thumbs}%
\end{verbatim}
is used, where \verb|\addthumbsoverviewtocontents| adds the thumb marks overview page to the table of contents.
(\textquotedblleft Edge index\textquotedblright{} would be the correct/{}formal term, see the Introduction in the manual.)
\smallskip

Generally it is desirable to have a hyperlink from the thumbs overview page to lead to the thumb mark
and not to some earlier place. Therefore automatically a~\verb|\phantomsection| is placed before each thumb mark.
But for example when using the thumb mark after a~\verb|\chapter{...}| command, it is probably nicer to have
the link point at the top of that chapter's title (instead of the line below it). The automatic placing of the
\verb|\phantomsection| can be disabled either globally by using option \texttt{nophantomsection},
or locally for the next thumb mark by the command \verb|\thumbsnophantom|. (When disabled globally,
still manual use of \verb|\phantomsection| is possible.)

\addthumbsoverviewtocontents{section}{Thumb marks overview}%
\thumbsoverview{Table of Thumbs}%

\noindent Those were the overview pages for the thumb marks.
\newpage

\section{The first section}
\addthumb{First section}{%
\space\Huge{\textbf{$1\textsuperscript{st}$}}}{yellow}{green}

\begin{verbatim}
\addthumb{First section}{%
\space\Huge{\textbf{$1\textsuperscript{st}$}}}{yellow}{green}
\end{verbatim}

A thumb mark is added for this section. The parameters are: title for the thumb mark,
the text to be displayed in the thumb mark (choose your own format),
the colour of the text in the thumb mark,
and the background colour of the thumb mark (parameters in this order).\newline

Now for some pages of \textquotedblleft content\textquotedblright\ldots

\newpage
\lipsum[1]
\newpage
\lipsum[1]
\newpage
\lipsum[1]
\newpage

\section{The second section}
\addthumb{Second section}{\Huge{\textbf{\arabic{section}}}}{green}{yellow}

For this section, the text to be displayed in the thumb mark was set to
\begin{verbatim}
\Huge{\textbf{\arabic{section}}}
\end{verbatim}
i.\,e. the number of the section will be displayed (huge \& bold).\newline

Let us change the thumb mark on a page with an even number:
\newpage

\section{The third section}
\addthumb{Third section}{\Huge{\textbf{\arabic{section}}}}{blue}{red}

No problem!

And you do not need to have a section to add a thumb:
\newpage

\addthumb{Still third section}{\Huge{\textbf{\arabic{section}b}}}{red}{blue}

This is still the third section, but there is a new thumb mark.

On the other hand, you can even get rid of the thumb marks for some page(s):
\newpage

\stopthumb

\noindent The command
\begin{verbatim}
\stopthumb
\end{verbatim}
was used here. Until another \verb|\addthumb| (with parameters) or
\begin{verbatim}
\continuethumb
\end{verbatim}
is used, there will be no more thumb marks.
\newpage
Still no thumb marks.\newpage
Still no thumb marks.\newpage
Still no thumb marks.\newpage

\continuethumb

Thumb mark continued (unchanged).\newpage
Thumb mark continued (unchanged).\newpage

Time for another thumb,

\addthumb{Another heading}{Small text}{white}{black}

and another.\bigskip

\addthumb{Huge Text paragraph}{\Huge{Huge\newline Text}}{yellow}{green}

\textquotedblleft {\Huge{Huge Text}}\textquotedblright\ is too wide for the thumb mark. When option
\texttt{width=\{autoauto\}} would be used, the thumb mark width would be automatically increased.
Now the text is either split over two lines (try \verb|Huge\newline Text| for another format)
or (in case \verb|Huge~Text| is used) is written over the border of the thumb mark. When the text is
too wide for the thumb mark and cannot be split, \LaTeX{} might nevertheless place the text into the
next line. By this the text is placed too low. Adding a
\hbox{\verb|\protect\vspace*{-| some length \verb|}|} to the text could help, for example\newline
\verb|\addthumb{Huge Text}{\protect\vspace*{-3pt}\Huge{Huge~Text}}...|.\label{HugeText}\bigskip

\addthumb{Huge Text}{\Huge{Huge~Text}}{red}{blue}
\addthumb{Huge Bold Text}{\Huge{\textbf{HBT}}}{black}{yellow}

\noindent When there is more than one thumb mark at one page, this is also no problem.\newpage
Some text\newpage
Some text\newpage
Some text\newpage

\section{xcolor}
\addthumb{xcolor}{\Huge{\textbf{xcolor}}}{magenta}{cyan}

It is probably a good idea to have a look at the \textsf{xcolor} package and use other colours than used
in this example.\newline
(About automatically increasing the thumb mark width to the thumb mark text width please see the note
at page~\pageref{HugeText}.)
\newpage

\addthumb{A mark}{\Huge{\textbf{A}}}{lime}{darkgray}

I just need to add further thumb marks to get them reaching the bottom of the page.\newline
Generally the vertical size of the thumb marks is set to the value given in the height option.
If it is \texttt{auto}, the size of the thumb marks is decreased, so that they fit all on one page.
But when they get smaller than \texttt{minheight}, instead of decreasing their size further,
a~new thumbs column is started (which will happen here).
\newpage

\addthumb{B mark}{\Huge{\textbf{B}}}{brown}{pink}

Another thumb mark.\newpage

\addthumb{C mark}{\Huge{\textbf{C}}}{brown}{pink}

There! A new thumb column was started automatically!

On the one hand you can, of course, keep the colour for more than one thumb mark.
On the other hand you can change the colour of an existing thumb mark,
shown at page~\pageref{E mark}f.\newpage

\addthumb{$1/1.\,955\,83$\, EUR}{\Huge{\textbf{D}}}{orange}{violet}

\noindent I am just adding further thumb marks.

\noindent If you are curious why the thumb mark between
\textquotedblleft C mark\textquotedblright\ and \textquotedblleft E mark\textquotedblright\ has
not been named \textquotedblleft D mark\textquotedblright\ but
\textquotedblleft $1/1.\,955\,83$\, EUR\textquotedblright :

$1\unit{DM}=1\unit{D\ Mark}=1\unit{Deutsche\ Mark}%
=\frac{1}{1.\,955\,83}\,$\euro$\,=1/1.\,955\,83\unit{Euro}=1/1.\,955\,83\unit{EUR}$.
\newpage
Just a page.
\newpage

\xdef\thetextchangingcolour{red}%
\colorlet{textchangingcolour}{\thetextchangingcolour}%

\xdef\themarkchangingcolour{green}%
\colorlet{markchangingcolour}{\themarkchangingcolour}%

  %% In the preamble:
  %%\DeclareRobustCommand{\thethumbchangingtextE}{\Huge{\textbf{E}}}
  %%\DeclareRobustCommand{\thethumbchangingtexte}{\Huge{\textbf{e}}}
  %%\gdef\thumbchangingtext{\thethumbchangingtextE}

\gdef\thumbchangingtext{\thethumbchangingtextE}%

\addthumb{E mark}{\protect\thumbchangingtext}{textchangingcolour}{markchangingcolour}
\label{E mark}

As announced above, it is not only possible to keep the colour for more than one thumb mark,
but also to change the colour(s) and the text of an existing thumb mark. Here a thumb mark added with
\begin{verbatim}
\addthumb{Name}{\protect\thumbchangingtext}{textchangingcolour}{%
markchangingcolour}
\end{verbatim}
appears as red text \textquotedblleft E\textquotedblright{} on green background,
but really it is not \texttt{red} but \texttt{textchangingcolour} (which was defined as \texttt{red})
and not \texttt{green} but\newline
\texttt{markchangingcolour} (which was defined as \texttt{green}). Also the text of a thumb mark
can be changed. Here \verb|thumbchangingtext| was used as text, which was defined as
\verb|thethumbchangingtextE|, which was predefined in the preamble:
\begin{verbatim}
\DeclareRobustCommand{\thethumbchangingtextE}{\Huge{\textbf{E}}}
\end{verbatim}
\newpage

\noindent Some pages(s) later:

\begin{verbatim}
\xdef\thetextchangingcolour{blue}%
\colorlet{textchangingcolour}{\thetextchangingcolour}%

\xdef\themarkchangingcolour{yellow}%
\colorlet{markchangingcolour}{\themarkchangingcolour}%

\gdef\thumbchangingtext{\thethumbchangingtexte}%
\end{verbatim}

\xdef\thetextchangingcolour{blue}%
\colorlet{textchangingcolour}{\thetextchangingcolour}%

\xdef\themarkchangingcolour{yellow}%
\colorlet{markchangingcolour}{\themarkchangingcolour}%

\gdef\thumbchangingtext{\thethumbchangingtexte}%

And here the thumb mark still uses \texttt{textchangingcolour} and\newline
\texttt{markchangingcolour}, but instead of \texttt{red} and \texttt{green}
those have been redefined to \texttt{blue} and \texttt{yellow}.
\verb|\thumbchangingtext| was still used as text, but redefined as \verb|\thethumbchangingtexte|,
which was predefined in the preamble:
\begin{verbatim}
\DeclareRobustCommand{\thethumbchangingtexte}{\Huge{\textbf{e}}}
\end{verbatim}
Instead of \verb|\thumbchangingtext| also some variable like \verb|\thepage| can be used.\newline
For the thumb marks overview page(s) it must be decided which colour(s) and text
to use and the according definitions have to be made before that page{/}those pages!
Otherwise inconsistencies or even \textquotedblleft\texttt{undefined}\textquotedblright{-}errors
would result.
\newpage
(This page was intentionally left blank.)
\newpage

To prevent inconsistencies between the different thumb mark overviews,
we redefine \texttt{textchangingcolour} to \texttt{red},
\texttt{markchangingcolour} to \texttt{green}, and
\verb|\thumbchangingtext| to \verb|\thethumbchangingtextE| again here.

\xdef\thetextchangingcolour{red}%
\colorlet{textchangingcolour}{\thetextchangingcolour}%

\xdef\themarkchangingcolour{green}%
\colorlet{markchangingcolour}{\themarkchangingcolour}%

\gdef\thumbchangingtext{\thethumbchangingtextE}%

\newpage

\makeatletter%
\ifx\thumbs@frame\thumbs@empty\relax%
  \gdef\thumbs@frame{2pt}%
  \gdef\th@mbs@fr@meinnercolor{\thepagecolor}% could also be another colour
  \gdef\thumbs@checkforexample{0}%
\else% somebody changed the example to use the frame option:
  \gdef\thumbs@checkforexample{1}%
  \PackageInfo{thumbs}{The frame option is already used,\MessageBreak%
                                    not changing here anything!}%
\fi%
\makeatother%

\textsc{Bjorn Victor} suggested to use thumb marks consisting only of frames
instead of filled rectangles. Options \texttt{frame} and \texttt{frameinnercolor} were introduced
to accomplish this. While these options have not been used in this example, it is possible to trick
\LaTeX{} into believing that they had been used. (Do not use this for a regular document!)
This is just for demonstrating how a thumb mark consisting just of a frame with
$2\unit{pt}$~thick lines would look like\newpage
\noindent on left and right pages.\newpage

Now let us restore the filled thumb marks again

\makeatletter%
\ifx\thumbs@checkforexample\thumbs@zero\relax%
  \gdef\thumbs@frame{\empty}%
  \gdef\th@mbs@fr@meinnercolor{\thepagecolor}%
\fi%
\makeatother%

and then have a look at \verb|\thumbsoverviewverso|:

\addthumbsoverviewtocontents{section}{Table of Thumbs, verso mode}%
\thumbsoverviewverso{Table of Thumbs, verso mode}

\newpage

And, of course, also at \verb|\thumbsoverviewdouble|:

\addthumbsoverviewtocontents{section}{Table of Thumbs, double mode}%
\thumbsoverviewdouble{Table of Thumbs, double mode}

\newpage

\addthumb{F mark}{\Huge{\textbf{F}}}{lightgray}{black}

I am just adding further thumb marks.
\newpage

\addthumb{G mark}{\Huge{\textbf{G}}}{magenta}{black}

Some text.

\newpage
\thumbnewcolumn
\addthumb{New thumb marks column}{\Huge{\textit{NC}}}{magenta}{black}

There! A new thumb column was started manually!

\newpage
Some text.
\newpage

\addthumb{H mark}{\Huge{\textbf{H}}}{orange}{violet}

I just added another thumb mark.
\newpage
The following two pages are completely coloured in green.
\newpage
\pagecolor{green}

\IfPackageLoadedT{hyperref}{\phantomsection}
\label{greenpage}

Some faulty, outdated version of a pdf-viewer sometimes (for the same document!)
added a white line at the bottom and right side of the document when presenting it. This did
not change the printed version. To test for this problem, this doublepage has been completely coloured.
(Probably better exclude these two pages from printing!)\newline

\textsc{Heiko Oberdiek} wrote at Tue, 26 Apr 2011 14:13:29 +0200 in the \newline
comp.text.tex newsgroup (see e.\,g.\newline
\url{https://groups.google.com/g/de.comp.text.tex/c/4GAcL6qlTZ8/m/NzccDqakrrMJ}):\newline
\textquotedblleft Der Ursprung ist 0 0, da gibt es nicht viel zu runden; bei den anderen
Seiten werden pt als bp in die PDF-Datei geschrieben, d.\,h.~der Balken ist um 72.27/72 zu
gro\ss{}, das sollte auch Rundungsfehler abdecken.\textquotedblright\newline

(The origin is 0 0, there is not much to be rounded; for the other sides the $\unit{pt}$
are written as $\unit{bp}$ into the pdf-file, i.\,e.~the rule is too large by $72.27/72$,
which should cover also rounding errors.)\newline

The thumb marks are also too wide -- on purpose! This has been done to assure,
that they cover the page up to its (paper) border, therefore they are placed
a little bit over the paper margin.\newline

Now I red somewhere in the net (should have remembered to note the url),
that white margins are presented, whenever there is some object outside of the page.
Thus, it is a feature, not a bug?!
What I do not understand: The same document sometimes is presented
with white lines and sometimes without (same viewer, same PC).\newline
But at least it does not influence the printed version.
\newpage
Another green page.
\newpage
\pagecolor{white}

It is possible to use the Table of Thumbs more than once (for example at the beginning and
at the end of the document) and to refer to them via e.\,g.~\verb|\pageref{TableOfThumbs1}|,\linebreak
\verb|\pageref{TableOfThumbs2}|,\ldots , here: page~\pageref{TableOfThumbs1},
page~\pageref{TableOfThumbs2}, and via e.\,g.~\verb|\pageref{TableOfThumbs}|
it is referred to the last used Table of Thumbs (for compatibility with older package versions).
If there is only one Table of Thumbs, this one is also the last one, of course.
Here it is at page~\pageref{TableOfThumbs}.\newline

Now let us have a look at \verb|\thumbsoverviewback|:

\addthumbsoverviewtocontents{section}{Table of Thumbs, back mode}%
\thumbsoverviewback{Table of Thumbs, back mode}%
\newpage

Text can be placed after any of the Tables of Thumbs, of course.
\end{document}
\endinput
%%
%% End of file `thumbs-example.tex'.
