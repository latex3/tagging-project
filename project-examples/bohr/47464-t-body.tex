% DPC: · replaced by \cdot ± replaced by \pm ° replaced by \textdegree
%% PG BOILERPLATE %%
\PGBoilerPlate
\begin{center}
\begin{minipage}{\textwidth}
\small
\begin{PGtext}
The Project Gutenberg EBook of The Theory of Spectra and Atomic
Constitution, by Niels (Niels Henrik David) Bohr

This eBook is for the use of anyone anywhere in the United States and
most other parts of the world at no cost and with almost no
restrictions whatsoever.  You may copy it, give it away or re-use it
under the terms of the Project Gutenberg License included with this
eBook or online at www.gutenberg.org.  If you are not located in the
United States, you'll have to check the laws of the country where you
are located before using this ebook.



Title: The Theory of Spectra and Atomic Constitution
       Three Essays

Author: Niels (Niels Henrik David) Bohr

Release Date: November 26, 2014 [EBook #47464]
Most recently updated: June 11, 2021

Language: English

Character set encoding: UTF-8

Modifications to the TeX Markup to produce Tagged PDF
LaTeX project, March 2004

*** START OF THIS PROJECT GUTENBERG EBOOK THEORY OF SPECTRA ***
\end{PGtext}
\end{minipage}
\end{center}
\newpage
%% Credits and transcriber's note %%
\begin{center}
\begin{minipage}{\textwidth}
\begin{PGtext}
Produced by Andrew D. Hwang
\end{PGtext}
\end{minipage}
\vfill
\end{center}

%UF do not put sectioning command inside a box ...
%\begin{minipage}{0.85\textwidth}
%\begin{small}
\BookMark{0}{Transcriber's Note.}
\subsection*{\centering\normalfont\scshape%
\normalsize\MakeLowercase{\TransNote}}%

{\small\raggedright
 \TransNoteText

 }
%\end{minipage}
%%%%%%%%%%%%%%%%%%%%%%%%%%% FRONT MATTER %%%%%%%%%%%%%%%%%%%%%%%%%%
\PageSep{i}
\iffalse
% [** TN: Omitting half-title page and verso]
The Theory of Spectra
and
Atomic Constitution
\PageSep{ii}
%[** TN: Publisher's information]

CAMBRIDGE UNIVERSITY PRESS
C. F. CLAY, Manager
LONDON : FETTER LANE, E.C. 4

%[** TN: Publisher's device]

LONDON : H. K. LEWIS AND CO., Ltd.,
136 Gower Street, W.C. 1
NEW YORK : THE MACMILLAN CO.
BOMBAY   }
CALCUTTA } MACMILLAN AND CO., Ltd.
MADRAS   }
TORONTO : THE MACMILLAN CO. OF
CANADA, Ltd.
TOKYO : MARUZEN-KABUSHIKI-KAISHA

ALL RIGHTS RESERVED
\fi
%[** TN: End of omitted half-title]
\PageSep{iii}
\newpage
\begin{center}
\Huge\bfseries
The Theory of Spectra \\
and \\
Atomic Constitution
\bigskip

\large\normalfont
THREE ESSAYS \\
BY \\
\Large
NIELS BOHR
\medskip

\normalsize
Professor of Theoretical Physics in~the~University~of~Copenhagen
\vfill

\Large
CAMBRIDGE \\
AT THE UNIVERSITY PRESS \\
1922
\end{center}
\newpage
\PageSep{iv}
\ifthenelse{\boolean{ForPrinting}}{% Publisher's verso
\begin{center}
\null\vfill
\footnotesize
PRINTED IN GREAT BRITAIN \\
AT THE CAMBRIDGE UNIVERSITY PRESS
\end{center}
}{}% Omit for screen-formatted version
\PageSep{v}

\FrontMatter

\Preface

\First{The} three essays which here appear in English all deal with
the application of the quantum theory to problems of atomic
structure, and refer to the different stages in the development of
this theory.

The first essay ``On the spectrum of hydrogen'' is a translation of
a Danish address given before the Physical Society of Copenhagen
on the 20th~of December 1913, and printed in \Title{Fysisk Tidsskrift},
\No{xii.}\ p.~97, 1914. Although this address was delivered at a time
when the formal development of the quantum theory was only at
its beginning, the reader will find the general trend of thought
very similar to that expressed in the later addresses, which
form the other two essays. As emphasized at several points the
theory does not attempt an ``explanation'' in the usual sense of
this word, but only the establishment of a connection between facts
which in the present state of science are unexplained, that is to
say the usual physical conceptions do not offer sufficient basis for
a detailed description.

The second essay ``On the series spectra of the elements'' is a
translation of a German address given before the Physical Society
of Berlin on the 27th~of April 1920, and printed in \Title{Zeitschrift für
Physik}, \No{vi.}\ p.~423, 1920. This address falls into two main parts.
The considerations in the first part are closely related to the contents
of the first essay; especially no use is made of the new
formal conceptions established through the later development of
the quantum theory. The second part contains a survey of the
results reached by this development. An attempt is made to
elucidate the problems by means of a general principle which postulates
a formal correspondence between the fundamentally different
conceptions of the classical electrodynamics and those of the
quantum theory. The first germ of this correspondence principle
may be found in the first essay in the deduction of the expression
for the constant of the hydrogen spectrum in terms of
Planck's constant and of the quantities which in Rutherford's
\PageSep{vi}
atomic model are necessary for the description of the hydrogen
atom.

The third essay ``The structure of the atom and the physical
and chemical properties of the elements'' is based on a Danish
address, given before a joint meeting of the Physical and Chemical
Societies of Copenhagen on the 18th~of October 1921, and printed
in \Title{Fysisk Tidsskrift}, \No{xix.}\ p.~153, 1921. While the first two essays
form verbal translations of the respective addresses, this essay
differs from the Danish original in certain minor points. Besides
the addition of a few new figures with explanatory text, certain
passages dealing with problems discussed in the second essay are
left out, and some remarks about recent contributions to the
subject are inserted. Where such insertions have been introduced
will clearly appear from the text. This essay is divided into
four parts. The first two parts contain a survey of previous results
concerning atomic problems and a short account of the theoretical
ideas of the quantum theory. In the following parts it is shown
how these ideas lead to a view of atomic constitution which seems
to offer an explanation of the observed physical and chemical
properties of the elements, and especially to bring the characteristic
features of the periodic table into close connection with the
interpretation of the optical and high frequency spectra of the
elements.

For the convenience of the reader all three essays are subdivided
into smaller paragraphs, each with a headline. Conforming to the
character of the essays there is, however, no question of anything
like a full account or even a proportionate treatment of the subject
stated in these headlines, the principal object being to emphasize
certain general views in a freer form than is usual in scientific
treatises or text books. For the same reason no detailed references
to the literature are given, although an attempt is made to mention
the main contributions to the development of the subject. As
regards further information the reader in the case of the second
essay is referred to a larger treatise ``On the quantum theory of
line spectra,'' two parts of which have appeared in the Transactions of
the Copenhagen Academy (\Title{D.\ Kgl.\ Danske Vidensk.\ Selsk.\ Skrifter},
8.\ Række, \No{iv.}~1, I~and~II, 1918),\footnote
  {See \href{http://www.gutenberg.org/ebooks/47167}{www.gutenberg.org/ebooks/47167}.---\textit{Trans.}}
where full references to the literature
may be found. The proposed continuation of this treatise, mentioned
\PageSep{vii}
at several places in the second essay, has for various reasons been
delayed, but in the near future the work will be completed by the
publication of a third part. It is my intention to deal more fully
with the problems discussed in the third essay by a larger systematic
account of the application of the quantum theory to atomic
problems, which is under preparation.

As mentioned both in the beginning and at the end of the
third essay, the considerations which it contains are clearly still
incomplete in character. This holds not only as regards the
elaboration of details, but also as regards the development of the
theoretical ideas. It may be useful once more to emphasize,
that---although the word ``explanation'' has been used more
liberally than for instance in the first essay---we are not concerned
with a description of the phenomena, based on a well-defined
physical picture. It may rather be said that hitherto every
progress in the problem of atomic structure has tended to emphasize
the well-known ``mysteries'' of the quantum theory more and more.
I hope the exposition in these essays is sufficiently clear, nevertheless,
to give the reader an impression of the peculiar charm
which the study of atomic physics possesses just on this account.

I wish to express my best thanks to Dr~A.~D. Udden, University
of Pennsylvania, who has undertaken the translation of the
original addresses into English, and to Mr~C.~D. Ellis, Trinity
College, Cambridge, who has looked through the manuscript and
suggested many valuable improvements in the exposition of the
subject.
\Signature{N. BOHR.}{Copenhagen,}{May}{1922.}
\PageSep{viii}

%UF TODO: breaks tagging, as there are manual toc lines.
%\TableofContents

%
\iffalse
%[** TN: Original ToC text (not manually verified)]
CONTENTS

ESSAY I
ON THE SPECTRUM OF HYDROGEN

PAGE

Empirical Spectral Laws 1
Laws of Temperature Radiation 4
The Nuclear Theory of the Atom 7
Quantum Theory of Spectra 10
Hydrogen Spectrum 12
The Pickering Lines 15
Other Spectra 18


ESSAY II
ON THE SERIES SPECTRA OF THE ELEMENTS

I. Introduction .20

II. General Principles of the Quantum Theory of Spectra . 23
Hydrogen Spectrum 24
The Correspondence Principle 27
General Spectral Laws 29
Absorption and Excitation of Radiation 32

III. Development of the Quantum Theory of Spectra . . 36
Effect of External Forces on the Hydrogen Spectrum . . 37
The Stark Effect 39
The Zeeman Effect 42
Central Perturbations . 44
Relativity Effect on Hydrogen Lines 46
Theory of Series Spectra 48
Correspondence Principle and Conservation of Angular Momentum 50
The Spectra of Helium and Lithium 54
Complex Structure of Series Lines 58

IV. Conclusion 59

\PageSep{ix}
CONTENTS

ESSAY III

THE STRUCTURE OF THE ATOM AND THE PHYSICAL
AND CHEMICAL PROPERTIES OF THE ELEMENTS

PAGE

I. Preliminary 61
The Nuclear Atom 61
The Postulates of the Quantum Theory 62
Hydrogen Atom 63
Hydrogen Spectrum and X-ray Spectra 65
The Fine Structure of the Hydrogen Lines .... 67
Periodic Table 69
Recent Atomic Models 74

II. Series Spectra and the Capture of Electrons by Atoms . 75
Arc and Spark Spectra 76
Series Diagram 78
Correspondence Principle 81

III. Formation of Atoms and the Periodic Table ... 85
First Period. Hydrogen---Helium 85
Second Period. Lithium---Neon 89
Third Period. Sodium---Argon 95
Fourth Period. Potassium---Krypton 100
Fifth Period. Rubidium--- Xenon 108
Sixth Period. Caesium---Niton 109
Seventh Period 111
Survey of the Periodic Table 113

IV. Reorganization of Atoms and X-ray Spectra . . .116
Absorption and Emission of X-rays and Correspondence Principle 117
X-ray Spectra and Atomic Structure 119
Classification of X-ray Spectra 121
Conclusion 125
\fi
%[** TN: End of original ToC text]
\PageSep{1}
\MainMatter

\Essay{I}{On the Spectrum of Hydrogen}
  {Address delivered before the Physical Society in Copenhagen, Dec.~20, 1913.}

\Section{Empirical spectral laws.} Hydrogen possesses not only the
smallest atomic weight of all the elements, but it also occupies a
peculiar position both with regard to its physical and its chemical
properties. One of the points where this becomes particularly apparent
is the hydrogen line spectrum.

The spectrum of hydrogen observed in an ordinary Geissler tube
consists of a series of lines, the strongest of which lies at the red
end of the spectrum, while the others extend out into the ultra\Add{-}violet,
the distance between the various lines, as well as their intensities,
constantly decreasing. In the ultra\Add{-}violet the series converges
to a limit.

Balmer, as we know, discovered (1885) that it was possible to
represent the wave lengths of these lines very accurately by the
simple law
\[
\frac{1}{\lambda_{n}} = R \left(\frac{1}{4} - \frac{1}{n^{2}}\right),
\Tag{(1)}
\]
where $R$~is a constant and $n$~is a whole number. The wave lengths
of the five strongest hydrogen lines, corresponding to $n = 3$, $4$,~$5$, $6$,~$7$,
measured in air at ordinary pressure and temperature, and the
values of these wave lengths multiplied by $\left(\dfrac{1}{4} - \dfrac{1}{n^{2}}\right)$ are given in
the following table:\Pagelabel{1}
\[
%[** TN: Original uses a period for multiplication and a center dot as a decimal point]
\begin{array}{*{2}{c<{\qquad\qquad}}c}
n & \lambda \cdot 10^{8} & \lambda \cdot \left(\dfrac{1}{4} - \dfrac{1}{n^{2}}\right) \cdot 10^{10} \\
3 & 6563.04 & 91153.3 \\
4 & 4861.49 & 91152.9 \\
5 & 4340.66 & 91153.9 \\
6 & 4101.85 & 91152.2 \\
7 & 3970.25 & 91153.7 \\
\end{array}
\]
The table shows that the product is nearly constant, while the deviations
are not greater than might be ascribed to experimental errors.

As you already know, Balmer's discovery of the law relating to
the hydrogen spectrum led to the discovery of laws applying to
the spectra of other elements. The most important work in this
\PageSep{2}
connection was done by Rydberg (1890) and Ritz (1908). Rydberg
pointed out that the spectra of many elements contain series of
lines whose wave lengths are given approximately by the formula
\[
\frac{1}{\lambda_{n}} = A - \frac{R}{(n + \alpha)^{2}},
\]
where $A$~and~$\alpha$ are constants having different values for the various
series, while $R$~is a universal constant equal to the constant in the
spectrum of hydrogen. If the wave lengths are measured in vacuo
Rydberg calculated the value of~$R$ to be~$109675$. In the spectra of
many elements, as opposed to the simple spectrum of hydrogen, there
are several series of lines whose wave lengths are to a close approximation
given by Rydberg's formula if different values are assigned to
the constants $A$~and~$\alpha$. Rydberg showed, however, in his earliest
work, that certain relations existed between the constants in the
various series of the spectrum of one and the same element. These
relations were later very successfully generalized by Ritz through
the establishment of the ``combination principle.'' According to
this principle, the wave lengths of the various lines in the spectrum
of an element may be expressed by the formula
\[
\frac{1}{\lambda} = F_{r}(n_{1}) - F_{s}(n_{2}).
\Tag{(2)}
\]
In this formula $n_{1}$~and~$n_{2}$ are whole numbers, and $F_{1}(n)$, $F_{2}(n)$,~\dots\ is
a series of functions of~$n$, which may be written approximately
\[
F_{r}(n) = \frac{R}{(n + \alpha_{r})^{2}},
\]
where $R$~is Rydberg's universal constant and $\alpha_{r}$ is a constant which
is different for the different functions. A particular spectral line will,
according to this principle, correspond to each combination of $n_{1}$~and~$n_{2}$,
as well as to the functions $F_{1}$, $F_{2}$,~\dots. The establishment of
this principle led therefore to the prediction of a great number of
lines which were not included in the spectral formulae previously
considered, and in a large number of cases the calculations were
found to be in close agreement with the experimental observations.
In the case of hydrogen Ritz assumed that formula~\Eq{(1)} was a special
case of the general formula
\[
\frac{1}{\lambda} = R\left(\frac{1}{n_{1}^{2}} - \frac{1}{n_{2}^{2}}\right),
\Tag{(3)}
\]
\PageSep{3}
and therefore predicted among other things a series of lines in the
infra\Add{-}red given by the formula
\[
\frac{1}{\lambda} = R\left(\frac{1}{9} - \frac{1}{n^{2}}\right).
\]
In 1909 Paschen succeeded in observing the first two lines of this
series corresponding to $n = 4$ and $n = 5$.

The part played by hydrogen in the development of our
knowledge of the spectral laws is not solely due to its ordinary
simple spectrum, but it can also be traced in other less direct
ways. At a time when Rydberg's laws were still in want of
further confirmation Pickering (1897) found in the spectrum of a
star a series of lines whose wave lengths showed a very simple relation
to the ordinary hydrogen spectrum, since to a very close
approximation they could be expressed by the formula
\[
\frac{1}{\lambda} = R\left(\frac{1}{4} - \frac{1}{(n + \frac{1}{2})^{2}}\right).
\]
Rydberg considered these lines to represent a new series of lines
in the spectrum of hydrogen, and predicted according to his theory
the existence of still another series of hydrogen lines the wave
lengths of which would be given by
\[
\frac{1}{\lambda} = R\left(\frac{1}{(\frac{3}{2})^{2}} - \frac{1}{n^{2}}\right).
\]
By examining earlier observations it was actually found that a line
had been observed in the spectrum of certain stars which coincided
closely with the first line in this series (corresponding to $n = 2$);
from analogy with other spectra it was also to be expected that this
would be the strongest line. This was regarded as a great triumph
for Rydberg's theory and tended to remove all doubt that the new
spectrum was actually due to hydrogen. Rydberg's view has therefore
been generally accepted by physicists up to the present moment.
Recently however the question has been reopened and Fowler
(1912) has succeeded in observing the Pickering lines in ordinary
laboratory experiments. We shall return to this question again
later.

The discovery of these beautiful and simple laws concerning the
line spectra of the elements has naturally resulted in many attempts
at a theoretical explanation. Such attempts are very alluring
\PageSep{4}
because the simplicity of the spectral laws and the exceptional accuracy
with which they apply appear to promise that the correct explanation
will be very simple and will give valuable information
about the properties of matter. I should like to consider some of
these theories somewhat more closely, several of which are extremely
interesting and have been developed with the greatest keenness
and ingenuity, but unfortunately space does not permit me to do
so here. I shall have to limit myself to the statement that not
one of the theories so far proposed appears to offer a satisfactory or
even a plausible way of explaining the laws of the line spectra.
Considering our deficient knowledge of the laws which determine
the processes inside atoms it is scarcely possible to give an explanation
of the kind attempted in these theories. The inadequacy of
our ordinary theoretical conceptions has become especially apparent
from the important results which have been obtained in recent years
from the theoretical and experimental study of the laws of temperature
radiation. You will therefore understand that I shall not
attempt to propose an explanation of the spectral laws; on the
contrary I shall try to indicate a way in which it appears possible
to bring the spectral laws into close connection with other properties
of the elements, which appear to be equally inexplicable on
the basis of the present state of the science. In these considerations
I shall employ the results obtained from the study of temperature
radiation as well as the view of atomic structure which has been
reached by the study of the radioactive elements.

\Section{Laws of temperature radiation.} I shall commence by mentioning
the conclusions which have been drawn from experimental
and theoretical work on temperature radiation.

Let us consider an enclosure surrounded by bodies which are in
temperature equilibrium. In this space there will be a certain
amount of energy contained in the rays emitted by the surrounding
substances and crossing each other in every direction. By making
the assumption that the temperature equilibrium will not be disturbed
by the mutual radiation of the various bodies Kirchhoff
(1860) showed that the amount of energy per unit volume as well
as the distribution of this energy among the various wave lengths
is independent of the form and size of the space and of the nature
\PageSep{5}
of the surrounding bodies and depends only on the temperature.
Kirchhoff's result has been confirmed by experiment, and the
amount of energy and its distribution among the various wave
lengths and the manner in which it depends on the temperature
are now fairly well known from a great amount of experimental
work; or, as it is usually expressed, we have a fairly
accurate experimental knowledge of the ``laws of temperature
radiation.''

Kirchhoff's considerations were only capable of predicting the
existence of a law of temperature radiation, and many physicists
have subsequently attempted to find a more thorough explanation
of the experimental results. You will perceive that the electromagnetic
theory of light together with the electron theory suggests
a method of solving this problem. According to the electron theory
of matter a body consists of a system of electrons. By making
certain definite assumptions concerning the forces acting on the
electrons it is possible to calculate their motion and consequently
the energy radiated from the body per second in the form of
electromagnetic oscillations of various wave lengths. In a similar
manner the absorption of rays of a given wave length by a substance
can be determined by calculating the effect of electromagnetic
oscillations upon the motion of the electrons. Having investigated
the emission and absorption of a body at all temperatures, and for
rays of all wave lengths, it is possible, as Kirchhoff has shown, to
determine immediately the laws of temperature radiation. Since
the result is to be independent of the nature of the body we are
justified in expecting an agreement with experiment, even though
very special assumptions are made about the forces acting upon
the electrons of the hypothetical substance. This naturally
simplifies the problem considerably, but it is nevertheless sufficiently
difficult and it is remarkable that it has been possible
to make any advance at all in this direction. As is well known
this has been done by Lorentz (1903). He calculated the
emissive as well as the absorptive power of a metal for long
wave lengths, using the same assumptions about the motions
of the electrons in the metal that Drude (1900) employed in
his calculation of the ratio of the electrical and thermal conductivities.
Subsequently, by calculating the ratio of the emissive
\PageSep{6}
to the absorptive power, Lorentz really obtained an expression
for the law of temperature radiation which for long wave lengths
agrees remarkably well with experimental facts. In spite of this
beautiful and promising result, it has nevertheless become apparent
that the electromagnetic theory is incapable of explaining the law
of temperature radiation. For, it is possible to show, that, if the
investigation is not confined to oscillations of long wave lengths,
as in Lorentz's work, but is also extended to oscillations corresponding
to small wave lengths, results are obtained which are
contrary to experiment. This is especially evident from Jeans'
investigations (1905) in which he employed a very interesting
statistical method first proposed by Lord Rayleigh.

We are therefore compelled to assume, that the classical electrodynamics
does not agree with reality, or expressed more carefully,
that it \Chg{can not}{cannot} be employed in calculating the absorption and
emission of radiation by atoms. Fortunately, the law of temperature
radiation has also successfully indicated the direction in which the
necessary changes in the electrodynamics are to be sought. Even
before the appearance of the papers by Lorentz and Jeans, Planck
(1900) had derived theoretically a formula for the black body radiation
which was in good agreement with the results of experiment.
Planck did not limit himself exclusively to the classical electrodynamics,
but introduced the further assumption that a system of
oscillating electrical particles (elementary resonators) will neither
radiate nor absorb energy continuously, as required by the ordinary
electrodynamics, but on the contrary will radiate and absorb discontinuously.
The energy contained within the system at any
moment is always equal to a whole multiple of the so-called
quantum of energy the magnitude of which is equal to~$h\nu$, where
$h$~is Planck's constant and $\nu$~is the frequency of oscillation of the
system per second. In formal respects Planck's theory leaves much
to be desired; in certain calculations the ordinary electrodynamics
is used, while in others assumptions distinctly at variance with it
are introduced without any attempt being made to show that it
is possible to give a consistent explanation of the procedure used.
Planck's theory would hardly have acquired general recognition
merely on the ground of its agreement with experiments on black
body radiation, but, as you know, the theory has also contributed
\PageSep{7}
quite remarkably to the elucidation of many different physical
phenomena, such as specific heats, photoelectric effect, X-rays and
the absorption of heat rays by gases. These explanations involve
more than the qualitative assumption of a discontinuous transformation
of energy, for with the aid of Planck's constant~$h$ it
seems to be possible, at least approximately, to account for a great
number of phenomena about which nothing could be said previously.
It is therefore hardly too early to express the opinion that, whatever
the final explanation will be, the discovery of ``energy quanta''
must be considered as one of the most important results arrived at
in physics, and must be taken into consideration in investigations
of the properties of atoms and particularly in connection with any
explanation of the spectral laws in which such phenomena as
the emission and absorption of electromagnetic radiation are
concerned.

\Section{The nuclear theory of the atom.} We shall now consider the
second part of the foundation on which we shall build, namely the
conclusions arrived at from experiments with the rays emitted by
radioactive substances. I have previously here in the Physical
Society had the opportunity of speaking of the scattering of $\alpha$~rays
in passing through thin plates, and to mention how Rutherford
(1911) has proposed a theory for the structure of the atom in
order to explain the remarkable and unexpected results of these
experiments. I shall, therefore, only remind you that the characteristic
feature of Rutherford's theory is the assumption of the
existence of a positively charged nucleus inside the atom. A number
of electrons are supposed to revolve in closed orbits around the
nucleus, the number of these electrons being sufficient to neutralize
the positive charge of the nucleus. The dimensions of the nucleus
are supposed to be very small in comparison with the dimensions
of the orbits of the electrons, and almost the entire mass of the
atom is supposed to be concentrated in the nucleus.


According to Rutherford's calculation the positive charge of the
nucleus corresponds to a number of electrons equal to about half
the atomic weight. This number coincides approximately with the
number of the particular element in the periodic system and it is
therefore natural to assume that the number of electrons in the
\PageSep{8}
atom is exactly equal to this number. This hypothesis, which was
first stated by van~den Broek (1912), opens the possibility of
obtaining a simple explanation of the periodic system. This assumption
is strongly confirmed by experiments on the elements
of small atomic weight. In the first place, it is evident that according
to Rutherford's theory the $\alpha$~particle is the same as the
nucleus of a helium atom. Since the $\alpha$~particle has a double positive
charge it follows immediately that a neutral helium atom contains
two electrons. Further the concordant results obtained from calculations
based on experiments as different as the diffuse scattering
of X-rays and the decrease in velocity of $\alpha$~rays in passing
through matter render the conclusion extremely likely that a
hydrogen atom contains only a single electron. This agrees most
beautifully with the fact that J.~J. Thomson in his well-known
experiments on rays of positive electricity has never observed a
hydrogen atom with more than a single positive charge, while all
other elements investigated may have several charges.

Let us now assume that a hydrogen atom simply consists of an
electron revolving around a nucleus of equal and opposite charge,
and of a mass which is very large in comparison with that of the
electron. It is evident that this assumption may explain the peculiar
position already referred to which hydrogen occupies among the
elements, but it appears at the outset completely hopeless to attempt
to explain anything at all of the special properties of hydrogen,
still less its line spectrum, on the basis of considerations relating
to such a simple system.

Let us assume for the sake of brevity that the mass of the nucleus
is infinitely large in proportion to that of the electron, and that the
velocity of the electron is very small in comparison with that of
light. If we now temporarily disregard the energy radiation, which,
according to the ordinary electrodynamics, will accompany the accelerated
motion of the electron, the latter in accordance with
Kepler's first law will describe an ellipse with the nucleus in one
of the foci. Denoting the frequency of revolution by~$\omega$, and the
major axis of the ellipse by~$2a$ we find that
\[
\omega^{2} = \frac{2W^{3}}{\pi^{2} e^{4} m},\quad
2a = \frac{e^{2}}{W},
\Tag{(4)}
\]
\PageSep{9}
where $e$~is the charge of the electron and $m$~its mass, while $W$~is
the work which must be added to the system in order to remove
the electron to an infinite distance from the nucleus.

These expressions are extremely simple and they show that the
magnitude of the frequency of revolution as well as the length of
the major axis depend only on~$W$, and are independent of the
\Chg{excentricity}{eccentricity} of the orbit. By varying~$W$ we may obtain all possible
values for $\omega$~and~$2a$. This condition shows, however, that it is not
possible to employ the above formulae directly in calculating the
orbit of the electron in a hydrogen atom. For this it will be necessary
to assume that the orbit of the electron \Chg{can not}{cannot} take on all values,
and in any event, the line spectrum clearly indicates that the
oscillations of the electron cannot vary continuously between wide
limits. The impossibility of making any progress with a simple
system like the one considered here might have been foretold from
a consideration of the dimensions involved; for with the aid of $e$~and
$m$~alone it is impossible to obtain a quantity which can be
interpreted as a diameter of an atom or as a frequency.

If we attempt to account for the radiation of energy in the manner
required by the ordinary electrodynamics it will only make matters
worse. As a result of the radiation of energy~$W$ would continually
increase, and the above expressions~\Eq{(4)} show that at the same time
the frequency of revolution of the system would increase, and the
dimensions of the orbit decrease. This process would not stop until
the particles had approached so closely to one another that they no
longer attracted each other. The quantity of energy which would
be radiated away before this happened would be very great. If we
were to treat these particles as geometrical points this energy would
be infinitely great, and with the dimensions of the electrons as
calculated from their mass (about $10^{-13}$~cm.), and of the nucleus as
calculated by Rutherford (about $10^{-12}$~cm.), this energy would be
many times greater than the energy changes with which we are
familiar in ordinary atomic processes.

It can be seen that it is impossible to employ Rutherford's atomic
model so long as we confine ourselves exclusively to the ordinary
electrodynamics. But this is nothing more than might have been
expected. As I have mentioned we may consider it to be an
established fact that it is impossible to obtain a satisfactory
\PageSep{10}
explanation of the experiments on temperature radiation with the
aid of electrodynamics, no matter what atomic model be employed.
The fact that the deficiencies of the atomic model we are
considering stand out so plainly is therefore perhaps no serious
drawback; even though the defects of other atomic models are
much better concealed they must nevertheless be present and will
be just as serious.

\Section{Quantum theory of spectra.} Let us now try to overcome these
difficulties by applying Planck's theory to the problem.

It is readily seen that there can be no question of a direct application
of Planck's theory. This theory is concerned with the emission
and absorption of energy in a system of electrical particles, which
oscillate with a given frequency per second, dependent only on the
nature of the system and independent of the amount of energy
contained in the system. In a system consisting of an electron and
a nucleus the period of oscillation corresponds to the period of
revolution of the electron. But the formula~\Eq{(4)} for~$\omega$ shows that the
frequency of revolution depends upon~$W$, \ie\ on the energy of the
system. Still the fact that we \Chg{can not}{cannot} immediately apply Planck's
theory to our problem is not as serious as it might seem to be, for
in assuming Planck's theory we have manifestly acknowledged the
inadequacy of the ordinary electrodynamics and have definitely
parted with the coherent group of ideas on which the latter theory
is based. In fact in taking such a step we \Chg{can not}{cannot} expect that all
cases of disagreement between the theoretical conceptions hitherto
employed and experiment will be removed by the use of Planck's
assumption regarding the quantum of the energy momentarily
present in an oscillating system. We stand here almost entirely on
virgin ground, and upon introducing new assumptions we need only
take care not to get into contradiction with experiment. Time will
have to show to what extent this can be avoided; but the safest
way is, of course, to make as few assumptions as possible.

With this in mind let us first examine the experiments on
temperature radiation. The subject of direct observation is the
distribution of radiant energy over oscillations of the various wave
lengths. Even though we may assume that this energy comes from
systems of oscillating particles, we know little or nothing about
\PageSep{11}
these systems. No one has ever seen a Planck's resonator, nor
indeed even measured its frequency of oscillation; we can observe
only the period of oscillation of the radiation which is emitted. It
is therefore very convenient that it is possible to show that to
obtain the laws of temperature radiation it is not necessary to
make any assumptions about the systems which emit the radiation
except that the amount of energy emitted each time shall be equal
to~$h\nu$, where $h$~is Planck's constant and $\nu$~is the frequency of the
radiation. Indeed, it is possible to derive Planck's law of radiation
from this assumption alone, as shown by Debye, who employed a
method which is a combination of that of Planck and of Jeans.
Before considering any further the nature of the oscillating systems
let us see whether it is possible to bring this assumption about the
emission of radiation into agreement with the spectral laws.

If the spectrum of some element contains a spectral line corresponding
to the frequency~$\nu$ it will be assumed that one of the
atoms of the element (or some other elementary system) can emit
an amount of energy~$h\nu$. Denoting the energy of the atom before
and after the emission of the radiation by $E_{1}$ and~$E_{2}$ we have
\[
h\nu = E_{1} - E_{2} \text{ or }
\nu = \frac{E_{1}}{h} - \frac{E_{2}}{h}.
\Tag{(5)}
\]

During the emission of the radiation the system may be regarded
as passing from one state to another; in order to introduce a name
for these states, we shall call them ``stationary'' states, simply
indicating thereby that they form some kind of waiting places
between which occurs the emission of the energy corresponding to
the various spectral lines. As previously mentioned the spectrum
of an element consists of a series of lines whose wave lengths may
be expressed by the formula~\Eq{(2)}. By comparing this expression
with the relation given above it is seen that---since $\nu = \dfrac{c}{\lambda}$, where $c$~is
the velocity of light---each of the spectral lines may be regarded
as being emitted by the transition of a system between two stationary
states in which the energy apart from an additive arbitrary
constant is given by $-ch F_{r}(n_{1})$ and $-ch F_{s}(n_{2})$ respectively. Using
this interpretation the combination principle asserts that a series
of stationary states exists for the given system, and that it can
\PageSep{12}
pass from one to any other of these states with the emission of
a monochromatic radiation. We see, therefore, that with a simple
extension of our first assumption it is possible to give a formal
explanation of the most general law of line spectra.

\Section{Hydrogen spectrum.} This result encourages us to make an
attempt to obtain a clear conception of the stationary states which
have so far only been regarded as formal. With this end in view,
we naturally turn to the spectrum of hydrogen. The formula
applying to this spectrum is given by the expression
\[
\frac{1}{\lambda} = \frac{R}{n_{1}^{2}} - \frac{R}{n_{2}^{2}}.
\]
According to our assumption this spectrum is produced by transitions
between a series of stationary states of a system, concerning
which we can for the present only say that the energy of the system
in the $n$th~state, apart from an additive constant, is given by
$-\dfrac{Rhc}{n^{2}}$. Let us now try to find a connection between this and the
model of the hydrogen atom. We assume that in the calculation
of the frequency of revolution of the electron in the stationary states
of the atom it will be possible to employ the above formula for~$\omega$.
It is quite natural to make this assumption; since, in trying to
form a reasonable conception of the stationary states, there is, for
the present at least, no other means available besides the ordinary
mechanics.

Corresponding to the $n$th~stationary state in formula~\Eq{(4)} for~$\omega$,
let us by way of experiment put $W = \dfrac{Rhc}{n^{2}}$. This gives us
\[
\omega_{n}^{2} = \frac{2}{\pi^{2}}\, \frac{R^{3} h^{3} c^{3}}{e^{4} mn^{6}}.
\Tag{(6)}
\]

The radiation of light corresponding to a particular spectral line
is according to our assumption emitted by a transition between
two stationary states, corresponding to two different frequencies of
revolution, and we are not justified in expecting any simple relation
between these frequencies of revolution of the electron and
the frequency of the emitted radiation. You understand, of course,
that I am by no means trying to give what might ordinarily be
described as an explanation; nothing has been said here about
\PageSep{13}
how or why the radiation is emitted. On one point, however, we
may expect a connection with the ordinary conceptions; namely,
that it will be possible to calculate the emission of slow electromagnetic
oscillations on the basis of the classical electrodynamics.
This assumption is very strongly supported by the result of
Lorentz's calculations which have already been described. From
the formula for~$\omega$ it is seen that the frequency of revolution decreases
as $n$~increases, and that the expression~$\dfrac{\omega_{n}}{\omega_{n+1}}$ approaches the
value~$1$.

According to what has been said above, the frequency of the
radiation corresponding to the transition between the $(n + 1)$th
and the $n$th~stationary state is given by
\[
\nu = Rc \left(\frac{1}{n^{2}} - \frac{1}{(n + 1)^{2}}\right).
\]
If $n$~is very large this expression is approximately equal to
\[
\nu = 2Rc/n^{3}.
\]
In order to obtain a connection with the ordinary electrodynamics
let us now place this frequency equal to the frequency of revolution,
that is
\[
\omega_{n} = 2Rc/n^{3}.
\]
Introducing this value of~$\omega_{n}$ in~\Eq{(6)} we see that $n$~disappears from
the equation, and further that the equation will be satisfied only if
\[
R = \frac{2\pi^{2} e^{4} m}{ch^{3}}.
\Tag{(7)}
\]
The constant~$R$ is very accurately known, and is, as I have said
before, equal to~$109675$. By introducing the most recent values
for $e$,~$m$ and~$h$ the expression on the right-hand side of the equation
becomes equal to $1.09 \cdot 10^{5}$. The agreement is as good as
could be expected, considering the uncertainty in the experimental
determination of the constants $e$,~$m$ and~$h$. The agreement between
our calculations and the classical electrodynamics is, therefore,
fully as good as we are justified in expecting.

We \Chg{can not}{cannot} expect to obtain a corresponding explanation of the
frequency values of the other stationary states. Certain simple
formal relations apply, however, to all the stationary states. By
introducing the expression, which has been found for~$R$, we
get for the $n$th~state $W_{n} = \frac{1}{2}nh\omega_{n}$. This equation is entirely
\PageSep{14}
analogous to Planck's assumption concerning the energy of a
resonator. $W$~in our system is readily shown to be equal to the
average value of the kinetic energy of the electron during a
single revolution. The energy of a resonator was shown by Planck
you may remember to be always equal to~$nh\nu$. Further the average
value of the kinetic energy of Planck's resonator is equal to its
potential energy, so that the average value of the kinetic energy
of the resonator, according to Planck, is equal to~$\frac{1}{2}nh\omega$. This
analogy suggests another manner of presenting the theory, and it
was just in this way that I was originally led into these considerations.
When we consider how differently the equation is
employed here and in Planck's theory it appears to me misleading
to use this analogy as a foundation, and in the account I have
given I have tried to free myself as much as possible from it.

Let us continue with the elucidation of the calculations, and in
the expression for~$2a$ introduce the value of~$W$ which corresponds
to the $n$th~stationary state. This gives us
\[
2a = n^{2} \cdot \frac{e^{2}}{chR}
   = n^{2} \cdot \frac{h^{2}}{2\pi^{2} me^{2}}
   = n^{2} \cdot 1.1 \cdot 10^{-8}.
\Tag{(8)}
\]

It is seen that for small values of~$n$, we obtain values for the
major axis of the orbit of the electron which are of the same
order of magnitude as the values of the diameters of the atoms
calculated from the kinetic theory of gases. For large values of~$n$,
$2a$~becomes very large in proportion to the calculated dimensions
of the atoms. This, however, does not necessarily disagree with
experiment. Under ordinary circumstances a hydrogen atom will
probably exist only in the state corresponding to $n = 1$. For this
state $W$~will have its greatest value and, consequently, the atom
will have emitted the largest amount of energy possible; this will
therefore represent the most stable state of the atom from which
the system \Chg{can not}{cannot} be transferred except by adding energy to it
from without. The large values for~$2a$ corresponding to large~$n$ need
not, therefore, be contrary to experiment; indeed, we may in these
large values seek an explanation of the fact, that in the laboratory
it has hitherto not been possible to observe the hydrogen lines
corresponding to large values of~$n$ in Balmer's formula, while they
have been observed in the spectra of certain stars. In order that
the large orbits of the electrons may not be disturbed by electrical
\PageSep{15}
forces from the neighbouring atoms the pressure will have to be
very low, so low, indeed, that it is impossible to obtain sufficient
light from a Geissler tube of ordinary dimensions. In the stars,
however, we may assume that we have to do with hydrogen which
is exceedingly attenuated and distributed throughout an enormously
large region of space.

\Section{The Pickering lines.} You have probably noticed that we have
not mentioned at all the spectrum found in certain stars which
according to the opinion then current was assigned to hydrogen,
and together with the ordinary hydrogen spectrum was considered
by Rydberg to form a connected system of lines completely
analogous to the spectra of other elements. You have probably
also perceived that difficulties would arise in interpreting this
spectrum by means of the assumptions which have been employed.
If such an attempt were to be made it would be necessary to give
up the simple considerations which lead to the expression~\Eq{(7)} for
the constant~$R$. We shall see, however, that it appears possible to
explain the occurrence of this spectrum in another way. Let us
suppose that it is not due to hydrogen, but to some other simple
system consisting of a single electron revolving about a nucleus
with an electrical charge~$Ne$. The expression for~$\omega$ becomes then
\[
\omega^{2} = \frac{2}{\pi^{2}}\, \frac{W^{3}}{N^{2} e^{4} m}.
\]
Repeating the same calculations as before only in the inverse
order we find, that this system will emit a line spectrum given by
the expression
\[
\frac{1}{\lambda}
  = \frac{2\pi^{2} N^{2} e^{4} m}{ch^{3}}\left(\frac{1}{n_{1}^{2}} - \frac{1}{n_{2}^{2}}\right)
  = R\raisebox{-4pt}{$\Biggl($}\frac{1}{\left(\dfrac{n_{1}}{N}\right)^{2}} - \frac{1}{\left(\dfrac{n_{2}}{N}\right)^{2}}\raisebox{-4pt}{$\Biggr)$}.
\Tag{(9)}
\]

By comparing this formula with the formula for Pickering's and
Rydberg's series, we see that the observed lines can be explained
on the basis of the theory, if it be assumed that the spectrum is
due to an electron revolving about a nucleus with a charge~$2e$, or
according to Rutherford's theory around the nucleus of a helium
atom. The fact that the spectrum in question is not observed in
an ordinary helium tube, but only in stars, may be accounted for
\PageSep{16}
by the high degree of ionization which is required for the production
of this spectrum; a neutral helium atom contains of course
two electrons while the system under consideration contains
only one.

These conclusions appear to be supported by experiment.
Fowler, as I have mentioned, has recently succeeded in observing
Pickering's and Rydberg's lines in a laboratory experiment. By
passing a very heavy current through a mixture of hydrogen and
helium Fowler observed not only these lines but also a new series
of lines. This new series was of the same general type, the wave
length being given approximately by
\[
\frac{1}{\lambda}
  = R\left(\frac{1}{(\frac{3}{2})^{2}} - \frac{1}{(n + \frac{1}{2})^{2}}\right).
\]
Fowler interpreted all the observed lines as the hydrogen spectrum
sought for. With the observation of the latter series of lines,
however, the basis of the analogy between the hypothetical
hydrogen spectrum and the other spectra disappeared, and thereby
also the foundation upon which Rydberg had founded his conclusions;
on the contrary it is seen, that the occurrence of the lines
was exactly what was to be expected on our view.

In the following table the first column contains the wave
lengths measured by Fowler, while the second contains the limiting
values of the experimental errors given by him; in the third
column we find the products of the wave lengths by the quantity
$\left(\dfrac{1}{n_{1}^{2}} - \dfrac{1}{n_{2}^{2}}\right) \Add{\cdot} 10^{10}$; the values employed for $n_{1}$~and~$n_{2}$ are enclosed in
parentheses in the last column.
\begin{table}[hbt]
\Pagelabel{16}
\[
\begin{array}{l*{2}{>{\qquad}l}l}
\ColHead{\lambda \cdot 10^{8}} &
\ColHead{\text{Limit of error}} &
\ColHead{\lambda \cdot \left(\dfrac{1}{n_{1}^{2}} - \dfrac{1}{n_{2}^{2}}\right) \cdot 10^{10}} & \\
4685.98 & 0.01 & 22779.1 & (3 : 4) \\
3203.30 & 0.05 & 22779.0 & (3 : 5) \\
2733.34 & 0.05 & 22777.8 & (3 : 6) \\
2511.31 & 0.05 & 22778.3 & (3 : 7) \\
2385.47 & 0.05 & 22777.9 & (3 : 8) \\
2306.20 & 0.10 & 22777.3 & (3 : 9) \\
2252.88 & 0.10 & 22779.1 & (3 : 10) \\
5410.5  & 1.0  & 22774   & (4 : 7) \\
4541.3  & 0.25 & 22777   & (4 : 9) \\
4200.3  & 0.5  & 22781   & (4 : 11) \\
\end{array}
\]
\end{table}
\PageSep{17}

The values of the products are seen to be very nearly equal,
while the deviations are of the same order of magnitude as the
limits of experimental error. The value of the product
\[
\lambda \left(\frac{1}{n_{1}^{2}} - \frac{1}{n_{2}^{2}}\right)
\]
should for this spectrum, according to the formula~\Eq{(9)}, be exactly
$\frac{1}{4}$~of the corresponding product for the hydrogen spectrum. From
the tables on pages \PageNum{1} and~\PageNum{16} we find for these products $91153$
and $22779$, and dividing the former by the latter we get $4.0016$.
This value is very nearly equal to~$4$; the deviation is, however,
much greater than can be accounted for in any way by the errors
of the experiments. It has been easy, however, to find a theoretical
explanation of this point. In all the foregoing calculations
we have assumed that the mass of the nucleus is infinitely great
compared to that of the electron. This is of course not the
case, even though it holds to a very close approximation; for a
hydrogen atom the ratio of the mass of the nucleus to that of the
electron will be about $1850$ and for a helium atom four times as
great.

If we consider a system consisting of an electron revolving about
a nucleus with a charge~$Ne$ and a mass~$M$, we find the following
expression for the frequency of revolution of the system:
\[
\omega^{2} = \frac{2}{\pi^{2}}\, \frac{W^{3} (M + m)}{N^{2} e^{4} Mm}.
\]

From this formula we find in a manner quite similar to that
previously employed that the system will emit a line spectrum,
the wave lengths of which are given by the formula
\[
\frac{1}{\lambda}
  = \frac{2\pi^{2} N^{2} e^{4} mM}{ch^{3} (M + m)}
  \left(\frac{1}{n_{1}^{2}} - \frac{1}{n_{2}^{2}}\right).
\Tag{(10)}
\]

If with the aid of this formula we try to find the ratio of the
product for the hydrogen spectrum, to that of the hypothetical
helium spectrum we get the value $4.00163$ which is in complete
agreement with the preceding value calculated from the experimental
observations.

I must further mention that Evans has made some experiments
to determine whether the spectrum in question is due to hydrogen
or helium. He succeeded in observing one of the lines in very
\PageSep{18}
pure helium; there was, at any rate, not enough hydrogen present
to enable the hydrogen lines to be observed. Since in any event
Fowler does not seem to consider such evidence as conclusive it is
to be hoped that these experiments will be continued. There is,
however, also another possibility of deciding this question. As is
evident from the formula~\Eq{(10)}, the helium spectrum under consideration
should contain, besides the lines observed by Fowler, a
series of lines lying close to the ordinary hydrogen lines. These
lines may be obtained by putting $n_{1} = 4$, $n_{2} = 6$, $8$, $10$,~etc. Even
if these lines were present, it would be extremely difficult to
observe them on account of their position with regard to the
hydrogen lines, but should they be observed this would probably
also settle the question of the origin of the spectrum, since no
reason would seem to be left to assume the spectrum to be due to
hydrogen.

\Section{Other spectra.} For the spectra of other elements the problem
becomes more complicated, since the atoms contain a larger
number of electrons. It has not yet been possible on the basis of
this theory to explain any other spectra besides those which I
have already mentioned. On the other hand it ought to be
mentioned that the general laws applying to the spectra are very
simply interpreted on the basis of our assumptions. So far as the
combination principle is concerned its explanation is obvious. In
the method we have employed our point of departure was largely
determined by this particular principle. But a simple explanation
can be also given of the other general law, namely, the occurrence
of Rydberg's constant in all spectral formulae. Let us assume
that the spectra under consideration, like the spectrum of hydrogen,
are emitted by a neutral system, and that they are produced by
the binding of an electron previously removed from the system.
If such an electron revolves about the nucleus in an orbit which
is large in proportion to that of the other electrons it will be
subjected to forces much the same as the electron in a hydrogen
atom, since the inner electrons individually will approximately
neutralize the effect of a part of the positive charge of the nucleus.
We may therefore assume that for this system there will exist a
series of stationary states in which the motion of the outermost
\PageSep{19}
electron is approximately the same as in the stationary states of a
hydrogen atom. I shall not discuss these matters any further,
but shall only mention that they lead to the conclusion that
Rydberg's constant is not exactly the same for all elements.
The expression for this constant will in fact contain the factor
$\dfrac{M}{M + m}$, where $M$~is the mass of the nucleus. The correction is
exceedingly small for elements of large atomic weight, but for
hydrogen it is, from the point of view of spectrum analysis, very
considerable. If the procedure employed leads to correct results, it
is not therefore permissible to calculate Rydberg's constant directly
from the hydrogen spectrum; the value of the universal constant
should according to the theory be~$109735$ and not~$109675$.

I shall not tire you any further with more details; I hope to
return to these questions here in the Physical Society, and to
show how, on the basis of the underlying ideas, it is possible
to develop a theory for the structure of atoms and molecules.
Before closing I only wish to say that I hope I have expressed
myself sufficiently clearly so that you have appreciated the extent
to which these considerations conflict with the admirably coherent
group of conceptions which have been rightly termed the classical
theory of electrodynamics. On the other hand, by emphasizing
this conflict, I have tried to convey to you the impression that it
may be also possible in the course of time to discover a certain
coherence in the new ideas.
\PageSep{20}


\Essay{II}{On the Series Spectra of the Elements}
  {Address delivered before the Physical Society in Berlin, April~27, 1920.}

\Chapter{I.}{Introduction}

The subject on which I have the honour to speak here, at the
kind invitation of the Council of your society, is very extensive and
it would be impossible in a single address to give a comprehensive
survey of even the most important results obtained in the theory
of spectra. In what follows I shall try merely to emphasize some
points of view which seem to me important when considering the
present state of the theory of spectra and the possibilities of its
development in the near future. I regret in this connection not to
have time to describe the history of the development of spectral
theories, although this would be of interest for our purpose. No
difficulty, however, in understanding this lecture need be experienced
on this account, since the points of view underlying previous
attempts to explain the spectra differ fundamentally from those upon
which the following considerations rest. This difference exists both
in the development of our ideas about the structure of the atom
and in the manner in which these ideas are used in explaining the
spectra.

We shall assume, according to Rutherford's theory, that an atom
consists of a positively charged nucleus with a number of electrons
revolving about it. Although the nucleus is assumed to be very
small in proportion to the size of the whole atom, it will contain
nearly the entire mass of the atom. I shall not state the reasons
which led to the establishment of this nuclear theory of the atom,
nor describe the very strong support which this theory has received
from very different sources. I shall mention only that result
which lends such charm and simplicity to the modern development
of the atomic theory. I refer to the idea that the number of electrons
in a neutral atom is exactly equal to the number, giving the
position of the element in the periodic table, the so-called ``atomic
number.'' This assumption, which was first proposed by van~den
Broek, immediately suggests the possibility ultimately of deriving
\PageSep{21}
the explanation of the physical and chemical properties of the
elements from their atomic numbers. If, however, an explanation
of this kind is attempted on the basis of the classical laws of
mechanics and electrodynamics, insurmountable difficulties are encountered.
These difficulties become especially apparent when we
consider the spectra of the elements. In fact, the difficulties are
here so obvious that it would be a waste of time to discuss them in
detail. It is evident that systems like the nuclear atom, if based
upon the usual mechanical and electrodynamical conceptions,
would not even possess sufficient stability to give a spectrum consisting
of sharp lines.

In this lecture I shall use the ideas of the quantum theory. It
will not be necessary, particularly here in Berlin, to consider in
detail how Planck's fundamental work on temperature radiation
has given rise to this theory, according to which the laws governing
atomic processes exhibit a definite element of discontinuity. I shall
mention only Planck's chief result about the properties of an exceedingly
simple kind of atomic system, the Planck ``oscillator.''
This consists of an electrically charged particle which can execute
harmonic oscillations about its position of equilibrium with a frequency
independent of the amplitude. By studying the statistical
equilibrium of a number of such systems in a field of radiation
Planck was led to the conclusion that the emission and absorption
of radiation take place in such a manner, that, so far as a statistical
equilibrium is concerned only certain distinctive states of the
oscillator are to be taken into consideration. In these states the
energy of the system is equal to a whole multiple of a so-called
``energy quantum,'' which was found to be proportional to the frequency
of the oscillator. The particular energy values are therefore
given by the well-known formula
\[
E_{n} = nh\omega,
\Tag{(1)}
\]
where $n$~is a whole number, $\omega$~the frequency of vibration of the
oscillator, and $h$~is Planck's constant.

If we attempt to use this result to explain the spectra of the
elements, however, we encounter difficulties, because the motion of
the particles in the atom, in spite of its simple structure, is in general
exceedingly complicated compared with the motion of a Planck
\PageSep{22}
oscillator. The question then arises, how Planck's result ought to
be generalized in order to make its application possible. Different
points of view immediately suggest themselves. Thus we might
regard this equation as a relation expressing certain characteristic
properties of the distinctive motions of an atomic system and try
to obtain the general form of these properties. On the other hand,
we may also regard equation~\Eq{(1)} as a statement about a property
of the process of radiation and inquire into the general laws which
control this process.

In Planck's theory it is taken for granted that the frequency of
the radiation emitted and absorbed by the oscillator is equal to its
own frequency, an assumption which may be written
\[
\nu \equiv \omega,
\Tag{(2)}
\]
if in order to make a sharp distinction between the frequency of
the emitted radiation and the frequency of the particles in the atoms,
we here and in the following denote the former by~$\nu$ and the latter
by~$\omega$. We see, therefore, that Planck's result may be interpreted to
mean, that the oscillator can emit and absorb radiation only in
``radiation quanta'' of magnitude
\[
\Delta E = h\nu.
\Tag{(3)}
\]
It is well known that ideas of this kind led Einstein to a theory
of the photoelectric effect. This is of great importance, since it
represents the first instance in which the quantum theory was
applied to a phenomenon of non-statistical character. I shall not
here discuss the familiar difficulties to which the ``hypothesis of
light quanta'' leads in connection with the phenomena of interference,
for the explanation of which the classical theory of radiation
has shown itself to be so remarkably suited. Above all I shall not
consider the problem of the nature of radiation, I shall only attempt
to show how it has been possible in a purely formal manner to
develop a spectral theory, the essential elements of which may be
considered as a simultaneous rational development of the two ways
of interpreting Planck's result.
\PageSep{23}


\Chapter{II.}{General Principles of the Quantum Theory
of\protect~Spectra}

In order to explain the appearance of line spectra we are compelled
to assume that the emission of radiation by an atomic system
takes place in such a manner that it is not possible to follow the
emission in detail by means of the usual conceptions. Indeed, these
do not even offer us the means of calculating the frequency of the
emitted radiation. We shall see, however, that it is possible to give
a very simple explanation of the general empirical laws for the
frequencies of the spectral lines, if for each emission of radiation
by the atom we assume the fundamental law to hold, that during
the entire period of the emission the radiation possesses one and
the same frequency~$\nu$, connected with the total energy emitted by
the \emph{frequency relation}
\[
h\nu = E' - E''.
\Tag{(4)}
\]
Here $E'$~and $E''$ represent the energy of the system before and
after the emission.

If this law is assumed, the spectra do not give us information
about the motion of the particles in the atom, as is supposed in the
usual theory of radiation, but only a knowledge of the energy
changes in the various processes which can occur in the atom.
From this point of view the spectra show the existence of certain,
definite energy values corresponding to certain distinctive states
of the atoms. These states will be called the \emph{stationary states} of
the atoms, since we shall assume that the atom can remain a finite
time in each state, and can leave this state only by a process of
transition to another stationary state. Notwithstanding the fundamental
departure from the ordinary mechanical and electrodynamical
conceptions, we shall see, however, that it is possible to give a
rational interpretation of the evidence provided by the spectra on
the basis of these ideas.

Although we must assume that the ordinary mechanics \Chg{can not}{cannot}
be used to describe the transitions between the stationary states,
nevertheless, it has been found possible to develop a consistent
theory on the assumption that the motion in these states can be
described by the use of the ordinary mechanics. Moreover, although
the process of radiation \Chg{can not}{cannot} be described on the basis of the
\PageSep{24}
ordinary theory of electrodynamics, according to which the nature
of the radiation emitted by an atom is directly related to the harmonic
components occurring in the motion of the system, there is
found, nevertheless, to exist a far-reaching \emph{correspondence} between
the various types of possible transitions between the stationary
states on the one hand and the various harmonic components of the
motion on the other hand. This correspondence is of such a nature,
that the present theory of spectra is in a certain sense to be regarded
as a rational generalization of the ordinary theory of radiation.

\Section{Hydrogen spectrum.} In order that the principal points may
stand out as clearly as possible I shall, before considering the more
complicated types of series spectra, first consider the simplest spectrum,
namely, the series spectrum of hydrogen. This spectrum
consists of a number of lines whose frequencies are given with great
exactness by Balmer's formula
\[
\nu = \frac{K}{(n'')^{2}} - \frac{K}{(n')^{2}},
\Tag{(5)}
\]
where $K$~is a constant, and $n'$~and $n''$ are whole numbers. If we put
$n'' = 2$ and give to~$n'$ the values $3$,~$4$,~etc., we get the well-known
Balmer series of hydrogen. If we put $n'' = 1$ or $n'' = 3$ we obtain
respectively the ultra-violet and infra-red series. We shall assume
the hydrogen atom simply to consist of a positively charged nucleus
with a single electron revolving about it. For the sake of simplicity
we shall suppose the mass of the nucleus to be infinite in comparison
with the mass of the electron, and further we shall disregard the
small variations in the motion due to the change in mass of the
electron with its velocity. With these simplifications the electron
will describe a closed elliptical orbit with the nucleus at one of the
foci. The frequency of revolution~$\omega$ and the major axis~$2a$ of the
orbit will be connected with the energy of the system by the following
equations:
\[
\omega = \sqrt{\frac{2W^{3}}{\pi^{2} e^{4} m}},\quad
2a = \frac{e^{2}}{W}.
\Tag{(6)}
\]
Here $e$~is the charge of the electron and $m$~its mass, while $W$~is the
work required to remove the electron to infinity.

The simplicity of these formulae suggests the possibility of using
them in an attempt to explain the spectrum of hydrogen. This,
\PageSep{25}
however, is not possible so long as we use the classical theory of
radiation. It would not even be possible to understand how hydrogen
could emit a spectrum consisting of sharp lines; for since $\omega$~varies
with~$W$, the frequency of the emitted radiation would vary continuously
during the emission. We can avoid these difficulties if
we use the ideas of the quantum theory. If for each line we form
the product~$h\nu$ by multiplying both sides of~\Eq{(5)} by~$h$, then, since
the right-hand side of the resulting relation may be written as
the difference of two simple expressions, we are led by comparison
with formula~\Eq{(4)} to the assumption that the separate lines of the
spectrum will be emitted by transitions between two stationary
states, forming members of an infinite series of states, in which the
energy in the $n$th~state apart from an arbitrary additive constant is
determined by the expression
\[
E_{n} = -\frac{Kh}{n^{2}}.
\Tag{(7)}
\]
The negative sign has been chosen because the energy of the atom
will be most simply characterized by the work~$W$ required to remove
the electron completely from the atom. If we now substitute $\dfrac{Kh}{n^{2}}$
for~$W$ in formula~\Eq{(6)}, we obtain the following expression for the frequency
and the major axis in the $n$th~stationary state:
\[
\omega_{n} = \frac{1}{n^{3}} \sqrt{\frac{2h^{3} K^{3}}{\pi^{2} e^{4} m}},\quad
2a_{n} = \frac{n^{2} e^{2}}{hK}.
\Tag{(8)}
\]
A comparison between the motions determined by these equations
and the distinctive states of a Planck resonator may be shown to
offer a theoretical determination of the constant~$K$. Instead of
doing this I shall show how the value of~$K$ can be found by a simple
comparison of the spectrum emitted with the motion in the stationary
states, a comparison which at the same time will lead us to the
principle of correspondence.

We have assumed that each hydrogen line is the result of a
transition between two stationary states of the atom corresponding
to different values of~$n$. Equations~\Eq{(8)} show that the frequency of
revolution and the major axis of the orbit can be entirely different
in the two states, since, as the energy decreases, the major axis of
the orbit becomes smaller and the frequency of revolution increases.
\PageSep{26}
In general, therefore, it will be impossible to obtain a relation between
the frequency of revolution of the electrons and the frequency
of the radiation as in the ordinary theory of radiation. If, however,
we consider the ratio of the frequencies of revolution in two stationary
states corresponding to given values of $n'$~and~$n''$, we see that this
ratio approaches unity as $n'$~and $n''$ gradually increase, if at the
same time the difference $n' - n''$ remains unchanged. By considering
transitions corresponding to large values of $n'$~and~$n''$ we may
therefore hope to establish a certain connection with the ordinary
theory. For the frequency of the radiation emitted by a transition,
we get according to~\Eq{(5)}
\[
\nu = \frac{K}{(n'')^{2}} - \frac{K}{(n')^{2}}
  = (n' - n'') K\, \frac{n' + n''}{(n')^{2} (n'')^{2}}.
\Tag{(9)}
\]
If now the numbers $n'$~and $n''$ are large in proportion to their difference,
we see that by equations~\Eq{(8)} this expression may be written
approximately,
\[
\nu \sim (n' - n'') \omega \sqrt{\frac{2\pi^{2} e^{4} m}{Kh^{3}}},
\Tag{(10)}
\]
where $\omega$~represents the frequency of revolution in the one or the
other of the two stationary states. Since $n' - n''$ is a whole number,
we see that the first part of this expression, \ie\ $(n' - n'')\omega$, is the
same as the frequency of one of the harmonic components into
which the elliptical motion may be decomposed. This involves the
well-known result that for a system of particles having a periodic
motion of frequency~$\omega$, the displacement~$\xi$ of the particles in a given
direction in space may be represented as a function of the time by
a trigonometric series of the form
\[
\xi = \sum C_{\tau} \cos 2\pi(\tau\omega t + c_{\tau}),
\Tag{(11)}
\]
where the summation is to be extended over all positive integral
values of~$\tau$.

We see, therefore, that the frequency of the radiation emitted
by a transition between two stationary states, for which the numbers
$n'$~and $n''$ are large in proportion to their difference, will coincide
with the frequency of one of the components of the radiation, which
according to the ordinary ideas of radiation would be expected from
the motion of the atom in these states, provided the last factor on
\PageSep{27}
the right-hand side of equation~\Eq{(10)} is equal to~$1$. This condition,
which is identical to the condition
\[
K = \frac{2\pi^{2} e^{4} m}{h^{3}},
\Tag{(12)}
\]
is in fact fulfilled, if we give to~$K$ its value as found from measurements
on the hydrogen spectrum, and if for $e$,~$m$ and~$h$ we use the
values obtained directly from experiment. This agreement clearly
gives us a \emph{connection between the spectrum and the atomic model of
hydrogen}, which is as close as could reasonably be expected considering
the fundamental difference between the ideas of the quantum
theory and of the ordinary theory of radiation.

\Section{The correspondence principle.} Let us now consider somewhat
more closely this relation between the spectra one would expect on
the basis of the quantum theory, and on the ordinary theory of
radiation. The frequencies of the spectral lines calculated according
to both methods agree completely in the region where the stationary
states deviate only little from one another. We must not forget,
however, that the mechanism of emission in both cases is different.
The different frequencies corresponding to the various harmonic
components of the motion are emitted simultaneously according to
the ordinary theory of radiation and with a relative intensity depending
directly upon the ratio of the amplitudes of these oscillations.
But according to the quantum theory the various spectral
lines are emitted by entirely distinct processes, consisting of transitions
from one stationary state to various adjacent states, so that
the radiation corresponding to the $\tau$th~``harmonic'' will be emitted
by a transition for which $n' - n'' = \tau$. The relative intensity
with which each particular line is emitted depends consequently
upon the relative probability of the occurrence of the different
transitions.

This correspondence between the frequencies determined by the
two methods must have a deeper significance and we are led to
anticipate that it will also apply to the intensities. This is equivalent
to the statement that, when the quantum numbers are large,
the relative probability of a particular transition is connected in a
simple manner with the amplitude of the corresponding harmonic
component in the motion.
\PageSep{28}

This peculiar relation suggests a \emph{general law for the occurrence
of transitions between stationary states}. Thus we shall assume that
even when the quantum numbers are small the possibility of
transition between two stationary states is connected with the
presence of a certain harmonic component in the motion of the
system. If the numbers $n'$~and $n''$ are not large in proportion to
their difference, the numerical value of the amplitudes of these
components in the two stationary states may be entirely different.
We must be prepared to find, therefore, that the exact connection
between the probability of a transition and the amplitude of the
corresponding harmonic component in the motion is in general
complicated like the connection between the frequency of the radiation
and that of the component. From this point of view, for
example, the green line~$H_{\beta}$ of the hydrogen spectrum which corresponds
to a transition from the fourth to the second stationary
state may be considered in a certain sense to be an ``octave'' of the
red line~$H_{\alpha}$, corresponding to a transition from the third to the
second state, even though the frequency of the first line is by no
means twice as great as that of the latter. In fact, the transition
giving rise to~$H_{\beta}$ may be regarded as due to the presence of a harmonic
oscillation in the motion of the atom, which is an octave
higher than the oscillation giving rise to the emission of~$H_{\alpha}$.

Before considering other spectra, where numerous opportunities
will be found to use this point of view, I shall briefly mention an
interesting application to the Planck oscillator. If from \Eq{(1)}~and \Eq{(4)}
we calculate the frequency, which would correspond to a transition
between two particular states of such an oscillator, we find
\[
\nu = (n' - n'')\omega,
\Tag{(13)}
\]
where $n'$~and $n''$ are the numbers characterizing the states. It was
an essential assumption in Planck's theory that the frequency of
the radiation emitted and absorbed by the oscillator is always equal
to~$\omega$. We see that this assumption is equivalent to the assertion
that transitions occur only between two successive stationary states
in sharp contrast to the hydrogen atom. According to our view,
however, this was exactly what might have been expected, for we
must assume that the essential difference between the oscillator
and the hydrogen atom is that the motion of the oscillator is simple

\PageSep{29}
harmonic. We can see that it is possible to develop a formal theory
of radiation, in which the spectrum of hydrogen and the simple
spectrum of a Planck oscillator appear completely analogous. This
theory can only be formulated by one and the same condition for
a system as simple as the oscillator. In general this condition
breaks up into two parts, one concerning the fixation of the stationary
states, and the other relating to the frequency of the radiation
emitted by a transition between these states.

\Section{General spectral laws.} Although the series spectra of the
elements of higher atomic number have a more complicated structure
than the hydrogen spectrum, simple laws have been discovered
showing a remarkable analogy to the Balmer formula. Rydberg
and Ritz showed that the frequencies in the series spectra of many
elements can be expressed by a formula of the type
\[
\nu = f_{k''}(n'') - f_{k'}(n'),
\Tag{(14)}
\]
where $n'$~and $n''$ are two whole numbers and $f_{k'}$~and $f_{k''}$ are two
functions belonging to a series of functions characteristic of the
element. These functions vary in a simple manner with~$n$ and in
particular converge to zero for increasing values of~$n$. The various
series of lines are obtained from this formula by allowing the first
term~$f_{k''}(n'')$ to remain constant, while a series of consecutive whole
numbers are substituted for~$n'$ in the second term~$f_{k'}(n')$. According
to the Ritz \emph{combination principle} the entire spectrum may then
be obtained by forming every possible combination of two values
among all the quantities~$f_{k}(n)$.

The fact that the frequency of each line of the spectrum may be
written as the difference of two simple expressions depending upon
whole numbers suggests at once that the terms on the right-hand
side multiplied by~$h$ may be placed equal to the energy in the
various stationary states of the atom. The existence in the spectra
of the other elements of a number of separate functions of~$n$ compels
us to assume the presence not of one but of a number of series of
stationary states, the energy of the $n$th~state of the $k$th~series apart
from an arbitrary additive constant being given by
\[
E_{k}(n) = -h f_{k}(n).
\Tag{(15)}
\]
This complicated character of the ensemble of stationary states of
atoms of higher atomic number is exactly what was to be expected
\PageSep{30}
from the relation between the spectra calculated on the quantum
theory, and the decomposition of the motions of the atoms into
harmonic oscillations. From this point of view we may regard the
simple character of the stationary states of the hydrogen atom as
intimately connected with the simple periodic character of this
atom. Where the neutral atom contains more than one electron, we
find much more complicated motions with correspondingly complicated
harmonic components. We must therefore expect a more
complicated ensemble of stationary states, if we are still to have a
corresponding relation between the motions in the atom and the
spectrum. In the course of the lecture we shall trace this correspondence
in detail, and we shall be led to a simple explanation of
the apparent capriciousness in the occurrence of lines predicted by
the combination principle.

The following figure gives a survey of the stationary states of
the sodium atom deduced from the series terms.
\Figure[Diagram of the series spectrum of sodium.]{}{30}[Diagram of the series spectrum of sodium]

The stationary states are represented by black dots whose distance
from the vertical line a---a is proportional to the numerical value
of the energy in the states. The arrows in the figure indicate the
transitions giving those lines of the sodium spectrum which appear
under the usual conditions of excitation. The arrangement of the
states in horizontal rows corresponds to the ordinary arrangement
of the ``spectral terms'' in the spectroscopic tables. Thus, the states
in the first row~($S$) correspond to the variable term in the ``sharp
series,'' the lines of which are emitted by transitions from these
states to the first state in the second row. The states in the second
\PageSep{31}
row~($P$) correspond to the variable term in the ``principal series''
which is emitted by transitions from these states to the first state
in the $S$~row. The $D$~states correspond to the variable term in the
``diffuse series,'' which like the sharp series is emitted by transitions
to the first state in the $P$~row, and finally the $B$~states correspond
to the variable term in the ``Bergmann'' series (fundamental series),
in which transitions take place to the first state in the $D$~row. The
manner in which the various rows are arranged with reference to
one another will be used to illustrate the more detailed theory
which will be discussed later. The apparent capriciousness of the
combination principle, which I mentioned, consists in the fact that
under the usual conditions of excitation not all the lines belonging
to possible combinations of the terms of the sodium spectrum appear,
but only those indicated in the figure by arrows.

The general question of the fixation of the stationary states of
an atom containing several electrons presents difficulties of a profound
character which are perhaps still far from completely solved.
It is possible, however, to obtain an immediate insight into the
stationary states involved in the emission of the series spectra by
considering the empirical laws which have been discovered about
the spectral terms. According to the well-known law discovered by
Rydberg for the spectra of elements emitted under the usual conditions
of excitation the functions~$f_{k}(n)$ appearing in formula~\Eq{(14)}
can be written in the form
\[
f_{k}(n) = \frac{K}{n^{2}} \phi_{k}(n),
\Tag{(16)}
\]
where $\phi_{k}(n)$~represents a function which converges to unity for
large values of~$n$. $K$~is the same constant which appears in formula~\Eq{(5)}
for the spectrum of hydrogen. This result must evidently be
explained by supposing the atom to be electrically neutral in these
states and one electron to be moving round the nucleus in an orbit
the dimensions of which are very large in proportion to the distance
of the other electrons from the nucleus. We see, indeed, that in
this case the electric force acting on the outer electron will to a
first approximation be the same as that acting upon the electron
in the hydrogen atom, and the approximation will be the better
the larger the orbit.
\PageSep{32}

On account of the limited time I shall not discuss how this
explanation of the universal appearance of Rydberg's constant in
the arc spectra is convincingly supported by the investigation of
the ``spark spectra.'' These are emitted by the elements under the
influence of very strong electrical discharges, and come from ionized
not neutral atoms. It is important, however, that I should indicate
briefly how the fundamental ideas of the theory and the assumption
that in the states corresponding to the spectra one electron moves
in an orbit around the others, are both supported by investigations
on selective absorption and the excitation of spectral lines by
bombardment by electrons.

\Section{Absorption and excitation of radiation.}\Pagelabel{32} Just as we have
assumed that each emission of radiation is due to a transition from
a stationary state of higher to one of lower energy, so also we must
assume absorption of radiation by the atom to be due to a transition
in the opposite direction. For an element to absorb light corresponding
to a given line in its series spectrum, it is therefore
necessary for the atom of this element to be in that one of the two
states connected with the line possessing the smaller energy value.
If we now consider an element whose atoms in the gaseous state
do not combine into molecules, it will be necessary to assume that
under ordinary conditions nearly all the atoms exist in that stationary
state in which the value of the energy is a minimum. This state
I shall call the \emph{normal state}. We must therefore expect that the
absorption spectrum of a monatomic gas will contain only those
lines of the series spectrum, whose emission corresponds to transitions
to the normal state. This expectation is completely confirmed
by the spectra of the alkali metals. The absorption spectrum of
sodium vapour, for example, exhibits lines corresponding only to
the principal series, which as mentioned in the description of the
figure corresponds with transitions to the state of minimum energy.
Further confirmation of this view of the process of absorption is
given by experiments on \emph{resonance radiation}. Wood first showed
that sodium vapour subjected to light corresponding to the first
line of the principal series---the familiar yellow line---acquires the
ability of again emitting a radiation consisting only of the light of
this line. We can explain this by supposing the sodium atom to
\PageSep{33}
have been transferred from the normal state to the first state in
the second row. The fact that the resonance radiation does not
exhibit the same degree of polarization as the incident light is
in perfect agreement with our assumption that the radiation from
the excited vapour is not a resonance phenomenon in the sense of
the ordinary theory of radiation, but on the contrary depends on a
process which is not directly connected with the incident radiation.

The phenomenon of the resonance radiation of the yellow sodium
line is, however, not quite so simple as I have indicated, since, as
you know, this line is really a doublet. This means that the variable
terms of the principal series are not simple but are represented by
two values slightly different from one another. According to our
picture of the origin of the sodium spectrum this means that the
$P$~states in the second row in the figure---as opposed to the $S$~states
in the first row---are not simple, but that for each place in this row
there are two stationary states. The energy values differ so little
from one another that it is impossible to represent them in the
figure as separate dots. The emission (and absorption) of the two
components of the yellow line are, therefore, connected with two
different processes. This was beautifully shown by some later researches
of Wood and Dunoyer. They found that if sodium vapour
is subjected to radiation from only one of the two components of
the yellow line, the resonance radiation, at least at low pressures,
consists only of this component. These experiments were later
continued by Strutt, and were extended to the case where the
exciting line corresponded to the second line in the principal series.
Strutt found that the resonance radiation consisted apparently only
to a small extent of light of the same frequency as the incident
light, while the greater part consisted of the familiar yellow line.
This result must appear very astonishing on the ordinary ideas of
resonance, since, as Strutt pointed out, no rational connection exists
between the frequencies of the first and second lines of the principal
series. It is however easily explained from our point of view. From
the figure it can be seen that when an atom has been transferred
into the second state in the second row, in addition to the direct
return to the normal state, there are still two other transitions
which may give rise to radiation, namely the transitions to the
second state in the first row and to the first state in the third row.
\PageSep{34}
The experiments seem to indicate that the second of these three
transitions is most probable, and I shall show later that there is
some theoretical justification for this conclusion. By this transition,
which results in the emission of an infra-red line which could not
be observed with the experimental arrangement, the atom is taken
to the second state of the first row, and from this state only
one transition is possible, which again gives an infra-red line. This
transition takes the atom to the first state in the second row, and
the subsequent transition to the normal state then gives rise to the
yellow line. Strutt discovered another equally surprising result,
that this yellow resonance radiation seemed to consist of both
components of the first line of the principal series, even when the
incident light consisted of only one component of the second line
of the principal series. This is in beautiful agreement with our
picture of the phenomenon. We must remember that the states in
the first row are simple, so when the atom has arrived in one of
these it has lost every possibility of later giving any indication
from which of the two states in the second row it originally came.

Sodium vapour, in addition to the absorption corresponding to
the lines of the principal series, exhibits a \emph{selective absorption in a
continuous spectral region} beginning at the limit of this series and
extending into the ultra\Add{-}violet. This confirms in a striking manner
our assumption that the absorption of the lines of the principal
series of sodium results in final states of the atom in which one of
the electrons revolves in larger and larger orbits. For we must
assume that this continuous absorption corresponds to transitions
from the normal state to states in which the electron is in a position
to remove itself infinitely far from the nucleus. This phenomenon
exhibits a complete analogy with the \emph{photoelectric effect} from an
illuminated metal plate in which, by using light of a suitable
frequency, electrons of any velocity can be obtained. The frequency,
however, must always lie above a certain limit connected according
to Einstein's theory in a simple manner with the energy necessary
to bring an electron out of the metal.



This view of the origin of the emission and absorption spectra
has been confirmed in a very interesting manner by experiments
on the \emph{excitation of spectral lines and production of ionization by
electron bombardment}. The chief advance in this field is due to the
\PageSep{35}
well-known experiments of Franck and Hertz. These investigators
obtained their first important results from their experiments on
mercury vapour, whose properties particularly facilitate such experiments.
On account of the great importance of the results, these
experiments have been extended to most gases and metals that can
be obtained in a gaseous state. With the aid of the figure I shall
briefly illustrate the results for the case of sodium vapour. It was
found that the electrons upon colliding with the atoms were thrown
back with undiminished velocity when their energy was less than
that required to transfer the atom from the normal state to the
next succeeding stationary state of higher energy value. In the
case of sodium vapour this means from the first state in the first
row to the first state in the second row. As soon, however, as the
energy of the electron reaches this critical value, a new type of
collision takes place, in which the electron loses all its kinetic
energy, while at the same time the vapour is excited and emits a
radiation corresponding to the yellow line. This is what would be
expected, if by the collision the atom was transferred from the
normal state to the first one in the second row. For some time it
was uncertain to what extent this explanation was correct, since
in the experiments on mercury vapour it was found that, together
with the occurrence of non-elastic impacts, ions were always formed
in the vapour. From our figure, however, we would expect ions
to be produced only when the kinetic energy of the electrons is
sufficiently great to bring the atom out of the normal state to the
common limit of the states. Later experiments, especially by Davis
and Goucher, have settled this point. It has been shown that ions
can only be directly produced by collisions when the kinetic energy
of the electrons corresponds to the limit of the series, and that the
ionization found at first was an indirect effect arising from the
photoelectric effect produced at the metal walls of the apparatus
by the radiation arising from the return of the mercury atoms to
the normal state. These experiments provide a direct and independent
proof of the reality of the distinctive stationary states,
whose existence we were led to infer from the series spectra. At
the same time we get a striking impression of the insufficiency of
the ordinary electrodynamical and mechanical conceptions for the
description of atomic processes, not only as regards the emission
\PageSep{36}
of radiation but also in such phenomena as the collision of free
electrons with atoms.


\Chapter{III.}{Development of the Quantum Theory
of Spectra}

We see that it is possible by making use of a few simple ideas
to obtain a certain insight into the origin of the series spectra.
But when we attempt to penetrate more deeply, difficulties arise.
In fact, for systems which are not simply periodic it is not possible
to obtain sufficient information about the motions of these systems
in the stationary states from the numerical values of the energy
alone; more determining factors are required for the fixation of
the motion. We meet the same difficulties when we try to explain
in detail the characteristic effect of external forces upon the spectrum
of hydrogen. A foundation for further advances in this field has
been made in recent years through a development of the quantum
theory, which allows a fixation of the stationary states not only in
the case of simple periodic systems, but also for certain classes of
non-periodic systems. These are the \emph{conditionally periodic systems}
whose equations of motion can be solved by a ``separation of the
variables.'' If generalized coordinates are used the description of
the motion of these systems can be reduced to the consideration
of a number of generalized ``components of motion.'' Each of these
corresponds to the change of only one of the coordinates and may
therefore in a certain sense be regarded as ``independent.'' The
method for the fixation of the stationary states consists in fixing
the motion of each of these components by a condition, which can
be considered as a direct generalization of condition~\Eq{(1)} for a
Planck oscillator, so that the stationary states are in general
characterized by as many whole numbers as the number of the
degrees of freedom which the system possesses. A considerable
number of physicists have taken part in this development of the
quantum theory, including Planck himself. I also wish to mention
the important contribution made by Ehrenfest to this subject on
the limitations of the applicability of the laws of mechanics to
atomic processes. The decisive advance in the application of the
quantum theory to spectra, however, is due to Sommerfeld and his
followers. However, I shall not further discuss the systematic form
\PageSep{37}
in which these authors have presented their results. In a paper which
appeared some time ago in the Transactions of the Copenhagen
Academy, I have shown that the spectra, calculated with the aid
of this method for the fixation of the stationary states, exhibit a
correspondence with the spectra which should correspond to the
motion of the system similar to that which we have already considered
in the case of hydrogen. With the aid of this general
correspondence I shall try in the remainder of this lecture to
show how it is possible to present the theory of series spectra
and the effects produced by external fields of force upon these
spectra in a form which may be considered as the natural generalization
of the foregoing considerations. This form appears to me
to be especially suited for future work in the theory of spectra,
since it allows of an immediate insight into problems for which
the methods mentioned above fail on account of the complexity of
the motions in the atom.

\Section{Effect of external forces on the hydrogen spectrum.} We
shall now proceed to investigate the effect of small perturbing
forces upon the spectrum of the simple system consisting of a single
electron revolving about a nucleus. For the sake of simplicity we
shall for the moment disregard the variation of the mass of the
electron with its velocity. The consideration of the small changes
in the motion due to this variation has been of great importance
in the development of Sommerfeld's theory which originated in the
explanation of the \emph{fine structure of the hydrogen lines}. This fine
structure is due to the fact, that taking into account the variation
of mass with velocity the orbit of the electron deviates a little
from a simple ellipse and is no longer exactly periodic. This deviation
from a Keplerian motion is, however, very small compared
with the perturbations due to the presence of external forces, such
as occur in experiments on the Zeeman and Stark effects. In atoms
of higher atomic number it is also negligible compared with the
disturbing effect of the inner electrons on the motion of the outer
electron. The neglect of the change in mass will therefore have no
important influence upon the explanation of the Zeeman and Stark
effects, or upon the explanation of the difference between the
hydrogen spectrum and the spectra of other elements.
\PageSep{38}

We shall therefore as before consider the motion of the unperturbed
hydrogen atom as simply periodic and inquire in the
first place about the stationary states corresponding to this motion.
The energy in these states will then be determined by expression~\Eq{(7)}
which was derived from the spectrum of hydrogen. The energy of
the system being given, the major axis of the elliptical orbit of the
electron and its frequency of revolution are also determined. Substituting
in formulae \Eq{(7)} and~\Eq{(8)} the expression for~$K$ given in~\Eq{(12)},
we obtain for the energy, major axis and frequency of revolution
in the $n$th~state of the unperturbed atom the expressions
\[
\BreakMath{%
E_{n} = -W_{n} = -\frac{1}{n^{2}}\, \frac{2\pi^{2} e^{4} m}{h^{2}},\quad
2a_{n} = n^{2}\, \frac{h^{2}}{2\pi^{2} e^{2} m},\quad
\omega_{n} = \frac{1}{n^{3}}\, \frac{4\pi^{2} e^{4} m}{h^{3}}.
}{%
\begin{gathered}
E_{n} = -W_{n} = -\frac{1}{n^{2}}\, \frac{2\pi^{2} e^{4} m}{h^{2}}, \\
2a_{n} = n^{2}\, \frac{h^{2}}{2\pi^{2} e^{2} m},\qquad
\omega_{n} = \frac{1}{n^{3}}\, \frac{4\pi^{2} e^{4} m}{h^{3}}.
\end{gathered}
}
\Tag{(17)}
\]

We must further assume that in the stationary states of the
unperturbed system the form of the orbit is so far undetermined
that the \Chg{excentricity}{eccentricity} can vary continuously. This is not only immediately
indicated by the principle of correspondence,---since the
frequency of revolution is determined only by the energy and not
by the \Chg{excentricity}{eccentricity},---but also by the fact that the presence of any
small external forces will in general, in the course of time, produce
a finite change in the position as well as in the \Chg{excentricity}{eccentricity} of the
periodic orbit, while in the major axis it can produce only small
changes proportional to the intensity of the perturbing forces.

In order to fix the stationary states of systems in the presence
of a given conservative external field of force, we shall have to
investigate, on the basis of the principle of correspondence, how
these forces affect the decomposition of the motion into harmonic
oscillations. Owing to the external forces the form and position of
the orbit will vary continuously. In the general case these changes
will be so complicated that it will not be possible to decompose the
perturbed motion into discrete harmonic oscillations. In such a
case we must expect that the perturbed system will not possess
any sharply separated stationary states. Although each emission
of radiation must be assumed to be monochromatic and to proceed
according to the general frequency condition we shall therefore
expect the final effect to be a broadening of the sharp spectral lines
of the unperturbed system. In certain cases, however, the perturbations
\PageSep{39}
will be of such a regular character that the perturbed system
can be decomposed into harmonic oscillations, although the ensemble
of these oscillations will naturally be of a more complicated kind
than in the unperturbed system. This happens, for example, when
the variations of the orbit with respect to time are periodic. In
this case harmonic oscillations will appear in the motion of the
system the frequencies of which are equal to whole multiples of the
period of the orbital perturbations, and in the spectrum to be
expected on the basis of the ordinary theory of radiation we would
expect components corresponding to these frequencies. According
to the principle of correspondence we are therefore immediately
led to the conclusion, that to each stationary state in the unperturbed
system there corresponds a number of stationary states in
the perturbed system in such a manner, that for a transition
between two of these states a radiation is emitted, whose frequency
stands in the same relationship to the periodic course of the
variations in the orbit, as the spectrum of a simple periodic system
does to its motion in the stationary states.

\Section{The Stark effect.} An instructive example of the appearance of
periodic perturbations is obtained when hydrogen is subjected to
the effect of a homogeneous electric field. The \Chg{excentricity}{eccentricity} and
the position of the orbit vary continuously under the influence of
the field. During these changes, however, it is found that the
centre of the orbit remains in a plane perpendicular to the direction
of the electric force and that its motion in this plane is
simply periodic. When the centre has returned to its starting
point, the orbit will resume its original \Chg{excentricity}{eccentricity} and position,
and from this moment the entire cycle of orbits will be repeated.
In this case the determination of the energy of the stationary
states of the disturbed system is extremely simple, since it is found
that the period of the disturbance does not depend upon the
original configuration of the orbits nor therefore upon the position
of the plane in which the centre of the orbit moves, but only upon
the major axis and the frequency of revolution. From a simple
calculation it is found that the period a is given by the following
formula
\[
\sigma = \frac{3eF}{8\pi^{2} ma\omega},
\Tag{(18)}
\]
\PageSep{40}
where $F$~is the intensity of the external electric field. From
analogy with the fixation of the distinctive energy values of a
Planck oscillator we must therefore expect that the energy difference
between two different states, corresponding to the same stationary
state of the unperturbed system, will simply be equal to a whole
multiple of the product of $h$~by the period~$\sigma$ of the perturbations.
We are therefore immediately led to the following expression for
the energy of the stationary states of the perturbed system,
\[
E = E_{n} + kh\sigma,
\Tag{(19)}
\]
where $E_{n}$~depends only upon the number~$n$ characterizing the
stationary state of the unperturbed system, while $k$~is a new whole
number which in this case may be either positive or negative. As
we shall see below, consideration of the relation between the energy
and the motion of the system shows that $k$~must be numerically
less than~$n$, if, as before, we place the quantity~$E_{n}$ equal to the
energy~$-W_{n}$ of the $n$th~stationary state of the undisturbed atom.
Substituting the values of $W_{n}$,~$\omega_{n}$ and~$a_{n}$ given by~\Eq{(17)} in formula~\Eq{(19)}
we get
\[
E = -\frac{1}{n^{2}}\, \frac{2\pi^{2} e^{4} m}{h^{2}} + nk\, \frac{3h^{2} F}{8\pi^{2} em}.
\Tag{(20)}
\]
To find the effect of an electric field upon the lines of the hydrogen
spectrum, we use the frequency condition~\Eq{(4)} and obtain for the
frequency~$\nu$ of the radiation emitted by a transition between two
stationary states defined by the numbers $n'$,~$k'$ and $n''$,~$k''$
\[
\nu = \frac{2\pi^{2} e^{4} m}{h^{3}} \left(\frac{1}{(n'')^{2}} - \frac{1}{(n'')^{2}}\right)
  + \frac{3h \cdot F}{8\pi^{2} em} (n'k' - n''k'').
\Tag{(21)}
\]

It is well known that this formula provides a complete explanation
of the Stark effect of the hydrogen lines. It corresponds
exactly with the one obtained by a different method by Epstein
and Schwarzschild. They used the fact that the hydrogen atom in
a homogeneous electric field is a conditionally periodic system
permitting a separation of variables by the use of parabolic coordinates.
The stationary states were fixed by applying quantum
conditions to each of these variables.

We shall now consider more closely the correspondence between
the changes in the spectrum of hydrogen due to the presence of

\PageSep{41}
an electric field and the decomposition of the perturbed motion
of the atom into its harmonic components. Instead of the simple
decomposition into harmonic components corresponding to a simple
Kepler motion, the displacement~$\xi$ of the electron in a given
direction in space can be expressed in the present case by the
formula
\[
\xi = \sum C_{\tau,\kappa} \cos 2\pi \bigl\{t(\tau\omega + \kappa\sigma) + c_{\tau,\kappa}\bigr\},
\Tag{(22)}
\]
where $\omega$~is the average frequency of revolution in the perturbed
orbit and $\sigma$~is the period of the orbital perturbations, while $C_{\tau,\kappa}$~and
$c_{\tau,\kappa}$ are constants. The summation is to be extended over all integral
values for $\tau$~and~$\kappa$.

If we now consider a transition between two stationary states
characterized by certain numbers $n'$,~$k'$ and $n''$,~$k''$, we find that in
the region where these numbers are large compared with their
differences $n' - n''$ and $k' - k''$, the frequency of the spectral line
which is emitted will be given approximately by the formula
\[
\nu \sim (n' - n'')\omega + (k' - k'')\sigma.
\Tag{(23)}
\]
We see, therefore, that we have obtained a relation between the
spectrum and the motion of precisely the same character as in the
simple case of the unperturbed hydrogen atom. We have here a
similar correspondence between the harmonic component in the
motion, corresponding to definite values for $\tau$~and $\kappa$ in formula~\Eq{(22)},
and the transition between two stationary states for which $n' - n'' = \tau$
and $k' - k'' = \kappa$.

A number of interesting results can be obtained from this
correspondence by considering the motion in more detail. Each
harmonic component in expression~\Eq{(22)} for which $\tau + \kappa$ is an even
number corresponds to a linear oscillation parallel to the direction
of the electric field, while each component for which $\tau + \kappa$ is odd
corresponds to an elliptical oscillation perpendicular to this direction.
The correspondence principle suggests at once that these
facts are connected with the \emph{characteristic polarization} observed in
the Stark effect. We would anticipate that a transition for which
$(n' - n'') + (k' - k'')$ is even would give rise to a component with an
electric vector parallel to the field, while a transition for which
$(n' - n'') + (k' - k'')$ is odd would correspond to a component with an
\PageSep{42}
electric vector perpendicular to the field. These results have been
fully confirmed by experiment and correspond to the empirical rule
of polarization, which Epstein proposed in his first paper on the
Stark effect.

The applications of the correspondence principle that have so
far been described have been purely qualitative in character. It is
possible however to obtain a quantitative estimate of the relative
intensity of the various components of the Stark effect of hydrogen,
by correlating the numerical values of the coefficients~$C_{\tau,\kappa}$ in formula~\Eq{(22)}
with the probability of the corresponding transitions between
the stationary states. This problem has been treated in detail by
Kramers in a recently published dissertation. In this he gives a
thorough discussion of the application of the correspondence principle
to the question of the intensity of spectral lines.

\Section{The Zeeman effect.} The problem of the effect of a homogeneous
magnetic field upon the hydrogen lines may be treated in an
entirely analogous manner. The effect on the motion of the hydrogen
atom consists simply of the superposition of a uniform rotation
upon the motion of the electron in the unperturbed atom.
The axis of rotation is parallel with the direction of the magnetic
force, while the frequency of revolution is given by the formula
\[
\sigma = \frac{eH}{4\pi mc},
\Tag{(24)}
\]
where $H$~is the intensity of the field and $c$~the velocity of light.

Again we have a case where the perturbations are simply
periodic and where the period of the perturbations is independent
of the form and position of the orbit, and in the present case, even
of the major axis. Similar considerations apply therefore as in the
case of the Stark effect, and we must expect that the energy in the
stationary states will again be given by formula~\Eq{(19)}, if we substitute
for~$\sigma$ the value given in expression~\Eq{(24)}. This result is
also in complete agreement with that obtained by Sommerfeld and
Debye. The method they used involved the solution of the equations
of motion by the method of the separation of the variables. The
appropriate coordinates are polar ones about an axis parallel to
the field.

If we try, however, to calculate directly the effect of the field by
\PageSep{43}
means of the frequency condition~\Eq{(4)}, we immediately meet with
an apparent disagreement which for some time was regarded as a
grave difficulty for the theory. As both Sommerfeld and Debye
have pointed out, lines are not observed corresponding to every
transition between the stationary states included in the formula.
We overcome this difficulty, however, as soon as we apply the
principle of correspondence. If we consider the harmonic components
of the motion we obtain a simple explanation both of the
non-occurrence of certain transitions and of the observed polarization.
In the magnetic field each elliptic harmonic component having
the frequency~$\tau\omega$ splits up into three harmonic components owing
to the uniform rotation of the orbit. Of these one is rectilinear
with frequency~$\tau\omega$ oscillating parallel to the magnetic field, and
two are circular with frequencies $\tau\omega + \sigma$ and $\tau\omega - \sigma$ oscillating in
opposite directions in a plane perpendicular to the direction of the
field. Consequently the motion represented by formula~\Eq{(22)} contains
no components for which $\kappa$~is numerically greater than~$1$, in contrast
to the Stark effect, where components corresponding to all values
of~$\kappa$ are present. Now formula~\Eq{(23)} again applies for large values
of $n$~and~$k$, and shows the asymptotic agreement between the
frequency of the radiation and the frequency of a harmonic component
in the motion. We arrive, therefore, at the conclusion that
transitions for which $k$~changes by more than unity \Chg{can not}{cannot} occur.
The argument is similar to that by which transitions between two
distinctive states of a Planck oscillator for which the values of~$n$
in~\Eq{(1)} differ by more than unity are excluded. We must further
conclude that the various possible transitions consist of two types.
For the one type corresponding to the rectilinear component, $k$~remains
unchanged, and in the emitted radiation which possesses
the same frequency~$\nu_{0}$ as the original hydrogen line, the electric
vector will oscillate parallel with the field. For the second type,
corresponding to the circular components, $k$~will increase or decrease
by unity, and the radiation viewed in the direction of the field will
be circularly polarized and have frequencies $\nu_{0} + \sigma$ and $\nu_{0} - \sigma$ respectively.
These results agree with those of the familiar Lorentz
theory. The similarity in the two theories is remarkable, when we
recall the fundamental difference between the ideas of the quantum
theory and the ordinary theories of radiation.
\PageSep{44}

\Section{Central perturbations.} An illustration based on similar considerations
which will throw light upon the spectra of other elements
consists in finding the effect of a small perturbing field of
force radially symmetrical with respect to the nucleus. In this case
neither the form of the orbit nor the position of its plane will
change with time, and the perturbing effect of the field will simply
consist of a uniform rotation of the major axis of the orbit. The
perturbations are periodic, so that we may assume that to each
energy value of a stationary state of the unperturbed system there
belongs a series of discrete energy values of the perturbed system,
characterized by different values of a whole number~$k$. The frequency~$\sigma$
of the perturbations is equal to the frequency of rotation
of the major axis. For a given law of force for the perturbing
field we find that $\sigma$~depends both on the major axis and on the
\Chg{excentricity}{eccentricity}. The change in the energy of the stationary states,
therefore, will not be given by an expression as simple as the
second term in formula~\Eq{(19)}, but will be a function of~$k$, which is
different for different fields. It is possible, however, to characterize
by one and the same condition the motion in the stationary states
of a hydrogen atom which is perturbed by any central field. In
order to show this we must consider more closely the fixation of
the motion of a perturbed hydrogen atom.

In the stationary states of the unperturbed hydrogen atom
only the major axis of the orbit is to be regarded as fixed,
while the \Chg{excentricity}{eccentricity} may assume any value. Since the change
in the energy of the atom due to the external field of force depends
upon the form and position of its orbit, the fixation of the
energy of the atom in the presence of such a field naturally
involves a closer determination of the orbit of the perturbed
system.

Consider, for the sake of illustration, the change in the hydrogen
spectrum due to the presence of homogeneous electric and magnetic
fields which was described by equation~\Eq{(19)}. It is found that
this energy condition can be given a simple geometrical interpretation.
In the case of an electric field the distance from the
nucleus to the plane in which the centre of the orbit moves determines
the change in the energy of the system due to the presence
of the field. In the stationary states this distance is simply equal
\PageSep{45}
to $\dfrac{k}{n}$~times half the major axis of the orbit. In the case of a magnetic
field it is found that the quantity which determines the change
of energy of the system is the area of the projection of the orbit
upon a plane perpendicular to the magnetic force. In the various
stationary states this area is equal to $\dfrac{k}{n}$~times the area of a circle
whose radius is equal to half the major axis of the orbit. In the
case of a perturbing central force the correspondence between
the spectrum and the motion which is required by the quantum
theory leads now to the simple condition that in the stationary
states of the perturbed system the minor axis of the rotating orbit
is simply equal to $\dfrac{k}{n}$~times the major axis. This condition was first
derived by Sommerfeld from his general theory for the determination
of the stationary states of a central motion. It is easily shown
that this fixation of the value of the minor axis is equivalent to
the statement that the parameter~$2p$ of the elliptical orbit is given
by an expression of exactly the same form as that which gives the
major axis~$2a$ in the unperturbed atom. The only difference from
the expression for~$2a_{n}$ in~\Eq{(17)} is that $n$~is replaced by~$k$, so that
the value of the parameter in the stationary states of the perturbed
atom is given by
\[
2p_{k} = k^{2}\, \frac{h^{2}}{2\pi^{2} e^{2} m}.
\Tag{(25)}
\]
The frequency of the radiation emitted by a transition between
two stationary states determined in this way for which $n'$~and~$n''$ are
large in proportion to their difference is given by an expression
which is the same as that in equation~\Eq{(23)}, if in this case $\omega$~is the
frequency of revolution of the electron in the slowly rotating orbit
and $\sigma$~represents the frequency of rotation of the major axis.

Before proceeding further, it might be of interest to note that
this fixation of the stationary states of the hydrogen atom perturbed
by external electric and magnetic forces does not coincide in certain
respects with the theories of Sommerfeld, Epstein and Debye.
According to the theory of conditionally periodic systems the stationary
states for a system of three degrees of freedom will in general
be determined by three conditions, and therefore in these theories
\PageSep{46}
each state is characterized by three whole numbers. This would
mean that the stationary states of the perturbed hydrogen atom
corresponding to a certain stationary state of the unperturbed
hydrogen atom, fixed by one condition, should be subject to two
further conditions and should therefore be characterized by two
new whole numbers in addition to the number~$n$. But the perturbations
of the Keplerian motion are simply periodic and the
energy of the perturbed atom will therefore be fixed completely
by one additional condition. The introduction of a second condition
will add nothing further to the explanation of the phenomenon,
since with the appearance of new perturbing forces, even if
these are too small noticeably to affect the observed Zeeman and
Stark effects, the forms of motion characterized by such a condition
may be entirely changed. This is completely analogous to the
fact that the hydrogen spectrum as it is usually observed is not
noticeably affected by small forces, even when they are large enough
to produce a great change in the form and position of the orbit of
the electron.

\Section{Relativity effect on hydrogen lines.} Before leaving the hydrogen
spectrum I shall consider briefly the effect of the variation of
the mass of the electron with its velocity. In the preceding sections
I have described how external fields of force split up the hydrogen
lines into several components, but it should be noticed that these
results are only accurate when the perturbations are large in comparison
with the small deviations from a pure Keplerian motion
due to the variation of the mass of the electron with its velocity.
When the variation of the mass is taken into account the motion
of the unperturbed atom will not be exactly periodic. Instead we
obtain a motion of precisely the same kind as that occurring in the
hydrogen atom perturbed by a small central field. According to
the correspondence principle an intimate connection is to be expected
between the frequency of revolution of the major axis of the
orbit and the difference of the frequencies of the fine structure
components, and the stationary states will be those orbits whose
parameters are given by expression~\Eq{(25)}. If we now consider the
effect of external forces upon the fine structure components of the
hydrogen lines it is necessary to keep in mind that this fixation of
\PageSep{47}
the stationary states only applies to the unperturbed hydrogen
atom, and that, as mentioned, the orbits in these states are in
general already strongly influenced by the presence of external
forces, which are small compared with those with which we are
concerned in experiments on the Stark and Zeeman effects. In
general the presence of such forces will lead to a great complexity
of perturbations, and the atom will no longer possess a group of
sharply defined stationary states. The fine structure components
of a given hydrogen line will therefore become diffuse and merged
together. There are, however, several important cases where this
does not happen on account of the simple character of the perturbations.
The simplest example is a hydrogen atom perturbed
by a central force acting from the nucleus. In this case it is evident
that the motion of the system will retain its centrally symmetrical
character, and that the perturbed motion will differ from the unperturbed
motion only in that the frequency of rotation of the major
axis will be different for different values of this axis and of the
parameter. This point is of importance in the theory of the
spectra of elements of higher atomic number, since, as we shall see,
the effect of the forces originating from the inner electrons may
to a first approximation be compared with that of a perturbing
central field. We \Chg{can not}{cannot} therefore expect these spectra to exhibit
a separate effect due to the variation of the mass of the electron
of the same kind as that found in the case of the hydrogen lines.
This variation will not give rise to a splitting up into separate
components but only to small displacements in the position of the
various lines.

We obtain still another simple example in which the hydrogen
atom possesses sharp stationary states, although the change of mass
of the electron is considered, if we take an atom subject to a homogeneous
magnetic field. The effect of such a field will consist in
the superposition of a rotation of the entire system about an axis
through the nucleus and parallel with the magnetic force. It follows
immediately from this result according to the principle of correspondence
that each fine structure component must be expected
to split up into a normal Zeeman effect (Lorentz triplet). The
problem may also be solved by means of the theory of conditionally
periodic systems, since the equations of motion in the presence
\PageSep{48}
of a magnetic field, even when the change in the mass is considered,
will allow of a separation of the variables using polar
coordinates in space. This has been pointed out by Sommerfeld
and Debye.

A more complicated case arises when the atom is exposed to a
homogeneous electric field which is not so strong that the effect
due to the change in the mass may be neglected. In this case there
is no system of coordinates by which the equations of motion can
be solved by separation of the variables, and the problem, therefore,
\Chg{can not}{cannot} be treated by the theory of the stationary states of conditionally
periodic systems. A closer investigation of the perturbations,
however, shows them to be of such a character that the motion
of the electrons may be decomposed into a number of separate harmonic
components. These fall into two groups for which the direction
of oscillation is either parallel with or perpendicular to the
field. According to the principle of correspondence, therefore, we
must expect that also in this case in the presence of the field each
hydrogen line will consist of a number of sharp, polarized components.
In fact by means of the principles I have described, it is
possible to give a unique fixation of the stationary states. The
problem of the effect of a homogeneous electric field upon the fine
structure components of the hydrogen lines has been treated in
detail from this point of view by Kramers in a paper which will
soon be published. In this paper it will be shown how it appears
possible to predict in detail the manner in which the fine structure
of the hydrogen lines gradually changes into the ordinary Stark
effect as the electric intensity increases.

\Section{Theory of series spectra.} Let us now turn our attention once
more to the problem of the series spectra of elements of higher
atomic number. The general appearance of the Rydberg constant
in these spectra is to be explained by assuming that the atom is
neutral and that one electron revolves in an orbit the dimensions
of which are large in comparison with the distance of the inner electrons
from the nucleus. In a certain sense, therefore, the motion of
the outer electron may be compared with the motion of the electron
of the hydrogen atom perturbed by external forces, and the appearance
of the various series in the spectra of the other elements is
\PageSep{49}
from this point of view to be regarded as analogous to the splitting
up of the hydrogen lines into components on account of such forces.

In his theory of the structure of series spectra of the type exhibited
by the alkali metals, Sommerfeld has made the assumption
that the orbit of the outer electron to a first approximation possesses
the same character as that produced by a simple perturbing
central field whose intensity diminishes rapidly with increasing
distance from the nucleus. He fixed the motion of the external
electron by means of his general theory for the fixation of the
stationary states of a central motion. The application of this
method depends on the possibility of separating the variables in
the equations of motion. In this manner Sommerfeld was able to
calculate a number of energy values which can be arranged in rows
just like the empirical spectral terms shown in the diagram of the
sodium spectrum (\PageRef[p.]{30}). The states grouped together by Sommerfeld
in the separate rows are exactly those which were characterized
by one and the same value of~$k$ in our investigation of the
hydrogen atom perturbed by a central force. The states in the
first row of the figure (row~$S$) correspond to the value $k = 1$, those
of the second row~($P$) correspond to $k = 2$, etc. The states corresponding
to one and the same value of~$n$ are connected by dotted
lines which are continued so that their vertical asymptotes correspond
to the energy value of the stationary states of the hydrogen
atom. The fact that for a constant~$n$ and increasing values of~$k$
the energy values approach the corresponding values for the unperturbed
hydrogen atom is immediately evident from the theory
since the outer electron, for large values of the parameter of its
orbit, remains at a great distance from the inner system during the
whole revolution. The orbit will become almost elliptical and the
period of rotation of the major axis will be very large. It can be
seen, therefore, that the effect of the inner system on the energy
necessary to remove this electron from the atom must become less
for increasing values of~$k$.

These beautiful results suggest the possibility of finding laws of
force for the perturbing central field which would account for the
spectra observed. Although Sommerfeld in this way has in fact
succeeded in deriving formulae for the spectral terms which vary
with~$n$ for a constant~$k$ in agreement with Rydberg's formulae, it
\PageSep{50}
has not been possible to explain the simultaneous variation with
both $k$~and~$n$ in any actual case. This is not surprising, since it is
to be anticipated that the effect of the inner electrons on the spectrum
could not be accounted for in such a simple manner. Further
consideration shows that it is necessary to consider not only the
forces which originate from the inner electrons but also to consider
the effect of the presence of the outer electron upon the motion of
the inner electrons.

Before considering the series spectra of elements of low atomic
number I shall point out how the occurrence or non-occurrence of
certain transitions can be shown by the correspondence principle
to furnish convincing evidence in favour of Sommerfeld's assumption
about the orbit of the outer electron. For this purpose we
must describe the motion of the outer electron in terms of its harmonic
components. This is easily performed if we assume that the
presence of the inner electrons simply produces a uniform rotation
of the orbit of the outer electron in its plane. On account of this
rotation, the frequency of which we will denote by~$\sigma$, two circular
rotations with the periods $\tau\omega + \sigma$ and $\tau\omega - \sigma$ will appear in the
motion of the perturbed electron, instead of each of the harmonic
elliptical components with a period $\tau\omega$ in the unperturbed motion.
The decomposition of the perturbed motion into harmonic components
consequently will again be represented by a formula of the
type~\Eq{(22)}, in which only such terms appear for which $\kappa$~is equal
to $+1$ or~$-1$. Since the frequency of the emitted radiation in the
regions where $n$~and $k$ are large is again given by the asymptotic
formula~\Eq{(23)}, we at once deduce from the correspondence principle
that the only transitions which can take place are those for which
the values of~$k$ differ by unity. A glance at the figure for the sodium
spectrum shows that this agrees exactly with the experimental
results. This fact is all the more remarkable, since in Sommerfeld's
theory the arrangement of the energy values of the stationary
states in rows has no special relation to the possibility of transition
between these states.

\Section{Correspondence principle and conservation of angular momentum.}
Besides these results the correspondence principle suggests
that the radiation emitted by the perturbed atom must
\PageSep{51}
exhibit circular polarization. On account of the indeterminateness
of the plane of the orbit, however, this polarization \Chg{can not}{cannot} be
directly observed. The assumption of such a polarization is a matter
of particular interest for the theory of radiation emission. On
account of the general correspondence between the spectrum of
an atom and the decomposition of its motion into harmonic
components, we are led to compare the radiation emitted during
the transition between two stationary states with the radiation
which would be emitted by a harmonically oscillating
electron on the basis of the classical electrodynamics. In particular
the radiation emitted according to the classical theory
by an electron revolving in a circular orbit possesses an angular
momentum and the energy~$\Delta E$ and the angular momentum~$\Delta P$ of
the radiation emitted during a certain time are connected by the
relation
\[
\Delta E = 2\pi\omega \cdot \Delta P.
\Tag{(26)}
\]

Here $\omega$~represents the frequency of revolution of the electron,
and according to the classical theory this is equal to the frequency~$\nu$
of the radiation. If we now assume that the total energy emitted
is equal to~$h\nu$ we obtain for the total angular momentum of the
radiation
\[
\Delta P = \frac{h}{2\pi}.
\Tag{(27)}
\]

It is extremely interesting to note that this expression is equal
to the change in the angular momentum which the atom suffers in
a transition where $k$~varies by unity. For in Sommerfeld's theory
the general condition for the fixation of the stationary states of a
central system, which in the special case of an approximately
Keplerian motion is equivalent to the relation~\Eq{(25)}, asserts that
the angular momentum of the system must be equal to a whole
multiple of~$\dfrac{h}{2\pi}$, a condition which may be written in our notation
\[
P = k\, \frac{h}{2\pi}.
\Tag{(28)}
\]
We see, therefore, that this condition has obtained direct support
from a simple consideration of the conservation of angular momentum
during the emission of the radiation. I wish to emphasize
that this equation is to be regarded as a rational generalization of
\PageSep{52}
Planck's original statement about the distinctive states of a harmonic
oscillator. It may be of interest to recall that the possible
significance of the angular momentum in applications of the
quantum theory to atomic processes was first pointed out by
Nicholson on the basis of the fact that for a circular motion the
angular momentum is simply proportional to the ratio of the
kinetic energy to the frequency of revolution.

In a previous paper which I presented to the Copenhagen
Academy I pointed out that these results confirm the conclusions
obtained by the application of the correspondence principle to
atomic systems possessing radial or axial symmetry. Rubinowicz
has independently indicated the conclusions which may be obtained
directly from a consideration of conservation of angular momentum
during the radiation process. In this way he has obtained several
of our results concerning the various types of possible transitions
and the polarization of the emitted radiation. Even for systems
possessing radial or axial symmetry, however, the conclusions which
we can draw by means of the correspondence principle are of a
more detailed character than can be obtained solely from a consideration
of the conservation of angular momentum. For example,
in the case of the hydrogen atom perturbed by a central force we
can only conclude that $k$~\Chg{can not}{cannot} change by more than unity, while
the correspondence principle requires that $k$~shall vary by unity
for every possible transition and that its value cannot remain unchanged.
Further, this principle enables us not only to exclude
certain transitions as being impossible---and can from this point of
view be considered as a ``selection principle''---but it also enables
us to draw conclusions about the relative probabilities of the various
possible types of transitions from the values of the amplitudes of
the harmonic components. In the present case, for example, the
fact that the amplitudes of those circular components which rotate
in the same sense as the electron are in general greater than the
amplitudes of those which rotate in the opposite sense leads us to
expect that lines corresponding to transitions for which $k$~decreases
by unity will in general possess greater intensity than lines during
the emission of which $k$~increases by unity. Simple considerations
like this, however, apply only to spectral lines corresponding to
transitions from one and the same stationary state. In other
\PageSep{53}
cases when we wish to estimate the relative intensities of two
spectral lines it is clearly necessary to take into consideration the
relative number of atoms which are present in each of the two
stationary states from which the transitions start. While the intensity
naturally \Chg{can not}{cannot} depend upon the number of atoms in the
final state, it is to be noticed, however, that in estimating the
probability of a transition between two stationary states it is necessary
to consider the character of the motion in the final as well as
in the initial state, since the values of the amplitudes of the components
of oscillation of both states are to be regarded as decisive
for the probability.

To show how this method can be applied I shall return for a
moment to the problem which I mentioned in connection with
Strutt's experiment on the resonance radiation of sodium vapour.
This involved the discussion of the relative probability of the various
possible transitions which can start from that state corresponding
to the second term in the second row of the figure on \PageRef[p.]{30}. These
were transitions to the first and second states in the first row and
to the first state in the third row, and the results of experiment
indicate, as we saw, that the probability is greatest for the second
transitions. These transitions correspond to those harmonic components
having frequencies $2\omega + \sigma$, $\omega + \sigma$ and~$\sigma$, and it is seen
that only for the second transition do the amplitudes of the corresponding
harmonic component differ from zero in the initial as
well as in the final state. [In the next essay the reader will find
that the values of quantum number~$n$ assigned in \Fig{1} to the
various stationary states must be altered. While this correction
in no way influences the other conclusions in this essay it involves
that the reasoning in this passage \Chg{can not}{cannot} be maintained.]

I have shown how the correspondence between the spectrum of
an element and the motion of the atom enables us to understand
the limitations in the direct application of the combination principle
in the prediction of spectral lines. The same ideas give an immediate
explanation of the interesting discovery made in recent years
by Stark and his collaborators, that certain \emph{new series of combination
lines} appear with considerable intensity when the radiating
atoms are subject to a strong external electric field. This phenomenon
is entirely analogous to the appearance of the so-called
\PageSep{54}
combination tones in acoustics. It is due to the fact that the
perturbation of the motion will not only consist in an effect upon
the components originally present, but in addition will give rise to
new components. The frequencies of these new components may be
$\tau\omega + \kappa\sigma$, where $\kappa$~is different from~$\pm1$. According to the correspondence
principle we must therefore expect that the electric field will
not only influence the lines appearing under ordinary circumstances,
but that it will also render possible new types of transitions which
give rise to the ``new'' combination lines observed. From an estimate
of the amplitudes of the particular components in the initial
and final states it has even been found possible to account for the
varying facility with which the new lines are brought up by the
external field.

The general problem of the effect of an electric field on the spectra
of elements of higher atomic number differs essentially from the
simple Stark effect of the hydrogen lines, since we are here concerned
not with the perturbation of a purely periodic system, but
with the effect of the field on a periodic motion already subject to
a perturbation. The problem to a certain extent resembles the
effect of a weak electric force on the fine structure components of
the hydrogen atom. In much the same way the effect of an electric
field upon the series spectra of the elements may be treated directly
by investigating the perturbations of the external electron. A
continuation of my paper in the Transactions of the Copenhagen
Academy will soon appear in which I shall show how this method
enables us to understand the interesting observations Stark and
others have made in this field.


\Section{The spectra of helium and lithium.} We see that it has been
possible to obtain a certain general insight into the origin of the
series spectra of a type like that of sodium. The difficulties encountered
in an attempt to give a detailed explanation of the
spectrum of a particular element, however, become very serious,
even when we consider the spectrum of helium whose neutral atom
contains only two electrons. The spectrum of this element has a
simple structure in that it consists of single lines or at any rate of
double lines whose components are very close together. We find,
however, that the lines fall into two groups each of which can be
\PageSep{55}
described by a formula of the type~\Eq{(14)}. These are usually called
the (ortho) helium and parhelium spectra. While the latter consists
of simple lines, the former possesses narrow doublets. The
discovery that helium, as opposed to the alkali metals, possesses
two complete spectra of the Rydberg type which do not exhibit any
mutual combinations was so surprising that at times there has been
a tendency to believe that helium consisted of two elements. This
way out of the difficulty is no longer open, since there is no room
for another element in this region of the periodic system, or more
correctly expressed, for an element possessing a new spectrum. The
existence of the two spectra can, however, be traced back to the fact
that in the stationary states corresponding to the series spectra we
have to do with a system possessing only one inner electron and in
consequence the motion of the inner system, in the absence of the
outer electron, will be simply periodic and therefore easily perturbed
by external forces.

In order to illustrate this point we shall have to consider more
carefully the stationary states connected with the origin of a series
spectrum. We must assume that in these states one electron revolves
in an orbit outside the nucleus and the other electrons. We
might now suppose that in general a number of different groups of
such states might exist, each group corresponding to a different
stationary state of the inner system considered by itself. Further
consideration shows, however, that under the usual conditions of
excitation those groups have by far the greatest probability for which
the motion of the inner electrons corresponds to the ``normal'' state
of the inner system, \ie\ to that stationary state having the least
energy. Further the energy required to transfer the inner system
from its normal state to another stationary state is in general very
large compared with the energy which is necessary to transfer an
electron from the normal state of the neutral atom to a stationary
orbit of greater dimensions. Lastly the inner system is in general
capable of a permanent existence only in its normal state. Now,
the configuration of an atomic system in its stationary states and
also in the normal state will, in general, be completely determined.
We may therefore expect that the inner system under the influence
of the forces arising from the presence of the outer electron can in
the course of time suffer only small changes. For this reason we
\PageSep{56}
must assume that the influence of the inner system upon the motion
of the external electron will, in general, be of the same character
as the perturbations produced by a constant external field upon
the motion of the electron in the hydrogen atom. We must therefore
expect a spectrum consisting of an ensemble of spectral terms,
which in general form a connected group, even though in the
absence of external perturbing forces not every combination actually
occurs. The case of the helium spectrum, however, is quite different
since here the inner system contains only one electron the motion
of which in the absence of the external electron is simple periodic
provided the small changes due to the variation in the mass of the
electron with its velocity are neglected. For this reason the form of
the orbit in the stationary states of the inner system considered by
itself will not be determined. In other words, the stability of the
orbit is so slight, even if the variation in the mass is taken into
account, that small external forces are in a position to change the
\Chg{excentricity}{eccentricity} in the course of time to a finite extent. In this case,
therefore, it is possible to have several groups of stationary states,
for which the energy of the inner system is approximately the same
while the form of the orbit of the inner electron and its position
relative to the motion of the other electrons are so essentially
different, that no transitions between the states of different groups
can occur even in the presence of external forces. It can be seen
that these conclusions summarize the experimental observations
on the helium spectra.

These\Pagelabel{56} considerations suggest an investigation of the nature of
the perturbations in the orbit of the inner electron of the helium
atom, due to the presence of the external electron. A discussion
of the helium spectrum from this point of view has recently been
given by Landé. The results of this work are of great interest particularly
in the demonstration of the large back effect on the outer
electron due to the perturbations of the inner orbit which themselves
arise from the presence of the outer electron. Nevertheless, it can
scarcely be regarded as a satisfactory explanation of the helium
spectrum. Apart from the serious objections which may be raised
against his calculation of the perturbations, difficulties arise if we
try to apply the correspondence principle to Landé's results in
order to account for the occurrence of two distinct spectra showing
\PageSep{57}
no mutual combinations. To explain this fact it seems necessary
to base the discussion on a more thorough investigation of the
mutual perturbations of the outer and the inner orbits. As a
result of these perturbations both electrons move in such an
extremely complicated way that the stationary states \Chg{can not}{cannot} be
fixed by the methods developed for conditionally periodic systems.
Dr~Kramers and I have in the last few years been engaged in such
an investigation, and in an address on atomic problems at the
meeting of the Dutch Congress of Natural and Medical Sciences
held in Leiden, April 1919, I gave a short communication of our
results. For various reasons we have up to the present time been
prevented from publishing, but in the very near future we hope to
give an account of these results and of the light which they seem
to throw upon the helium spectrum.

The problem presented by the spectra of elements of higher
atomic number is simpler, since the inner system is better defined
in its normal state. On the other hand the difficulty of the mechanical
problem of course increases with the number of the particles in
the atom. We obtain an example of this in the case of lithium
with three electrons. The differences between the spectral terms
of the lithium spectrum and the corresponding spectral terms of
hydrogen are very small for the variable term of the principal series
($k = 2$) and for the diffuse series ($k = 3$), on the other hand it is very
considerable for the variable term of the sharp series ($k = 1$). This
is very different from what would be expected if it were possible to
describe the effect of the inner electron by a central force varying
in a simple manner with the distance. This must be because the
parameter of the orbit of the outer electron in the stationary states
corresponding to the terms of the sharp series is not much greater
than the linear dimensions of the orbits of the inner electrons.
According to the principle of correspondence the frequency of rotation
of the major axis of the orbit of the outer electron is to be regarded
as a measure of the deviation of the spectral terms from the corresponding
hydrogen terms. In order to calculate this frequency it
appears necessary to consider in detail the mutual effect of all three
electrons, at all events for that part of the orbit where the outer
electron is very close to the other two electrons. Even if we assumed
that we were fully acquainted with the normal state of the inner
\PageSep{58}
system in the absence of the outer electron---which would be
expected to be similar to the normal state of the neutral helium
atom---the exact calculation of this mechanical problem would
evidently form an exceedingly difficult task.

\Section{Complex structure of series lines.} For the spectra of elements
of still higher atomic number the mechanical problem which has to
be solved in order to describe the motion in the stationary states
becomes still more difficult. This is indicated by the extraordinarily
complicated structure of many of the observed spectra. The fact that
the series spectra of the alkali metals, which possess the simplest
structure, consist of double lines whose separation increases with
the atomic number, indicates that here we have to do with systems
in which the motion of the outer electron possesses in general a
somewhat more complicated character than that of a simple central
motion. This gives rise to a more complicated ensemble of stationary
states. It would, however, appear that in the sodium atom the major
axis and the parameter of the stationary states corresponding to
each pair of spectral terms are given approximately by formulae
\Eq{(17)} and~\Eq{(25)}. This is indicated not only by the similar part played
by the two states in the experiments on the resonance radiation of
sodium vapour, but is also shown in a very instructive manner by
the peculiar effect of magnetic fields on the doublets. For small
fields each component splits up into a large number of sharp lines
instead of into the normal Lorentz triplet. With increasing field
strength Paschen and Back found that this \emph{anomalous Zeeman effect}
changed into the normal Lorentz triplet of a single line by a gradual
fusion of the components.

This effect of a magnetic field upon the doublets of the alkali
spectrum is of interest in showing the intimate relation of the components
and confirms the reality of the simple explanation of the
general structure of the spectra of the alkali metals. If we may
again here rely upon the correspondence principle we have unambiguous
evidence that the effect of a magnetic field on the motion
of the electrons simply consists in the superposition of a uniform
rotation with a frequency given by equation~\Eq{(24)} as in the case of
the hydrogen atom. For if this were the case the correspondence
principle would indicate under all conditions a normal Zeeman effect
\PageSep{59}
for each component of the doublets. I want to emphasize that the
difference between the simple effect of a magnetic field, which the
theory predicts for the fine structure of components of the hydrogen
lines, and the observed effect on the alkali doublets is in no way to
be considered as a contradiction. The fine structure components
are not analogous to the individual doublet components, but each
single fine structure component corresponds to the ensemble of
components (doublet, triplet) which makes up one of the series lines
in Rydberg's scheme. The occurrence in strong fields of the effect
observed by Paschen and Back must therefore be regarded as a
strong support for the theoretical prediction of the effect of a magnetic
field on the fine structure components of the hydrogen lines.

It does not appear necessary to assume the ``anomalous'' effect
of small fields on the doublet components to be due to a failure of
the ordinary electrodynamical laws for the description of the motion
of the outer electron, but rather to be connected with an effect of
the magnetic field on that intimate interaction between the motion
of the inner and outer electrons which is responsible for the occurrence
of the doublets. Such a view is probably not very different
from the ``coupling theory'' by which Voigt was able to account
formally for the details of the anomalous Zeeman effect. We might
even expect it to be possible to construct a theory of these effects
which would exhibit a formal analogy with the Voigt theory similar
to that between the quantum theory of the normal Zeeman effect and
the theory originally developed by Lorentz. Time unfortunately
does not permit me to enter further into this interesting problem, so
I must refer you to the continuation of my paper in the Transactions
of the Copenhagen Academy, which will contain a general discussion
of the origin of series spectra and of the effects of electric and
magnetic fields.


\Chapter{IV.}{Conclusion}

In this lecture I have purposely not considered the question of
the structure of atoms and molecules although this is of course most
intimately connected with the kind of spectral theory I have developed.
We are encouraged to use results obtained from the spectra,
since even the simple theory of the hydrogen spectrum gives a
value for the major axis of the orbit of the electron in the normal
\PageSep{60}
state ($n = 1$) of the same order of magnitude as that derived from
the kinetic theory of gases. In my first paper on the subject I
attempted to sketch a theory of the structure of atoms and of
molecules of chemical compounds. This theory was based on a
simple generalization of the results for the stationary states of the
hydrogen atom. In several respects the theory was supported by
experiment, especially in the general way in which the properties
of the elements change with increasing atomic number, shown most
clearly by Moseley's results. I should like, however, to use this
occasion to state, that in view of the recent development of the
quantum theory, many of the special assumptions will certainly have
to be changed in detail. This has become clear from various sides
by the lack of agreement of the theory with experiment. It appears
no longer possible to justify the assumption that in the normal
states the electrons move in orbits of special geometrical simplicity,
like ``electronic rings.'' Considerations relating to the stability of
atoms and molecules against external influences and concerning the
possibility of the formation of an atom by successive addition of
the individual electrons compel us to claim, first that the configurations
of electrons are not only in mechanical equilibrium
but also possess a certain stability in the sense required by
ordinary mechanics, and secondly that the configurations employed
must be of such a nature that transitions to these from other
stationary states of the atom are possible. These requirements are
not in general fulfilled by such simple configurations as electronic
rings and they force us to look about for possibilities of more complicated
motions. It will not be possible here to consider further
these still open questions and I must content myself by referring
to the discussion in my forthcoming paper. In closing, however,
I should like to emphasize once more that in this lecture I have
only intended to bring out certain general points of view lying at
the basis of the spectral theory. In particular it was my intention
to show that, in spite of the fundamental differences between these
points of view and the ordinary conceptions of the phenomena of
radiation, it still appears possible on the basis of the general correspondence
between the spectrum and the motion in the atom to
employ these conceptions in a certain sense as guides in the investigation
of the spectra.
\PageSep{61}


\Essay{III}{The Structure of~the~Atom and the~Physical
and~Chemical~Properties of~the~Elements}
{Address delivered before a joint meeting of the Physical and Chemical
Societies in Copenhagen, October~18, 1921.}

\Chapter{I.}{Preliminary}

In an address which I delivered to you about a year ago I
described the main features of a theory of atomic structure which
I shall attempt to develop this evening. In the meantime this
theory has assumed more definite form, and in two recent letters
%[** TN: Footnote mark before punctuation in the original]
to \Title{Nature} I have given a somewhat further sketch of the development.\footnote
  {\Title{Nature}, March~24, and October~13, 1921.}
The results which I am about to present to you are
of no final character; but I hope to be able to show you how this
view renders a correlation of the various properties of the elements
in such a way, that we avoid the difficulties which previously
appeared to stand in the way of a simple and consistent explanation.
Before proceeding, however, I must ask your forbearance if initially
I deal with matters already known to you, but in order to introduce
you to the subject it will first be necessary to give a brief
description of the most important results which have been obtained
in recent years in connection with the work on atomic structure.

\Section{The nuclear atom.} The conception of atomic structure which
will form the basis of all the following remarks is the so-called
nuclear atom according to which an atom is assumed to consist of
a nucleus surrounded by a number of electrons whose distances
from one another and from the nucleus are very large compared
to the dimensions of the particles themselves. The nucleus
possesses almost the entire mass of the atom and has a positive
charge of such a magnitude that the number of electrons in a
neutral atom is equal to the number of the element in the periodic
system, the so-called \emph{atomic number}. This idea of the atom, which
is due principally to Rutherford's fundamental researches on radioactive
substances, exhibits extremely simple features, but just this
simplicity appears at first sight to present difficulties in explaining
the properties of the elements. When we treat this question on
\PageSep{62}
the basis of the ordinary mechanical and electrodynamical theories
it is impossible to find a starting point for an explanation of the
marked properties exhibited by the various elements, indeed not
even of their permanency. On the one hand the particles of the
atom apparently could not be at rest in a state of stable equilibrium,
and on the other hand we should have to expect that every motion
which might be present would give rise to the emission of electromagnetic
radiation which would not cease until all the energy of
the system had been emitted and all the electrons had fallen into
the nucleus. A method of escaping from these difficulties has now
been found in the application of ideas belonging to the quantum
theory, the basis of which was laid by Planck in his celebrated
work on the law of temperature radiation. This represented a
radical departure from previous conceptions since it was the first
instance in which the assumption of a discontinuity was employed
in the formulation of the general laws of nature.

\Section{The postulates of the quantum theory.}\Pagelabel{62} The quantum theory
in the form in which it has been applied to the problems of atomic
structure rests upon two postulates which have a direct bearing
on the difficulties mentioned above. According to the first postulate
there are certain states in which the atom can exist without
emitting radiation, although the particles are supposed to have an
accelerated motion relative to one another. These \emph{stationary states}
are, in addition, supposed to possess a peculiar kind of stability, so
that it is impossible either to add energy to or remove energy from
the atom except by a process involving a transition of the atom
into another of these states. According to the second postulate
each emission of radiation from the atom resulting from such a
transition always consists of a train of purely harmonic waves.
The frequency of these waves does not depend directly upon the
motion of the atom, but is determined by a \emph{frequency relation},
according to which the frequency multiplied by the universal constant
introduced by Planck is equal to the total energy emitted
during the process. For a transition between two stationary states
for which the values of the energy of the atom before and after the
emission of radiation are $E'$~and $E''$ respectively, we have therefore
\[
h\nu = E' - E'',
\Tag{(1)}
\]
\PageSep{63}
where $h$~is Planck's constant and $\nu$~is the frequency of the emitted
radiation. Time does not permit me to give a systematic survey
of the quantum theory, the recent development of which has gone
hand in hand with its applications to atomic structure. I shall
therefore immediately proceed to the consideration of those applications
of the theory which are of direct importance in connection
with our subject.

\Section{Hydrogen atom.} We shall commence by considering the
simplest atom conceivable, namely, an atom consisting of a nucleus
and one electron. If the charge on the nucleus corresponds to that
of a single electron and the system consequently is neutral we have
a hydrogen atom. Those developments of the quantum theory which
have made possible its application to atomic structure started with
the interpretation of the well-known simple spectrum emitted by
hydrogen. This spectrum consists of a series of lines, the frequencies
of which are given by the extremely simple Balmer formula
\[
\nu = K\left(\frac{1}{(n'')^{2}} - \frac{1}{(n')^{2}}\right),
\Tag{(2)}
\]
where $n''$~and $n'$ are integers. According to the quantum theory
we shall now assume that the atom possesses a series of stationary
states characterized by a series of integers, and it can be seen how
the frequencies given by formula~\Eq{(2)} may be derived from the
frequency relation if it is assumed that a hydrogen line is connected
with a radiation emitted during a transition between two
of these states corresponding to the numbers $n'$~and~$n''$, and if the
energy in the $n$th~state apart from an arbitrary additive constant
is supposed to be given by the formula
\[
E_{n} = -\frac{Kh}{n^{2}}.
\Tag{(3)}
\]
The negative sign is used because the energy of the atom is
measured most simply by the work required to remove the electron
to infinite distance from the nucleus, and we shall assume that the
numerical value of the expression on the right-hand side of formula~\Eq{(3)}
is just equal to this work.

As regards the closer description of the stationary states we find
that the electron will very nearly describe an ellipse with the
nucleus in the focus. The major axis of this ellipse is connected
\PageSep{64}
with the energy of the atom in a simple way, and corresponding to
the energy values of the stationary states given by formula~\Eq{(3)}
there are a series of values for the major axis~$2a$ of the orbit of the
electron given by the formula
\[
2a_{n} = \frac{n^{2} e^{2}}{hK},
\Tag{(4)}
\]
where $e$~is the numerical value of the charge of the electron and
the nucleus.

On the whole we may say that the spectrum of hydrogen shows
us the \emph{formation of the hydrogen atom}, since the stationary states
may be regarded as different stages of a process by which the electron
under the emission of radiation is bound in orbits of smaller
and smaller dimensions corresponding to states with decreasing
values of~$n$. It will be seen that this view has certain characteristic
features in common with the binding process of an electron
to the nucleus if this were to take place according to the ordinary
electrodynamics, but that our view differs from it in just such a
way that it is possible to account for the observed properties of
hydrogen. In particular it is seen that the final result of the
binding process leads to a quite definite stationary state of the
atom, namely that state for which $n = 1$. This state which corresponds
to the minimum energy of the atom will be called the
\emph{normal state} of the atom. It may be stated here that the values of
the energy of the atom and the major axis of the orbit of the
electron which are found if we put $n = 1$ in formulae \Eq{(3)} and~\Eq{(4)}
are of the same order of magnitude as the values of the firmness
of binding of electrons and of the dimensions of the atoms which
have been obtained from experiments on the electrical and mechanical
properties of gases. A more accurate check of formulae
\Eq{(3)} and~\Eq{(4)} can however not be obtained from such a comparison,
because in such experiments hydrogen is not present in the form
of simple atoms but as molecules.

The formal basis of the quantum theory consists not only of the
frequency relation, but also of conditions which permit the determination
of the stationary states of atomic systems. The latter
conditions, like that assumed for the frequency, may be regarded as
natural generalizations of that assumption regarding the interaction
between simple electrodynamic systems and a surrounding field of
\PageSep{65}
electromagnetic radiation which forms the basis of Planck's theory
of temperature radiation. I shall not here go further into the
nature of these conditions but only mention that by their means
the stationary states are characterized by a number of integers,
the so-called \emph{quantum numbers}. For a purely periodic motion like
that assumed in the case of the hydrogen atom only a single
quantum number is necessary for the determination of the stationary
states. This number determines the energy of the atom and also
the major axis of the orbit of the electron, but not its \Chg{excentricity}{eccentricity}.
The energy in the various stationary states, if the small influence
of the motion of the nucleus is neglected, is given by the following
formula:
\[
E_{n} = -\frac{2\pi^{2} N^{2} e^{4} m}{n^{2} h^{2}},
\Tag{(5)}
\]
where $e$~and $m$ are respectively the charge and the mass of the
electron, and where for the sake of subsequent applications the
charge on the nucleus has been designated by~$Ne$.

For the atom of hydrogen $N = 1$, and a comparison with
equation~\Eq{(3)} leads to the following theoretical expression for~$K$ in
formula~\Eq{(2)}, namely
\[
K = \frac{2\pi^{2} e^{4} m}{h^{3}}.
\Tag{(6)}
\]
This agrees with the empirical value of the constant for the spectrum
of hydrogen within the limit of accuracy with which the various
quantities can be determined.

\Section{Hydrogen spectrum and X-ray spectra.} If in the above
formula we put $N = 2$ which corresponds to an atom consisting of
an electron revolving around a nucleus with a double charge, we
get values for the energies in the stationary states, which are four
times larger than the energies in the corresponding states of the
hydrogen atom, and we obtain the following formula for the
spectrum which would be emitted by such an atom:
\[
\nu = 4K \left(\frac{1}{(n'')^{2}} - \frac{1}{(n')^{2}}\right).
\Tag{(7)}
\]
This formula represents certain lines which have been known for
some time and which had been attributed to hydrogen on account
of the great similarity between formulae \Eq{(2)} and~\Eq{(7)} since it had
\PageSep{66}
never been anticipated that two different substances could exhibit
properties so closely resembling each other. According to the theory
we may, however, expect that the emission of the spectrum given by~\Eq{(7)}
corresponds to the \emph{first stage of the formation of the helium atom},
\ie\ to the binding of a first electron by the doubly charged nucleus
of this atom. This interpretation has been found to agree with
more recent information. For instance it has been possible to
obtain this spectrum from pure helium. I have dwelt on this point
in order to show how this intimate connection between the properties
of two elements, which at first sight might appear quite
surprising, is to be regarded as an immediate expression of the
characteristic simple structure of the nuclear atom. A short time
after the elucidation of this question, new evidence of extraordinary
interest was obtained of such a similarity between the properties of
the elements. I refer to Moseley's fundamental researches on the
X-ray spectra of the elements. Moseley found that these spectra
varied in an extremely simple manner from one element to the
next in the periodic system. It is well known that the lines of
the X-ray spectra may be divided into groups corresponding to the
different characteristic absorption regions for X-rays discovered by
Barkla. As regards the $K$~group which contains the most penetrating
X-rays, Moseley found that the strongest line for all elements
investigated could be represented by a formula which with
a small simplification can be written
\[
\nu = N^{2} K \left(\frac{1}{1^{2}} - \frac{1}{2^{2}}\right).
\Tag{(8)}
\]
$K$~is the same constant as in the hydrogen spectrum, and $N$~the
atomic number. The great significance of this discovery lies in
the fact that it would seem firmly to establish the view that this
atomic number is equal to the number of electrons in the atom.
This assumption had already been used as a basis for work on
atomic structure and was first stated by van~den Broek. While
the significance of this aspect of Moseley's discovery was at once
clear to all, it has on the other hand been more difficult to understand
the very great similarity between the spectrum of hydrogen
and the X-ray spectra. This similarity is shown, not only by the
lines of the $K$~group, but also by groups of less penetrating X-rays.
\PageSep{67}
Thus Moseley found for all the elements he investigated that the
frequencies of the strongest line in the $L$~group may be represented
by a formula which with a simplification similar to that employed
in formula~\Eq{(8)} can be written
\[
\nu = N^{2} K \left(\frac{1}{2^{2}} - \frac{1}{3^{2}}\right).
\Tag{(9)}
\]
Here again we obtain an expression for the frequency which corresponds
to a line in the spectrum which would be emitted by the
\emph{binding of an electron to a nucleus, whose charge is~$Ne$}.

\Section{The fine structure of the hydrogen lines.} This similarity between
the structure of the X-ray spectra and the hydrogen spectrum
was still further extended in a very interesting manner by Sommerfeld's
important theory of the fine structure of the hydrogen lines.
The calculation given above of the energy in the stationary states
of the hydrogen system, where each state is characterized by a
single quantum number, rests upon the assumption that the orbit
of the electron in the atom is simply periodic. This is, however,
only approximately true. It is found that if the change in the mass
of the electron due to its velocity is taken into consideration the
orbit of the electron no longer remains a simple ellipse, but its
motion may be described as a \emph{central motion} obtained by superposing
a slow and uniform rotation upon a simple periodic motion in a
very nearly elliptical orbit. For a central motion of this kind the
stationary states are characterized by \emph{two quantum numbers}. In the
case under consideration one of these may be so chosen that to a
very close approximation it will determine the energy of the atom
in the same manner as the quantum number previously used
determined the energy in the case of a simple elliptical orbit. This
quantum number which will always be denoted by~$n$ will therefore
be called the ``principal quantum number.'' Besides this condition,
which to a very close approximation determines the major axis in the
rotating and almost elliptical orbit, a second condition will be imposed
upon the stationary states of a central orbit, namely that the angular
momentum of the electron about the centre shall be equal to a whole
multiple of Planck's constant divided by~$2\pi$. The whole number, which
occurs as a factor in this expression, may be regarded as the second
quantum number and will be denoted by~$k$. The latter condition fixes
\PageSep{68}
the \Chg{excentricity}{eccentricity} of the rotating orbit which in the case of a simple
periodic orbit was undetermined. It should be mentioned that the
possible importance of the angular momentum in the quantum theory
was pointed out by Nicholson before the application of this theory to
the spectrum of hydrogen, and that a determination of the stationary
states for the hydrogen atom similar to that employed by Sommerfeld
was proposed almost simultaneously by Wilson, although the
latter did not succeed in giving a physical application to his results.

The simplest description of the form of the rotating nearly
elliptical electronic orbit in the hydrogen atom is obtained by
considering the chord which passes through the focus and is
perpendicular to the major axis, the so-called ``parameter.'' The
length~$2p$ of this parameter is given to a very close approximation
by an expression of exactly the same form as the expression for the
major axis, except that $k$~takes the place of~$n$. Using the same
notation as before we have therefore
\[
2a = n^{2}\, \frac{h^{2}}{2\pi^{2} N e^{2} m},\quad
2p = k^{2}\, \frac{h^{2}}{2\pi^{2} N e^{2} m}.
\Tag{(10)}
\]
For each of the stationary states which had previously been denoted
by a given value of~$n$, we obtain therefore a set of stationary states
corresponding to values of~$k$ from $1$ to~$n$. Instead of the simple
formula~\Eq{(5)} Sommerfeld found a more complicated expression for
the energy in the stationary states which depends on~$k$ as well as~$n$.
Taking the variation of the mass of the electron with velocity
into account and neglecting terms of higher order of magnitude he
obtained
\[
E_{n,k} = -\frac{2\pi^{2} N^{2} e^{4} m}{n^{2} h^{2}}
  \left[1 + \frac{4\pi^{2} N^{2} e^{4}}{h^{2} c^{2}}\left(-\frac{3}{4n^{2}} + \frac{1}{nk}\right)\right],
\Tag{(11)}
\]
where $c$~is the velocity of light.

Corresponding to each of the energy values for the stationary
states of the hydrogen atom given by the simple formula~\Eq{(5)} we
obtain $n$~values differing only very little from one another, since
the second term within the bracket is very small. With the aid of
the general frequency relation~\Eq{(1)} we therefore obtain a number of
components with nearly coincident frequencies instead of each
hydrogen line given by the simple formula~\Eq{(2)}. Sommerfeld has
now shown that this calculation actually agrees with measurements
\PageSep{69}
of the fine structure. This agreement applies not only to the fine
structure of the hydrogen lines which is very difficult to measure
on account of the extreme proximity of the components, but it is
also possible to account in detail for the fine structure of the helium
lines given by formula~\Eq{(7)} which has been very carefully investigated
by Paschen. Sommerfeld in connection with this theory
also pointed out that formula~\Eq{(11)} could be applied to the X-ray
spectra. Thus he showed that in the $K$~and $L$ groups pairs of lines
appeared the differences of whose frequencies could be determined
by the expression~\Eq{(11)} for the energy in the stationary states which
correspond to the binding of a single electron by a nucleus of
charge~$Ne$.

\Section{Periodic table.} In spite of the great formal similarity between
the X-ray spectra and the hydrogen spectrum indicated by these
results a far-reaching difference must be assumed to exist between
the processes which give rise to the appearance of these two types
of spectra. While the emission of the hydrogen spectrum, like the
emission of the ordinary optical spectra of other elements, may be
assumed to be connected with the binding of an electron by an
atom, observations on the appearance and absorption of X-ray
spectra clearly indicate that these spectra are connected with a
process which may be described as a \emph{reorganization of the electronic
arrangement} after a disturbance within the atom due to the effect
of external agencies. We should therefore expect that the appearance
of the X-ray spectra would depend not only upon the direct
interaction between a single electron and the nucleus, but also on
the manner in which the electrons are arranged in the completely
formed atom.

The peculiar manner in which the properties of the elements
vary with the atomic number, as expressed in the periodic system,
provides a guide of great value in considering this latter problem.
A simple survey of this system is given in \Fig{1}. The number preceding
each element indicates the atomic number, and the elements
within the various vertical columns form the different ``periods'' of
the system. The lines, which connect pairs of elements in successive
columns, indicate homologous properties of such elements. Compared
with usual representations of the periodic system, this method,
\PageSep{70}
proposed more than twenty years ago by Julius Thomsen, of indicating
the periodic variations in the properties of the elements is
more suited for comparison with theories of atomic constitution.
The meaning of the frames round certain sequences of elements
within the later periods of the table will be explained later. They
refer to certain characteristic features of the theory of atomic
constitution.
\Figure{1}{70}[Showing properties of the elements vary with the atomic number]


In an explanation of the periodic system it is natural to assume
a division of the electrons in the atom into distinct groups
in such a manner that the grouping of the elements in the system
is attributed to the gradual formation of the groups of electrons
in the atoms as the atomic number increases. Such a grouping
\PageSep{71}
of the electrons in the atom has formed a prominent part of all
more detailed views of atomic structure ever since J.~J. Thomson's
famous attempt to explain the periodic system on the basis
of an investigation of the stability of various electronic configurations.
Although Thomson's assumption regarding the distribution
of the positive electricity in the atom is not consistent with more
recent experimental evidence, nevertheless his work has exerted
great influence upon the later development of the atomic theory on
account of the many original ideas which it contained.

With the aid of the information concerning the binding of
electrons by the nucleus obtained from the theory of the hydrogen
spectrum I attempted in the same paper in which this theory was
set forth to sketch in broad outlines a picture of the structure of
the nucleus atom. In this it was assumed that each electron in its
normal state moved in a manner analogous to the motion in
the last stages of the binding of a single electron by a nucleus.
As in Thomson's theory, it was assumed that the electrons moved
in circular orbits and that the electrons in each separate group
during this motion occupied positions with reference to one another
corresponding to the vertices of plane regular polygons. Such an
arrangement is frequently described as a distribution of the electrons
in ``rings.'' By means of these assumptions it was possible to
account for the orders of magnitude of the dimensions of the atoms
as well as the firmness with which the electrons were bound by the
atom, a measure of which may be obtained by means of experiments
on the excitation of the various types of spectra. It was not
possible, however, in this way to arrive at a detailed explanation
of the characteristic properties of the elements even after it had
become apparent from the results of Moseley and the work of
Sommerfeld and others that this simple picture ought to be extended
to include orbits in the fully formed atom characterized by
higher quantum numbers corresponding to previous stages in the
formation of the hydrogen atom. This point has been especially
emphasized by Vegard.

The difficulty of arriving at a satisfactory picture of the atom is
intimately connected with the difficulty of accounting for the pronounced
``stability'' which the properties of the elements demand.
As I emphasized when considering the formation of the hydrogen
\PageSep{72}
atom, the postulates of the quantum theory aim directly at this
point, but the results obtained in this way for an atom containing
a single electron do not permit of a direct elucidation of problems
like that of the distribution in groups of the electrons in an atom
containing several electrons. If we imagine that the electrons in
the groups of the atom are orientated relatively to one another at any
moment, like the vertices of regular polygons, and rotating in either
circles or ellipses, the postulates do not give sufficient information to
determine the difference in the stability of electronic configurations
with different numbers of electrons in the groups.

The peculiar character of stability of the atomic structure, demanded
by the properties of the elements, is brought out in an
interesting way by Kossel in two important papers. In the first
paper he shows that a more detailed explanation of the origin of
the high frequency spectra can be obtained on the basis of the
group structure of the atom. He assumes that a line in the X-ray
spectrum is due to a process which may be described as follows: an
electron is removed from the atom by some external action after
which an electron in one of the other groups takes its place; this
exchange of place may occur in as many ways as there are groups
of more loosely bound electrons. This view of the origin of the
characteristic X-rays afforded a simple explanation of the peculiar
absorption phenomena observed. It has also led to the prediction
of certain simple relations between the frequencies of the X-ray
lines from one and the same element and has proved to be a suitable
basis for the classification of the complete spectrum. However it has
not been possible to develop a theory which reconciles in a satisfactory
way Sommerfeld's work on the fine structure of the X-ray
lines with Kossel's general scheme. As we shall see later the
adoption of a new point of view when considering the stability of
the atom renders it possible to bring the different results in a natural
way in connection with one another.

In his second paper Kossel investigates the possibilities for an
explanation of the periodic system on the basis of the atomic theory.
Without entering further into the problem of the causes of the
division of the electrons into groups, or the reasons for the different
stability of the various electronic configurations, he points out in
connection with ideas which had already played a part in Thomson's
\PageSep{73}
theory, how the periodic system affords evidence of a periodic appearance
of especially stable configurations of electrons. These configurations
appear in the neutral atoms of elements occupying the
final position in each period in \Fig{1}, and the stability in question is
assumed in order to explain not only the inactive chemical properties
of these elements but also the characteristic active properties of the
immediately preceding or succeeding elements. If we consider for
instance an inactive gas like argon, the atomic number of which is~$18$,
we must assume that the $18$~electrons in the atom are arranged in
an exceedingly regular configuration possessing a very marked
stability. The pronounced electronegative character of the preceding
element, chlorine, may then be explained by supposing the neutral
atom which contains only $17$~electrons to possess a tendency to
capture an additional electron. This gives rise to a negative chlorine
ion with a configuration of $18$~electrons similar to that occurring
in the neutral argon atom. On the other hand the marked electropositive
character of potassium may be explained by supposing
one of the $19$~electrons in the neutral atom to be as it were superfluous,
and that this electron therefore is easily lost; the rest of the
atom forming a positive ion of potassium having a constitution similar
to that of the argon atom. In a corresponding manner it is possible
to account for the electronegative and electropositive character of
elements like sulphur and calcium, whose atomic numbers are $16$ and~$20$.
In contrast to chlorine and potassium these elements are divalent,
and the stable configuration of $18$~electrons is formed by the addition
of two electrons to the sulphur atom and by the loss of two electrons
from the calcium atom. Developing these ideas Kossel has succeeded
not only in giving interesting explanations of a large number of
chemical facts, but has also been led to certain general conclusions
about the grouping of the electrons in elements belonging to the
first periods of the periodic system, which in certain respects are
in conformity with the results to be discussed in the following
paragraphs. Kossel's\Pagelabel{73} work was later continued in an interesting
manner by Ladenburg with special reference to the grouping of the
electrons in atoms of elements belonging to the later periods of the
periodic table. It will be seen that Ladenburg's conclusions also
exhibit points of similarity with the results which we shall discuss
later.
\PageSep{74}

\Section{Recent atomic models.} Up to the present time it has not been
possible to obtain a satisfactory account based upon a consistent application
of the quantum theory to the nuclear atom of the ultimate
cause of the pronounced stability of certain arrangements of electrons.
Nevertheless it has been apparent for some time that the solution
should be sought for by investigating the possibilities of a \emph{spatial
distribution of the electronic orbits} in the atom instead of limiting
the investigation to configurations in which all electrons belonging
to a particular group move in the same plane as was assumed for
simplicity in my first papers on the structure of the atom. The
necessity of assuming a spatial distribution of the configurations
of electrons has been drawn attention to by various writers. Born
and Landé, in connection with their investigations of the structure
and properties of crystals, have pointed out that the assumption of
spatial configurations appears necessary for an explanation of these
properties. Landé has pursued this question still further, and as
will be mentioned later has proposed a number of different ``spatial
atomic models'' in which the electrons in each separate group of
the atom at each moment form configurations possessing regular
polyhedral symmetry. These models constitute in certain respects
a distinct advance, although they have not led to decisive results
on questions of the stability of atomic structure.

The importance of spatial electronic configurations has, in addition,
been pointed out by Lewis and Langmuir in connection with their
atomic models. Thus Lewis, who in several respects independently
came to the same conclusions as Kossel, suggested that the number~$8$
characterizing the first groups of the periodic system might indicate
a constitution of the outer atomic groups where the electrons
within each group formed a configuration like the corners of a cube.
He emphasized how a configuration of this kind leads to instructive
models of the molecular structure of chemical combinations. It is
to be remarked, however, that such a ``static'' model of electronic
configuration will not be possible if we assume the forces within
the atom to be due exclusively to the electric charges of the
particles. Langmuir, who has attempted to develop Lewis' conceptions
still further and to account not only for the occurrence of
the first octaves, but also for the longer periods of the periodic
system, supposes therefore the structure of the atoms to be governed
\PageSep{75}
by forces whose nature is unknown to us. He conceives the atom
to possess a ``cellular structure,'' so that each electron is in advance
assigned a place in a cell and these cells are arranged in shells in
such a manner, that the various shells from the nucleus of the atom
outward contain exactly the same number of places as the periods
in the periodic system proceeding in the direction of increasing
atomic number. Langmuir's work has attracted much attention
among chemists, since it has to some extent thrown light on the
conceptions with which empirical chemical science is concerned.
On his theory the explanation of the properties of the various
elements is based on a number of postulates about the structure of
the atoms formulated for that purpose. Such a descriptive theory
is sharply differentiated from one where an attempt is made to
explain the specific properties of the elements with the aid of
general laws applying to the interaction between the particles in
each atom. The principal task of this lecture will consist in an
attempt to show that an advance along these lines appears by no
means hopeless, but on the contrary that with the aid of a consistent
application of the postulates of the quantum theory it
actually appears possible to obtain an insight into the structure
and stability of the atom.


\Chapter{II.}{Series Spectra and the Capture of Electrons
by\protect~Atoms}

We attack the problem of atomic constitution by asking the
question: ``How may an atom be formed by the successive capture
and binding of the electrons one by one in the field of force surrounding
the nucleus?''

Before attempting to answer this question it will first be
necessary to consider in more detail what the quantum theory
teaches us about the general character of the binding process. We
have already seen how the hydrogen spectrum gives us definite
information about the course of this process of binding the electron
by the nucleus. In considering the formation of the atoms of other
elements we have also in their spectra sources for the elucidation
of the formation processes, but the direct information obtained in
this way is not so complete as in the case of the hydrogen atom.
For an element of atomic number~$N$ the process of formation may
\PageSep{76}
be regarded as occurring in $N$~stages, corresponding with the successive
binding of $N$~electrons in the field of the nucleus. A spectrum
must be assumed to correspond to each of these binding processes;
but only for the first two elements, hydrogen and helium, do we
possess a detailed knowledge of these spectra. For other elements
of higher atomic number, where several spectra will be connected
with the formation of the atom, we are at present acquainted with
only two types, called the ``arc'' and ``spark'' spectra respectively,
according to the experimental conditions of excitation. Although
these spectra show a much more complicated structure than the
hydrogen spectrum, given by formula~\Eq{(2)} and the helium spectrum
given by formula~\Eq{(7)}, nevertheless in many cases it has been
possible to find simple laws for the frequencies exhibiting a close
analogy with the laws expressed by these formulae.

\Section{Arc and spark spectra.} If for the sake of simplicity we disregard
the complex structure shown by the lines of most spectra
(occurrence of doublets, triplets etc.), the frequency of the lines of
many arc spectra can be represented to a close approximation by
the Rydberg formula
\[
\nu = \frac{K}{(n'' + \alpha_{k''})^{2}} - \frac{K}{(n' + \alpha_{k'})^{2}},
\Tag{(12)}
\]
where $n'$~and $n''$ are integral numbers, $K$~the same constant as in
the hydrogen spectrum, while $\alpha_{k'}$~and $\alpha_{k''}$ are two constants belonging
to a set characteristic of the element. A spectrum with a
structure of this kind is, like the hydrogen spectrum, called a series
spectrum, since the lines can be arranged into series in which the
frequencies converge to definite limiting values. These series are
for example represented by formula~\Eq{(12)} if, using two definite
constants for $\alpha_{k''}$~and~$\alpha_{k'}$, $n''$~remains unaltered, while $n'$~assumes a
series of successive, gradually increasing integral values.

Formula~\Eq{(12)} applies only approximately, but it is always found
that the frequencies of the spectral lines can be written, as in
formulae \Eq{(2)} and~\Eq{(12)}, as a difference of two functions of integral
numbers. Thus the latter formula applies accurately, if the
quantities~$\alpha_{k}$ are not considered as constants, but as representatives
of a set of series of numbers~$\alpha_{k}(n)$ characteristic of the element,
whose values for increasing~$n$ within each series quickly approach
\PageSep{77}
a constant limiting value. The fact that the frequencies of the
spectra always appear as the difference of two terms, the so-called
``spectral terms,'' from the combinations of which the complete
spectrum is formed, has been pointed out by Ritz, who with the
establishment of the combination principle has greatly advanced
the study of the spectra. The quantum theory offers an immediate
interpretation of this principle, since, according to the frequency
relation we are led to consider the lines as due to transitions
between stationary states of the atom, just as in the hydrogen
spectrum, only in the spectra of the other elements we have to do
not with a single series of stationary states, but with a set of such
series. From formula~\Eq{(12)} we thus obtain for an arc spectrum---if
we temporarily disregard the structure of the individual lines---information
about an ensemble of stationary states, for which the
energy of the atom in the $n$th~state of the $k$th~series is given by
\[
E_{k}(n) = -\frac{Kh}{(n + \alpha_{k})^{2}}
\Tag{(13)}
\]
very similar to the simple formula~\Eq{(3)} for the energy in the stationary
states of the hydrogen atom.

As regards the spark spectra, the structure of which has been
cleared up mainly by Fowler's investigations, it has been possible
in the case of many elements to express the frequencies approximately
by means of a formula of exactly the same type as~\Eq{(12)},
only with the difference that~$K$, just as in the helium spectrum
given by formula~\Eq{(7)}, is replaced by a constant, which is four times
as large. For the spark spectra, therefore, the energy values in the
corresponding stationary states of the atom will be given by an
expression of the same type as~\Eq{(13)}, only with the difference that
$K$~is replaced by~$4K$.

This remarkable similarity between the structure of these types
of spectra and the simple spectra given by \Eq{(2)}~and~\Eq{(7)} is explained
simply by assuming the arc spectra to be connected with the \emph{last
stage in the formation of the neutral atom} consisting in the capture
and binding of the $N$th~electron. On the other hand the spark
spectra are connected with the \emph{last stage but one in the formation
of the atom}, namely the binding of the $(N - 1)$th~electron. In these
cases the field of force in which the electron moves will be much
\PageSep{78}
the same as that surrounding the nucleus of a hydrogen or helium
atom respectively, at least in the earlier stages of the binding
process, where during the greater part of its revolution it moves
at a distance from the nucleus which is large in proportion to the
dimensions of the orbits of the electrons previously bound. From
analogy with formula~\Eq{(3)} giving the stationary states of the
hydrogen atom, we shall therefore assume that the numerical value
of the expression on the right-hand side of~\Eq{(13)} will be equal to the
work required to remove the last captured electron from the atom,
the binding of which gives rise to the arc spectrum of the element.

\Section{Series diagram.} While the origin of the arc and spark spectra
was to this extent immediately interpreted on the basis of the
original simple theory of the hydrogen spectrum, it was Sommerfeld's
theory of the fine structure of the hydrogen lines which first gave
us a clear insight into the characteristic difference between the
hydrogen spectrum and the spark spectrum of helium on the one
hand, and the arc and spark spectra of other elements on the other.
When we consider the binding not of the first but of the subsequent
electrons in the atom, the orbit of the electron under consideration---at
any rate in the latter stages of the binding process where the
electron last bound comes into intimate interaction with those
previously bound---will no longer be to a near approximation a
closed ellipse, but on the contrary will to a first approximation be a
central orbit of the same type as in the hydrogen atom, when we
take into account the change with velocity in the mass of the
electron. This motion, as we have seen, may be resolved into a
plane periodic motion upon which a uniform rotation is superposed
in the plane of the orbit; only the superposed rotation will in this
case be comparatively much more rapid and the deviation of the
periodic orbit from an ellipse much greater than in the case of the
hydrogen atom. For an orbit of this type the stationary states, just
as in the theory of the fine structure, will be determined by two
quantum numbers which we shall denote by $n$~and~$k$, connected in
a very simple manner with the kinematic properties of the orbit.
For brevity I shall only mention that while the quantum number~$k$
is connected with the value of the constant angular momentum
of the electron about the centre in the simple manner previously
\PageSep{79}
indicated, the determination of the principal quantum number~$n$
requires an investigation of the whole course of the orbit and for
an arbitrary central orbit will not be related in a simple way to
the dimensions of the rotating periodic orbit if this deviates essentially
from a Keplerian ellipse.
\Figure{2}{79}[a survey of the origin of the sodium spectrum]

These results are represented in \Fig{2} which is a reproduction
of an illustration I have used on a previous occasion
(see Essay~II, \PageRef{30}), and which gives a survey of the origin
of the sodium spectrum. The black dots represent the stationary
states corresponding to the various series of spectral terms,
shown on the right by the letters $S$,~$P$,~$D$ and~$B$. These letters
correspond to the usual notations employed in spectroscopic
literature and indicate the nature of the series (sharp series,
principal series, diffuse series, etc.)\ obtained by combinations of
the corresponding spectral terms. The distances of the separate
points from the vertical line at the right of the figure are proportional
to the numerical value of the energy of the atom given
by equation~\Eq{(13)}. The oblique, black arrows indicate finally the
transitions between the stationary states giving rise to the
appearance of the lines in the commonly observed sodium
spectrum. The values of $n$~and $k$ attached to the various states
indicate the quantum numbers, which, according to Sommerfeld's
theory, from a preliminary consideration might be regarded as
characterizing the orbit of the outer electron. For the sake of
convenience the states which were regarded as corresponding to
the same value of~$n$ are connected by means of dotted lines, and these
are so drawn that their vertical asymptotes correspond to the
\PageSep{80}
terms in the hydrogen spectrum which belong to the same value
of the principal quantum number. The course of the curves illustrates
how the deviation from the hydrogen terms may be expected
to decrease with increasing values of~$k$, corresponding to states,
where the minimum distance between the electron in its revolution
and the nucleus constantly increases.

It should be noted that even though the theory represents the
principal features of the structure of the series spectra it has not
yet been possible to give a detailed account of the spectrum of any
element by a closer investigation of the electronic orbits which may
occur in a simple field of force possessing central symmetry. As
I have mentioned already the lines of most spectra show a complex
structure. In the sodium spectrum for instance the lines of the
principal series are doublets indicating that to each $P$-term not
one stationary state, but two such states correspond with slightly
different values of the energy. This difference is so little that
it would not be recognizable in a diagram on the same scale as
\Fig{2}. The appearance of these doublets is undoubtedly due to
the small deviations from central symmetry of the field of force
originating from the inner system in consequence of which the
general type of motion of the external electron will possess a
more complicated character than that of a simple central motion.
As a result the stationary states must be characterized by more
than two quantum numbers, in the same way that the occurrence
of deviations of the orbit of the electron in the hydrogen atom from
a simple periodic orbit requires that the stationary states of this
atom shall be characterized by more than one quantum number.
Now the rules of the quantum theory lead to the introduction of
a third quantum number through the condition that the resultant
angular momentum of the atom, multiplied by~$2\pi$, is equal to an
entire multiple of Planck's constant. This determines the orientation
of the orbit of the outer electron relative to the axis of the
inner system.

In this way Sommerfeld, Landé and others have shown that it
is possible not only to account in a formal way for the complex
structure of the lines of the series spectra, but also to obtain a
promising interpretation of the complicated effect of external
magnetic fields on this structure. We shall not enter here on these
\PageSep{81}
problems but shall confine ourselves to the problem of the fixation
of the two quantum numbers $n$~and~$k$, which to a first approximation
describe the orbit of the outer electron in the stationary
states, and whose determination is a matter of prime importance
in the following discussion of the formation of the atom. In
the determination of these numbers we at once encounter difficulties
of a profound nature, which---as we shall see---are intimately
connected with the question of the remarkable stability of atomic
structure. I shall here only remark that the values of the quantum
number~$n$, given in the figure, undoubtedly \Chg{can not}{cannot} be retained,
neither for the~$S$ nor the $P$~series. On the other hand, so far as
the values employed for the quantum number~$k$ are concerned, it
may be stated with certainty, that the interpretation of the properties
of the orbits, which they indicate, is correct. A starting
point for the investigation of this question has been obtained from
considerations of an entirely different kind from those previously
mentioned, which have made it possible to establish a close connection
between the motion in the atom and the appearance of
spectral lines.

\Section{Correspondence principle.} So far as the principles of the
quantum theory are concerned, the point which has been emphasized
hitherto is the radical departure of these principles from our
usual conceptions of mechanical and electrodynamical phenomena.
As I have attempted to show in recent years, it appears
possible, however, to adopt a point of view which suggests that the
quantum theory may, nevertheless, be regarded as a rational
generalization of our ordinary conceptions. As may be seen from
the postulates of the quantum theory, and particularly the frequency
relation, a direct connection between the spectra and the motion
of the kind required by the classical dynamics is excluded, but at
the same time the form of these postulates leads us to another
relation of a remarkable nature. Let us consider an electrodynamic
system and inquire into the nature of the radiation which would
result from the motion of the system on the basis of the ordinary
conceptions. We imagine the motion to be decomposed into purely
harmonic oscillations, and the radiation is assumed to consist of
the simultaneous emission of series of electromagnetic waves
\PageSep{82}
possessing the same frequency as these harmonic components and
intensities which depend upon the amplitudes of the components.
An investigation of the formal basis of the quantum theory shows
us now, that it is possible to trace the question of the origin of the
radiation processes which accompany the various transitions back
to an investigation of the various harmonic components, which
appear in the motion of the atom. The possibility, that a particular
transition shall occur, may be regarded as being due to the
presence of a definitely assignable ``corresponding'' component in
the motion. This principle of correspondence at the same time
throws light upon a question mentioned several times previously,
namely the relation between the number of quantum numbers,
which must be used to describe the stationary states of an atom,
and the types to which the orbits of the electrons belong. The
classification of these types can be based very simply on a decomposition
of the motion into its harmonic components. Time does
not permit me to consider this question any further, and I shall
confine myself to a statement of some simple conclusions, which
the correspondence principle permits us to draw concerning the
occurrence of transitions between various pairs of stationary states.
These conclusions are of decisive importance in the subsequent
argument.

The simplest example of such a conclusion is obtained by
considering an atomic system, which contains a particle describing
a \emph{purely periodic orbit}, and where the stationary states are characterized
by a single quantum number~$n$. In this case the motion
can according to Fourier's theorem be decomposed into a simple
series of harmonic oscillations whose frequency may be written~$\tau\omega$,
where $\tau$~is a whole number, and $\omega$~is the frequency of revolution
in the orbit. It can now be shown that a transition between two
stationary states, for which the values of the quantum number are
respectively equal to $n'$~and~$n''$, will correspond to a harmonic
component, for which $\tau = n' - n''$. This throws at once light upon
the remarkable difference which exists between the possibilities
of transitions between the stationary states of a hydrogen atom
on the one hand and of a simple system consisting of an electric
particle capable of executing simple harmonic oscillations about a
position of equilibrium on the other. For the latter system, which
\PageSep{83}
is frequently called a Planck oscillator, the energy in the stationary
states is determined by the familiar formula $E = nh\omega$, and with the
aid of the frequency relation we obtain therefore for the radiation
which will be emitted during a transition between two stationary
states $\nu = (n' - n'') \omega$. Now, an important assumption, which is not
only essential in Planck's theory of temperature radiation, but
which also appears necessary to account for the molecular absorption
in the infra-red region of radiation, states that a harmonic oscillator
will only emit and absorb radiation, for which the frequency~$\nu$ is
equal to the frequency of oscillation~$\omega$ of the oscillator. We are
therefore compelled to assume that in the case of the oscillator
transitions can occur only between stationary states which are
characterized by quantum numbers differing by only one unit,
while in the hydrogen spectrum represented by formula~\Eq{(2)} all
possible transitions could take place between the stationary states
given by formula~\Eq{(5)}. From the point of view of the principle of
correspondence it is seen, however, that this apparent difficulty is
explained by the occurrence in the motion of the hydrogen atom,
as opposed to the motion of the oscillator, of harmonic components
corresponding to values of~$\tau$, which are different from~$1$; or using
a terminology well known from acoustics, there appear overtones
in the motion of the hydrogen atom.

Another simple example of the application of the correspondence
principle is afforded by a \emph{central motion}, to the investigation of
which the explanation of the series spectra in the first approximation
may be reduced. Referring once more to the figure of the
sodium spectrum, we see that the black arrows, which correspond
to the spectral lines appearing under the usual conditions of
excitation, only connect pairs of points in consecutive rows. Now
it is found that this remarkable limitation of the occurrence of
combinations between spectral terms may quite naturally be
explained by an investigation of the harmonic components into
which a central motion can be resolved. It can readily be shown
that such a motion can be decomposed into two series of harmonic
components, whose frequencies can be expressed by $\tau\omega + \sigma$ and
$\tau\omega - \sigma$ respectively, where $\tau$~is a whole number, $\omega$~the frequency of
revolution in the rotating periodic orbit and $\sigma$~the frequency of the
superposed rotation. These components correspond with transitions
\PageSep{84}
where the principal number~$n$ decreases by $\tau$~units, while the
quantum number~$k$ decreases or increases, respectively, by one
unit, corresponding exactly with the transitions indicated by the
black arrows in the figure. This may be considered as a very
important result, because we may say, that the quantum theory,
which for the first time has offered a simple interpretation of the
fundamental principle of combination of spectral lines has at the
same time removed the mystery which has hitherto adhered
to the application of this principle on account of the apparent
capriciousness of the appearance of predicted combination lines.
Especially attention may be drawn to the simple interpretation
which the quantum theory offers of the appearance observed by
Stark and his collaborators of certain new series of lines, which do
not appear under ordinary circumstances, but which are excited
when the emitting atoms are subject to intense external electric
fields. In fact, on the correspondence principle this is immediately
explained from an examination of the perturbations in the motion
of the outer electron which give rise to the appearance in this
motion---besides the harmonic components already present in a
simple central orbit---of a number of constituent harmonic vibrations
of new type and of amplitudes proportional to the intensity
of the external forces.

It may be of interest to note that an investigation of the
limitation of the possibility of transitions between stationary
states, based upon a simple consideration of conservation of angular
momentum during the process of radiation, does not, contrary to
what has previously been supposed (compare Essay~II, \PageRef{62}),
suffice to throw light on the remarkably simple structure of series
spectra illustrated by the figure. As mentioned above we must
assume that the ``complexity'' of the spectral terms, corresponding
to given values of $n$~and~$k$, which we witness in the fine
structure of the spectral lines, may be ascribed to states, corresponding
to different values of this angular momentum, in
which the plane of the electronic orbit is orientated in a different
manner, relative to the configuration of the previously bound
electrons in the atom. Considerations of conservation of angular
momentum can, in connection with the series spectra, therefore only
contribute to an understanding of the limitation of the possibilities
\PageSep{85}
of combination observed in the peculiar laws applying to the
number of components in the complex structure of the lines. So
far as the last question is concerned, such considerations offer a
direct support for the consequences of the correspondence principle.


\Chapter{III.}{Formation of Atoms and the Periodic Table}

A correspondence has been shown to exist between the motion
of the electron last captured and the occurrence of transitions
between the stationary states corresponding to the various stages
of the binding process. This fact gives a point of departure for a
choice between the numerous possibilities which present themselves
when considering the formation of the atoms by the successive
capture and binding of the electrons. Among the processes which
are conceivable and which according to the quantum theory might
occur in the atom we shall reject those whose occurrence \Chg{can not}{cannot} be
regarded as consistent with a correspondence of the required nature.

\Section{First Period. Hydrogen---Helium.} It will not be necessary to
concern ourselves long with the question of the constitution of the
hydrogen atom. From what has been said previously we may assume
that the final result of the process of \emph{binding of the first electron} in
any atom will be a stationary state, where the energy of the atom
is given by~\Eq{(5)}, if we put $n = 1$, or more precisely by formula~\Eq{(11)},
if we put $n = 1$ and $k = 1$. The orbit of the electron will be a circle
whose radius will be given by formulae~\Eq{(10)}, if $n$~and $k$ are each
put equal to~$1$. Such an orbit will be called a $1$-quantum orbit,
and in general an orbit for which the principal quantum number
has a given value~$n$ will be called an $n$-quantum orbit. Where it
is necessary to differentiate between orbits corresponding to various
values of the quantum number~$k$, a central orbit, characterized by
given values of the quantum numbers $n$~and~$k$, will be referred to
as an $n_{k}$~orbit.

In the question of the constitution of the helium atom we meet
the much more complicated problem of the \emph{binding of the second
electron}. Information about this binding process may, however, be
obtained from the arc spectrum of helium. This spectrum, as
opposed to most other simple spectra, consists of two complete
systems of lines with frequencies given by formulae of the type~\Eq{(12)}.
\PageSep{86}
On this account helium was at first assumed to be a mixture
of two different gases, ``orthohelium'' and ``parhelium,'' but now we
know that the two spectra simply mean that the binding of the second
electron can occur in two different ways. A theoretical explanation of
the main features of the helium spectrum has recently been attempted
in an interesting paper by Landé. He supposes the emission of the
orthohelium spectrum to be due to transitions between stationary
states where both electrons move in the same plane and revolve
in the same sense. The parhelium spectrum, on the other hand, is
ascribed by him to stationary states where the planes of the orbits
form an angle with each other. Dr~Kramers and I have made a
closer investigation of the interaction between the two orbits in
the different stationary states. The results of our investigation
which was begun several years before the appearance of Landé's
work have not yet been published. Without going into details
I may say, that even though our results in several respects differ
materially from those of Landé (compare Essay~II, \PageRef{56}), we agree
with his general conclusions concerning the origin of the orthohelium
and parhelium spectra.

The final result of the binding of the second electron is intimately
related to the origin of the two helium spectra. Important
information on this point has been obtained recently by Franck
and his co-workers. As is well known he has thrown light upon
many features of the structure of the atom and of the origin
of spectra by observing the effect of bombarding atoms by
electrons of various velocities. A short time ago these experiments
showed that the impact of electrons could bring helium into a
``metastable'' state from which the atom cannot return to its
normal state by means of a simple transition accompanied by the
emission of radiation, but only by means of a process analogous to
a chemical reaction involving interaction with atoms of other
elements. This result is closely connected with the fact that the
binding of the second electron can occur in two different ways, as
is shown by the occurrence of two distinct spectra. Thus it is
evident from Franck's experiments that the normal state of the
atom is the last stage in the binding process involving the emission
of the parhelium spectrum by which the electron last captured as
well as the one first captured will be bound in a $1_{1}$~orbit. The
\PageSep{87}
metastable state, on the contrary, is the final stage of the process
giving the orthohelium spectrum. In this case the second electron,
as opposed to the first, will move in a $2_{1}$~orbit. This corresponds to
a firmness of binding which is about six times less than for the
electron in the normal state of the atom.

If we now consider somewhat more closely this apparently
surprising result, it is found that a clear grasp of it may be obtained
from the point of view of correspondence. It can be shown that
the coherent class of motions to which the orthohelium orbits
belong does not contain a $1_{1}$~orbit. If on the whole we would claim
the existence of a state where the two electrons move in $1_{1}$~orbits
in the same plane, and if in addition it is claimed that the motion
should possess the periodic properties necessary for the definition
of stationary states, then there seems that no possibility is afforded
other than the assumption that the two electrons move around the
nucleus in one and the same orbit, in such a manner that at each
moment they are situated at the ends of a diameter. This extremely
simple ring-configuration might be expected to correspond to
the firmest possible binding of the electrons in the atom, and it
was on this account proposed as a model for the helium atom in
my first paper on atomic structure. If, however, we inquire about
the possibility of a transition from one of the orthohelium states
to a configuration of this type we meet conditions which are very
different from those which apply to transitions between two of
the orthohelium orbits. In fact, the occurrence of each of these
transitions is due to the existence of well-defined corresponding
constituent harmonic vibration in the central orbits which the outer
electron describes in the class of motions to which the stationary
states belong. The transition we have to discuss, on the other
hand, is one by which the last captured electron is transferred from
a state in which it is moving ``outside'' the other to a state in which
it moves round the nucleus on equal terms with the other electron.
Now it is impossible to find a series of simple intermediate forms
for the motion of those two electrons in which the orbit of the last
captured electron exhibits a sufficient similarity to a central motion
that for this transition there could be a correspondence of the
necessary kind. It is therefore evident, that where the two electrons
move in the same plane, the electron captured last \Chg{can not}{cannot} be
\PageSep{88}
bound firmer than in a $2_{1}$~orbit. If, on the other hand, we consider
the binding process which accompanies the emission of the parhelium
spectrum and where the electrons in the stationary states move in
orbits whose planes form angles with one another we meet essentially
different conditions. A corresponding intimate change in the
interaction between the electron last captured and the one previously
bound is not required here for the two electrons in the atom to
become equivalent. We may therefore imagine the last stage of
the binding process to take place in a manner similar to those
stages corresponding to transitions between orbits characterized by
greater values of $n$~and~$k$.

In the \emph{normal state of the helium atom} the two electrons must
be assumed to move in equivalent $1_{1}$~orbits. As a first approximation
these may be described as two circular orbits, whose planes make
an angle of~$120\mbox{\textdegree}$ with one another, in agreement with the conditions
which the angular momentum of an atom according to the quantum
theory must satisfy. On account of the interaction between the
two electrons these planes at the same time turn slowly around
the fixed impulse axis of the atom. Starting from a distinctly
different point of view Kemble has recently suggested a similar
model for the helium atom. He has at the same time directed
attention to a possible type of motion of very marked symmetry
in which the electrons during their entire revolution assume
symmetrical positions with reference to a fixed axis. Kemble has
not, however, investigated this motion further. Previous to the
appearance of this paper Kramers had commenced a closer investigation
of precisely this type of motion in order to find out to what
extent it was possible from such a calculation to account for the
firmness with which the electrons are bound in the helium atom,
that is to account for the ionization potential. Early measurements
of this potential had given values corresponding approximately to
that which would result from the ring-configuration already mentioned.
This requires $17/8$~as much work to remove a single
electron as is necessary to remove an electron from the hydrogen
atom in its normal state. As the theoretical value for the latter
amount of work---which for the sake of simplicity will be represented
by~$W$---corresponds to an ionization potential of $13.53$~volts,
the ionization potential of helium would be expected to be $28.8$~volts.
\PageSep{89}
Recent and more accurate determinations, however, have
given a value for the ionization potential of helium which is considerably
lower and lies in the neighbourhood of $25$~volts. This
showed therefore the untenability of the ring-configuration quite
independently of any other considerations. A careful investigation of
the spatial atomic configuration requires elaborate calculation, and
Kramers has not yet obtained final results. With the approximation
to which they have been so far completed the calculations point to
the possibility of an agreement with the experimental results. The
final result may be awaited with great interest, since it offers in
the simplest case imaginable a test of the principles by which we
are attempting to determine stationary states of atoms containing
more than one electron.

Hydrogen and helium, as seen in the survey of the periodic
system given in \Fig{1}, together form the first period in the system
of elements, since helium is the first of the inactive gases. The great
difference in the chemical properties of hydrogen and helium is
closely related to the great difference in the nature of the binding
of the electron. This is directly indicated by the spectra and
ionization potentials. While helium possesses the highest known
ionization potential of all the elements, the binding of the electron
in the hydrogen atom is sufficiently loose to account for the tendency
of hydrogen to form positive ions in aqueous solutions and chemical
combinations. Further consideration of this particular question
requires, however, a comparison between the nature and firmness
of the electronic configurations of other atoms, and it can therefore
not be discussed at the moment.

\Section{Second Period. Lithium---Neon.} When considering the atomic
structure of elements which contain more than two electrons in the
neutral atom, we shall assume first of all that what has previously
been said about the formation of the helium atom will in the main
features also apply to the capture and binding of the first two
electrons. These electrons may, therefore, in the normal state of
the atom be regarded as moving in equivalent orbits characterized
by the quantum symbol~$1_{1}$. We obtain direct information about
the \emph{binding of the third electron} from the spectrum of lithium.
This spectrum shows the existence of a number of series of
\PageSep{90}
stationary states, where the firmness with which the last captured
electron is bound is very nearly the same as in the stationary states
of the hydrogen atom. These states correspond to orbits where $k$~is
greater than or equal to~$2$, and where the last captured electron
moves entirely outside the region where the first two electrons
move. But in addition this spectrum gives us information about a
series of states corresponding to $k = 1$ in which the energy differs
essentially from the corresponding stationary states of the hydrogen
atom. In these states the last captured electron, even if it remains
at a considerable distance from the nucleus during the greater part
of its revolution, will at certain moments during the revolution
approach to a distance from the nucleus which is of the same order
of magnitude as the dimensions of the orbits of the previously
bound electrons. On this account the electrons will be bound with
a firmness which is considerably greater than that with which the
electrons are bound in the stationary states of the hydrogen atom
corresponding to the same value of~$n$.

Now as regards the lithium spectrum as well as the other alkali
spectra we are so fortunate (see \PageRef{32}) as to possess definite evidence
about the normal state of the atom from experiments on selective
absorption. In fact these experiments tell us that the first member
of the sequence of $S$-terms corresponds to this state. This term
corresponds to a strength of binding which is only a little more than
a third of that of the hydrogen atom. We must therefore conclude
that the outer electron in the normal state of the lithium atom
moves in a $2_{1}$~orbit, just as the outer electron in the metastable
state of the helium atom. The reason why the binding of the
outer electron \Chg{can not}{cannot} proceed to an orbit characterized by a smaller
value for the total quantum number may also be considered as
analogous in the two cases. In fact, a transition by which the third
electron in the lithium atom was ultimately bound in a $1_{1}$~orbit
would lead to a state in the atom in which the electron would play
an equivalent part with the two electrons previously bound. Such
a process would be of a type entirely different from the transitions
between the stationary states connected with the emission of the
lithium spectrum, and would, contrary to these, not exhibit a
correspondence with a harmonic component in the motion of the
atom.
\PageSep{91}

We obtain, therefore, a picture of the formation and structure of
the lithium atom which offers a natural explanation of the great
difference of the chemical properties of lithium from those of helium
and hydrogen. This difference is at once explained by the fact that
the firmness by which the last captured electron is bound in its
$2_{1}$~orbit in the lithium atom is only about a third of that with which
the electron in the hydrogen atom is held, and almost five times
smaller than the firmness of the binding of the electrons in the
helium atom.

What has been said here applies not alone to the formation of
the lithium atom, but may also be assumed to apply to the binding
of the third electron in every atom, so that in contrast to the first
two electrons which move in $1_{1}$~orbits this may be assumed to move
in a $2_{1}$~orbit. As regards the \emph{binding of the fourth, fifth and sixth
electrons} in the atom, we do not possess a similar guide as no simple
series spectra are known of beryllium, boron and carbon. Although
conclusions of the same degree of certainty \Chg{can not}{cannot} be reached it
seems possible, however, to arrive at results consistent with general
physical and chemical evidence by proceeding by means of considerations
of the same kind as those applied to the binding of the
first three electrons. In fact, we shall assume that the fourth, fifth
and sixth electrons will be bound in $2_{1}$~orbits. The reason why the
binding of a first electron in an orbit of this type will not prevent the
capture of the others in two quanta orbits may be ascribed to the fact
that $2_{1}$~orbits are not circular but very \Chg{excentric}{eccentric}; For example, the
$3$rd~electron cannot keep the remaining electrons away from the inner
system in the same way in which the first two electrons bound in
the lithium atom prevent the third from being bound in a
$1$-quantum orbit. Thus we shall expect that the $4$th, $5$th and $6$th
electrons in a similar way to the $3$rd will at certain moments of
their revolution enter into the region where the first two
bound electrons move. We must not imagine, however, that these
visits into the inner system take place at the same time, but
that the four electrons visit the nucleus separately at equal
intervals of time. In earlier work on atomic structure it was supposed
that the electrons in the various groups in the atom moved
in separate regions within the atom and that at each moment the
electrons within each separate group were arranged in configurations
\PageSep{92}
possessing symmetry like that of a regular polygon or polyhedron.
Among other things this involved that the electrons in each group
were supposed to be at the point of the orbit nearest the nucleus
at the same time. A structure of this kind may be described as one
where the motions of the electrons within the groups are coupled
together in a manner which is largely independent of the interaction
between the various groups. On the contrary, the characteristic
feature of a structure like that I have suggested is the \emph{intimate
coupling between the motions of the electrons in the various groups}
characterized by different quantum numbers, as well as the \emph{greater
independence in the mode of binding within one and the same group
of electrons} the orbits of which are characterized by the same
quantum number. In emphasizing this last feature I have two
points in mind. Firstly the smaller effect of the presence of previously
bound electrons on the firmness of binding of succeeding
electrons in the same group. Secondly the way in which the motions
of the electrons within the group reflect the independence both of
the processes by which the group can be formed and by which it
can be reorganized by change of position of the different electrons
in the atom after a disturbance by external forces. The last point
will be considered more closely when we deal with the origin and
nature of the X-ray spectra; for the present we shall continue the
consideration of the structure of the atom to which we are led by
the investigation of the processes connected with the successive
capture of the electrons.

The preceding considerations enable us to understand the fact
that the two elements beryllium and boron immediately succeeding
lithium can appear electropositively with $2$~and $3$~valencies respectively
in combination with other substances. For like the third
electron in the lithium atom, the last captured electrons in these
elements will be much more lightly bound than the first two
electrons. At the same time we understand why the electropositive
character of these elements is less marked than in the case of
lithium, since the electrons in the $2$-quanta orbits will be much
more firmly bound on account of the stronger field in which they
are moving. New conditions arise, however, in the case of the
next element, carbon, as this element in its typical chemical combinations
\Chg{can not}{cannot} be supposed to occur as an ion, but rather as a
\PageSep{93}
neutral atom. This must be assumed to be due not only to the great
firmness in the binding of the electrons but also to be an essential
consequence of the symmetrical configuration of the electrons.

With the binding of the $4$th, $5$th and $6$th electrons in $2_{1}$~orbits,
the spatial symmetry of the regular configuration of the orbits
must be regarded as steadily increasing, until with the binding of
the $6$th electron the orbits of the four last bound electrons may be
expected to form an exceptionally symmetrical configuration in
which the normals to the planes of the orbits occupy positions
relative to one another nearly the same as the lines from the centre
to the vertices of a regular tetrahedron. Such a configuration
of groups of $2$-quanta orbits in the carbon atom seems capable
of furnishing a suitable foundation for explaining the structure of
organic compounds. I shall not discuss this question any further,
for it would require a thorough study of the interaction between
the motions of the electrons in the atoms forming the molecule.
I might mention, however, that the types of molecular models to
which we are led are very different from the molecular models
which were suggested in my first papers. In these the chemical
``valence bonds'' were represented by ``electron rings'' of the same
type as those which were assumed to compose the groups of
electrons within the individual atoms. It is nevertheless possible
to give a general explanation of the chemical properties of the
elements without touching on those matters at all. This is largely
due to the fact that the structures of combinations of atoms of the
same element and of many organic compounds do not have the
same significance for our purpose as those molecular structures in
which the individual atoms occur as electrically charged ions. The
latter kind of compounds, to which the greater number of simple
inorganic compounds belong, is frequently called ``heteropolar'' and
possesses a far more typical character than the first compounds
which are called ``homoeopolar,'' and whose properties to quite a
different degree exhibit the individual peculiarities of the elements.
My main purpose will therefore be to consider the fitness which
the configurations of the electrons in the various atoms offer for
the formation of ions.

Before leaving the carbon atom I should mention, that a model
of this atom in which the orbits of the four most lightly bound
\PageSep{94}
electrons possess a pronounced tetrahedric symmetry had already
been suggested by Landé. In order to agree with the measurements
of the size of the atoms he also assumed that these electrons moved
in $2_{1}$~orbits. There is, however, this difference between Landé's
view and that given here, that while Landé deduced the characteristic
properties of the carbon atom solely from an investigation of
the simplest form of motion which four electrons can execute
employing spatial symmetry, our view originates from a consideration
of the stability of the whole atom. For our assumptions about
the orbits of the electrons are based directly on an investigation of
the interaction between these electrons and the first two bound
electrons. The result is that our model of the carbon atom has
dynamic properties which are essentially different from the properties
of Landé's model.

In order to account for the properties of \emph{the elements in the second
half of the second period} it will first of all be necessary to show
why the configuration of ten electrons occurring in the neutral atom
of neon possesses such a remarkable degree of stability. Previously
it has been assumed that the properties of this configuration were
due to the interaction between eight electrons which moved in
equivalent orbits outside the nucleus and an inner group of two
electrons like that in the helium atom. It will be seen, however,
that the solution must be sought in an entirely different direction.
It \Chg{can not}{cannot} be expected that \emph{the $7$th electron} will be bound in a $2_{1}$~orbit
equivalent to the orbits of the four preceding electrons. The occurrence
of five such orbits would so definitely destroy the symmetry
in the interaction of these electrons that it is inconceivable that a
process resulting in the accession of a fifth electron to this group
would be in agreement with the correspondence principle. On the
contrary it will be necessary to assume that the four electrons in
their exceptionally symmetrical orbital configuration will keep out
later captured electrons with the result that these electrons will be
bound in orbits of other types.

The orbits which come into consideration for the $7$th electron in
the nitrogen atom and the $7$th, $8$th, $9$th and $10$th electrons in the
atoms of the immediately following elements will be circular orbits
of the type~$2_{2}$. The diameters of these orbits are considerably larger
than those of the $l_{1}$~orbits of the first two electrons; on the other
\PageSep{95}
hand the outermost part of the \Chg{excentric}{eccentric} $2_{1}$~orbits will extend some
distance beyond these circular $2_{2}$~orbits. I shall not here discuss the
capture and binding of these electrons. This requires a further investigation
of the interaction between the motions of the electrons
in the two types of $2$-quanta orbits. I shall simply mention, that
in the atom of neon in which we will assume that there are four
electrons in $2_{2}$~orbits the planes of these orbits must be regarded not
only as occupying a position relative to one another characterized
by a high degree of spatial symmetry, but also as possessing a
configuration harmonizing with the four elliptical $2_{1}$~orbits. An
interaction of this kind in which the orbital planes do not
coincide can be attained only if the configurations in both subgroups
exhibit a systematic deviation from tetrahedral symmetry.
This will have the result that the electron groups with $2$-quanta
orbits in the neon atom will have only a single axis of symmetry
which must be supposed to coincide with the axis of symmetry of
the innermost group of two electrons.

Before leaving the description of the elements within the second
period it may be pointed out that the above considerations offer a
basis for interpreting that tendency of the neutral atoms of oxygen
and fluorine for capturing further electrons which is responsible for
the marked electronegative character of these elements. In fact,
this tendency may be ascribed to the fact that the orbits of
the last captured electrons will find their place within the region,
in which the previously captured electrons move in $2_{1}$~orbits. This
suggests an explanation of the great difference between the properties
of the elements in the latter half of the second period of the
periodic system and those of the elements in the first half, in whose
atoms there is only a single type of $2$-quanta orbits.

\Section{Third Period. Sodium---Argon.} We shall now consider the
structure of atoms of elements in the third period of the periodic
system. This brings us immediately 'to the question of \emph{the binding
of the $11$th electron} in the atom. Here we meet conditions which
in some respects are analogous to those connected with the binding
of the $7$th electron. The same type of argument that applied to
the carbon atom shows that the symmetry of the configuration in
the neon atom would be essentially, if not entirely, destroyed by
\PageSep{96}
the addition of another electron in an orbit of the same type as
that in which the last captured electrons were bound. Just as in
the case of the $3$rd~and $7$th electrons we may therefore expect to
meet a new type of orbit for the 11th electron in the atom, and the
orbits which present themselves this time are the $3_{1}$~orbits. An
electron in such an orbit will for the greater part of the time remain
outside the orbits of the first ten electrons. But at certain moments
during the revolution it will penetrate not only into the region of
the $2$-quanta orbits, but like the $2_{1}$~orbits it will penetrate to
distances from the nucleus which are smaller than the radii of
the $1$-quantum orbits of the two electrons first bound. This fact,
which has a most important bearing on the stability of the atom,
leads to a peculiar result as regards the binding of the $11$th electron.
In the sodium atom this electron will move in a field which so far
as the outer part of the orbit is concerned deviates only very little
from that surrounding the nucleus in the hydrogen atom, but the
dimensions of this part of the orbit will, nevertheless, be essentially
different from the dimensions of the corresponding part of a $3_{1}$~orbit
in the hydrogen atom. This arises from the fact, that even
though the electron only enters the inner configuration of the first
ten electrons for short intervals during its revolution, this part of
the orbit will nevertheless exert an essential influence upon the
determination of the principal quantum number. This is directly
related to the fact that the motion of the electron in the first part
of the orbit deviates only a little from the motion which each of
the previously bound electrons in $2_{1}$~orbits executes during a complete
revolution. The uncertainty which has prevailed in the
determination of the quantum numbers for the stationary states
corresponding to a spectrum like that of sodium is connected with
this. This question has been discussed by several physicists. From
a comparison of the spectral terms of the various alkali metals,
Roschdestwensky has drawn the conclusion that the normal state
does not, as we might be inclined to expect a~priori, correspond to
a $1_{1}$~orbit as shown in \Fig{2} on \PageRef{79}, but that this state corresponds
to a $2_{1}$~orbit. Schrödinger has arrived at a similar result
in an attempt to account for the great difference between the
$S$~terms and the terms in the $P$~and $D$ series of the alkali spectra.
He assumes that the ``outer'' electron in the states corresponding
\PageSep{97}
to the $S$~terms---in contrast to those corresponding to the $P$~and
$D$ terms---penetrates partly into the region of the orbits of the
inner electrons during the course of its revolution. These investigations
contain without doubt important hints, but in reality the
conditions must be very different for the different alkali spectra.
Instead of a $2_{1}$~orbit as in lithium we must thus assume for
the spectrum of sodium not only that the first spectral term in
the $S$~series corresponds to a $3_{1}$~orbit, but also, as a more detailed
consideration shows, that the first term in the $P$~series corresponds
not to a $2_{2}$~orbit as indicated in \Fig{2}, but to a $3_{2}$~orbit. If the
numbers in this figure were correct, it would require among other
things that the $P$~terms should be smaller than the hydrogen terms
\Figure{3}{97}[sodium spectrum]
corresponding to the same principal quantum number. This would
mean that the average effect of the inner electrons could be described
as a repulsion greater than would occur if their total electrical charge
were united in the nucleus. This, however, \Chg{can not}{cannot} be expected from
our view of atomic structure. The fact that the last captured electron,
at any rate for low values of~$k$, revolves partly inside the orbits of the
previously bound electrons will on the contrary involve that the
presence of these electrons will give rise to a virtual repulsion
which is considerably smaller than that which would be due to
their combined charges. Instead of the curves drawn between
points in \Fig{2} which represent stationary states corresponding
to the same value of the principal quantum number running from
right to left, we obtain curves which run from left to right, as
is indicated in \Fig{3}. The stationary states are labelled with
\PageSep{98}
quantum numbers corresponding to the structure I have described.
According to the view underlying \Fig{2} the sodium spectrum
might be described simply as a distorted hydrogen spectrum,
whereas according to \Fig{3} there is not only distortion but also
complete disappearance of certain terms of low quantum numbers.
It may be stated, that this view not only appears to offer an explanation
of the magnitude of the terms, but that the complexity
of the terms in the $P$~and $D$ series finds a natural explanation in
the deviation of the configuration of the ten electrons first bound
from a purely central symmetry. This lack of symmetry has its
origin in the configuration of the two innermost electrons and
``transmits'' itself to the outer parts of the atomic structure, since
the $2_{1}$~orbits penetrate partly into the region of these electrons.

This view of the sodium spectrum provides at the same time an
immediate explanation of the pronounced electropositive properties
of sodium, since the last bound electron in the sodium atom is still
more loosely bound than the last captured electron in the lithium
atom. In this connection it might be mentioned that the increase
in atomic volume with increasing atomic number in the family of
the alkali metals finds a simple explanation in the successively
looser binding of the valency electrons. In his work on the X-ray
spectra Sommerfeld at an earlier period regarded this increase in
the atomic volumes as supporting the assumption that the principal
quantum number of the orbit of the valency electrons increases by
unity as we pass from one metal to the next in the family. His
later investigations on the series spectra have led him, however,
definitely to abandon this assumption. At first sight it might also
appear to entail a far greater increase in the atomic volume than
that actually observed. A simple explanation of this fact is however
afforded by realizing that the orbit of the electron will run
partly inside the region of the inner orbit and that therefore the
``effective'' quantum number which corresponds to the outer almost
elliptical loop will be much smaller than the principal quantum
number, by which the whole central orbit is described. It may
be mentioned that Vegard in his investigations on the X-ray spectra
has also proposed the assumption of successively increasing quantum
numbers for the electronic orbits in the various groups of the atom,
reckoned from the nucleus outward. He has introduced assumptions
\PageSep{99}
about the relations between the numbers of electrons in the various
groups of the atom and the lengths of the periods in the periodic
system which exhibit certain formal similarities with the results
presented here. But Vegard's considerations do not offer points of
departure for a further consideration of the evolution and stability
of the groups, and consequently no basis for a detailed interpretation
of the properties of the elements.

When we consider the elements following sodium in the third
period of the periodic system we meet in \emph{the binding of the $12$th,
$13$th and $14$th electrons} conditions which are analogous to those
we met in the binding of the $4$th, $5$th and $6$th electrons. In the
elements of the third periods, however, we possess a far more
detailed knowledge of the series spectra. Too little is known
about the beryllium spectrum to draw conclusions about the
binding of the fourth electron, but we may infer directly from the
well-known arc spectrum of magnesium that the $12$th electron
in the atom of this element is bound in a $3_{1}$~orbit. As regards
the binding of the $13$th electron we meet in aluminium an
absorption spectrum different in structure to that of the alkali
metals. In fact here not the lines of the principal series but the
lines of the sharp and diffuse series are absorption lines. Consequently
it is the first member of the $P$~terms and not of the $S$~terms
which corresponds to the normal state of the aluminium
atom, and we must assume that the $13$th electron is bound in
a $3_{2}$~orbit. This, however, would hardly seem to be a general
property of the binding of the $13$th electron in atoms, but rather
to arise from the special conditions for the binding of the last
electron in an atom, where already there are two other electrons
bound as loosely as the valency electron of aluminium. At the
present state of the theory it seems best to assume that in the
silicon atom the four last captured electrons will move in $3_{1}$~orbits
forming a configuration possessing symmetrical properties
similar to the outer configuration of the four electrons in $2_{1}$~orbits
in carbon. Like what we assumed for the latter configuration we
shall expect that the configuration of the $3_{1}$~orbits occurring for the
first time in silicon possesses such a completion, that the addition
of a further electron in a $3_{1}$~orbit to the atom of the following elements
is impossible, and that \emph{the $15$th electron} in the elements of
\PageSep{100}
higher atomic number will be bound in a new type of orbit. In this
case, however, the orbits with which we meet will not be circular,
as in the capture of the $7$th electron, but will be rotating \Chg{excentric}{eccentric}
orbits of the type~$3_{2}$. This is very closely related to the fact, mentioned
above, that the non-circular orbits will correspond to a
firmer binding than the circular orbits having the same value for
the principal quantum number, since the electrons will at certain
moments penetrate much farther into the interior of the atom.
Even though a $3_{2}$~orbit will not penetrate into the innermost configuration
of $1_{1}$~orbits, it will penetrate to distances from the nucleus
which are considerably less than the radii of the circular $2_{}2$~orbits.
In the case of the $16$th, $17$th and $18$th electrons the conditions are
similar to those for the $15$th. So for argon we may expect a configuration
in which the ten innermost electrons move in orbits of
the same type as in the neon atom while the last eight electrons will
form a configuration of four $3_{1}$~orbits and four $3_{2}$~orbits, whose
symmetrical properties must be regarded as closely corresponding
to the configuration of $2$-quanta orbits in the neon atom. At the
same time, as this picture suggests a qualitative explanation of the
similarity of the chemical properties of the elements in the latter
part of the second and third periods, it also opens up the possibility
of a natural explanation of the conspicuous difference from a
quantitative aspect.

\Section{Fourth Period. Potassium---Krypton.} In the fourth period
we meet at first elements which resemble chemically those at the
beginning of the two previous periods. This is also what we should
expect. We must thus assume that \emph{the $19$th electron} is bound in
a new type of orbit, and a closer consideration shows that this will
be a $4_{1}$~orbit. The points which were emphasized in connection
with the binding of the last electron in the sodium atom will be
even more marked here on account of the larger quantum number
by which the orbits of the inner electrons are characterized. In
fact, in the potassium atom the $4_{1}$~orbit of the $19$th electron will,
as far as inner loops are concerned, coincide closely with the shape
of a $3_{1}$~orbit. On this account, therefore, the dimensions of the
outer part of the orbit will not only deviate greatly from the
dimensions of a $4_{1}$~orbit in the hydrogen atom, but will coincide
\PageSep{101}
closely with a hydrogen orbit of the type~$2_{1}$, the dimensions of
which are about four times smaller than the $4_{1}$~hydrogen orbit.
This result allows an immediate explanation of the main features of
the chemical properties and the spectrum of potassium. Corresponding
results apply to calcium, in the neutral atom of which
there will be two valency electrons in equivalent $4_{1}$~orbits.

After calcium the properties of the elements in the fourth period
of the periodic system deviate, however, more and more from the
corresponding elements in the previous periods, until in the family
of the iron metals we meet elements whose properties are essentially
different. Proceeding to still higher atomic numbers we again
meet different conditions. Thus we find in the latter part of the
fourth period a series of elements whose chemical properties approach
more and more to the properties of the elements at the end
of the preceding periods, until finally with atomic number~$36$ we
again meet one of the inactive gases, namely krypton. This is
exactly what we should expect. The formation and stability of the
atoms of the elements in the first three periods require that each
of the first $18$ electrons in the atom shall be bound in each succeeding
element in an orbit of the same principal quantum number
as that possessed by the particular electron, when it first appeared.
It is readily seen that this is no longer the case for the $19$th
electron. With increasing nuclear charge and the consequent
decrease in the difference between the fields of force inside and
outside the region of the orbits of the first $18$ bound electrons, the
dimensions of those parts of a $4_{1}$~orbit which fall outside will
approach more and more to the dimensions of a $4$-quantum orbit
calculated on the assumption that the interaction between the
electrons in the atom may be neglected. \emph{With increasing atomic
number a point will therefore be reached where a $3_{3}$~orbit will correspond
to a firmer binding of the $19$th electron than a $4_{1}$~orbit}, and
this occurs as early as at the beginning of the fourth period. This
cannot only be anticipated from a simple calculation but is confirmed
in a striking way from an examination of the series spectra. While
the spectrum of potassium indicates that the $4_{1}$~orbit corresponds
to a binding which is more than twice as firm as in a $3_{3}$~orbit
corresponding to the first spectral term in the $D$~series, the conditions
are entirely different as soon as calcium is reached. We
\PageSep{102}
shall not consider the arc spectrum which is emitted during the
capture of the $20$th electron but the spark spectrum which corresponds
to the capture and binding of the $19$th electron. While the
spark spectrum of magnesium exhibits great similarity with the
sodium spectrum as regards the values of the spectral terms in the
various series---apart from the fact that the constant appearing in
formula~\Eq{(12)} is four times as large as the Rydberg constant---we
meet in the spark spectrum of calcium the remarkable condition
\Figure{4}{102}[a survey of the stationary states corresponding to the arc spectra of sodium and potassium]
that the first term of the $D$~series is larger than the first term of
the $P$~series and is only a little smaller than the first term of the
$S$~series, which may be regarded as corresponding to the binding
of the $19$th electron in the normal state of the calcium atom.
These facts are shown in \Fig[figure]{4} which gives a survey of the
stationary states corresponding to the arc spectra of sodium and
potassium. As in figures \FigNum{2} and~\FigNum{3} of the sodium spectrum, we
have disregarded the complexity of the spectral terms, and the
numbers characterizing the stationary states are simply the quantum
\PageSep{103}
numbers $n$~and~$k$. For the sake of comparison the scale in which the
energy of the different states is indicated is chosen four times as
small for the spark spectra as for the arc spectra. Consequently
the vertical lines indicated with various values of~$n$ correspond for
the arc spectra to the spectral terms of hydrogen, for the spark
spectra to the terms of the helium spectrum given by formula~\Eq{(7)}.
Comparing the change in the relative firmness in the binding of
the $19$th electron in a $4_{1}$~and $3_{3}$~orbit for potassium and calcium we
see that we must be prepared already for the next element,
scandium, to find that the $3_{3}$~orbit will correspond to a stronger
binding of this electron than a $4_{1}$~orbit. On the other hand it
follows from previous remarks that the binding will be much lighter
than for the first $18$ electrons which agrees that in chemical combinations
scandium appears electropositively with three valencies.

If we proceed to the following elements, a still larger number of
$3_{3}$~orbits will occur in the normal state of these atoms, since the
number of such electron orbits will depend upon the firmness of
their binding compared to the firmness with which an electron is
bound in a $4_{1}$~orbit, in which type of orbit at least the last captured
electron in the atom may be assumed to move. We therefore meet
conditions which are essentially different from those which we have
considered in connection with the previous periods, so that here
we have to do with \emph{the successive development of one of the inner
groups of electrons in the atom}, in this case with groups of electrons
in $3$-quanta orbits. Only when the development of this group has
been completed may we expect to find once more a corresponding
change in the properties of the elements with increasing atomic
number such as we find in the preceding periods. The properties
of the elements in the latter part of the fourth period show
immediately that the group, when completed, will possess $18$~electrons.
Thus in krypton, for example, we may expect besides
the groups of $1$,~$2$ and $3$-quanta orbits a markedly symmetrical
configuration of $8$~electrons in $4$-quanta orbits consisting of four $4_{1}$~orbits
and four $4_{2}$~orbits.

The question now arises: In which way will the gradual formation
of the group of electrons having $3$-quanta orbits take place?
From analogy with the constitution of the groups of electrons with
$2$-quanta orbits we might at first sight be inclined to suppose that
\PageSep{104}
the complete group of $3$-quanta orbits would consist of three subgroups
of four electrons each in orbits of the types $3_{1}$,~$3_{2}$ and~$3_{3}$
respectively, so that the total number of electrons would be $12$
instead of~$18$. Further consideration shows, however, that such an
expectation would not be justified. The stability of the configuration
of eight electrons with $2$-quanta orbits occurring in neon must
be ascribed not only to the symmetrical configuration of the electronic
orbits in the two subgroups of $2_{1}$~and $2_{2}$ orbits respectively,
but fully as much to the possibility of bringing the orbits inside these
subgroups into harmonic relation with one another. The situation
is different, however, for the groups of electrons with $3$-quanta
orbits. Three subgroups of four orbits each \Chg{can not}{cannot} in this case be
expected to come into interaction with one another in a correspondingly
simple manner. On the contrary we must assume that
the presence of electrons in $3_{3}$~orbits will diminish the harmony of
the orbits within the first two $3$-quanta subgroups, at any rate
when a point is reached where the $19$th electron is no longer, as
was the case with scandium, bound considerably more lightly than
the previously bound electrons in $3$-quanta orbits, but has been
drawn so far into the atom that it revolves within essentially
the same region of the atom where these electrons move. We
shall now assume that this decrease in the harmony will so to
say ``open'' the previously ``closed'' configuration of electrons
in orbits of these types. As regards the final result, the number~$18$
indicates that after the group is finally formed there will
be three subgroups containing six electrons each. Even if it has
not at present been possible to follow in detail the various
steps in the formation of the group this result is nevertheless
confirmed in an interesting manner by the fact that it is possible
to arrange three configurations having six electrons each in a simple
manner relative to one another. The configuration of the subgroups
does not exhibit a tetrahedral symmetry like the groups of $2$-quanta
orbits in carbon, but a symmetry which, so far as the relative
orientation of the normals to the planes of the orbits is concerned,
may be described as trigonal.

In spite of the great difference in the properties of the elements
of this period, compared with those of the preceding period, the
completion of the group of $18$~electrons in $3$-quanta orbits in the
\PageSep{105}
fourth period may to a certain extent be said to have the same
characteristic results as the completion of the group of $2$-quanta
orbits in the second period. As we have seen, this determined not
only the properties of neon as an inactive gas, but in addition the
electronegative properties of the preceding elements and the
electropositive properties of the elements which follow. The fact
that there is no inactive gas possessing an outer group of $18$~electrons
is very easily accounted for by the much larger dimensions
which a $3_{3}$~orbit has in comparison with a $2_{2}$~orbit revolving in the
same field of force. On this account a complete $3$-quanta group
\Chg{can not}{cannot} occur as the outermost group in a neutral atom, but only
in positively charged ions. The characteristic decrease in valency
which we meet in copper, shown by the appearance of the singly
charged cuprous ions, indicates the same tendency towards the
completion of a symmetrical configuration of electrons that we
found in the marked electronegative character of an element like
fluorine. Direct evidence that a complete group of $3$-quanta orbits
is present in the cuprous ion is given by the spectrum of copper
which, in contrast to the extremely complicated spectra of the
preceding elements resulting from the unsymmetrical character of
the inner system, possesses a simple structure very much like that
of the sodium spectrum. This may no doubt be ascribed to a
simple symmetrical structure present in the cuprous ion similar to
that in the sodium ion, although the great difference in the constitution
of the outer group of electrons in these ions is shown
both by the considerable difference in the values of the spectral
terms and in the separation of the doublets in the $P$~terms of the
two spectra. The occurrence of the cupric compounds shows, however,
that the firmness of binding in the group of $3$-quanta orbits
in the copper atom is not as great as the firmness with which the
electrons are bound in the group of $2$-quanta orbits in the sodium
atom. Zinc, which is always divalent, is the first element in which
the groups of the electrons are so firmly bound that they \Chg{can not}{cannot}
be removed by ordinary chemical processes.

The picture I have given of the formation and structure of the
atoms of the elements in the fourth period gives an explanation of
the chemical and spectral properties. In addition it is supported
by evidence of a different nature to that which we have hitherto
\PageSep{106}
used. It is a familiar fact, that the elements in the fourth period
differ markedly from the elements in the preceding periods
partly in their \emph{magnetic properties} and partly in the \emph{characteristic
colours} of their compounds. Paramagnetism and colours do occur
in elements belonging to the foregoing periods, but not in simple
compounds where the atoms considered enter as ions. Many
elements of the fourth period, on the contrary, exhibit paramagnetic
properties and characteristic colours even in dissociated
aqueous solutions. The importance of this has been emphasized
by Ladenburg in his attempt to explain the properties of the
elements in the long periods of the periodic system (see \PageRef{73}).
Langmuir in order to account for the difference between the fourth
period and the preceding periods simply assumed that the atom,
in addition to the layers of cells containing $8$~electrons each, possesses
an outer layer of cells with room for $18$~electrons which is completely
filled for the first time in the case of krypton. Ladenburg,
on the other hand, assumes that for some reason or other an
intermediate layer is developed between the inner electronic
configuration in the atom appearing already in argon, and the
external group of valency electrons. This layer commences with
scandium and is completed exactly at the end of the family of iron
metals. In support of this assumption Ladenburg not only mentions
the chemical properties of the elements in the fourth period, but
also refers to the paramagnetism and colours which occur exactly
in the elements, where this intermediate layer should be in
development. It is seen that Ladenburg's ideas exhibit certain
formal similarities with the interpretation I have given above of
the appearance of the fourth period, and it is interesting to note that
our view, based on a direct investigation of the conditions for the
formation of the atoms, enables us to understand the relation
emphasized by Ladenburg.

Our ordinary electrodynamic conceptions are probably insufficient
to form a basis for an explanation of atomic magnetism. This is
hardly to be wondered at when we remember that they have not
proved adequate to account for the phenomena of radiation which
are connected with the intimate interaction between the electric
and magnetic forces arising from the motion of the electrons. In
whatever way these difficulties may be solved it seems simplest to
\PageSep{107}
assume that the occurrence of magnetism, such as we meet in the
elements of the fourth period, results from a lack of symmetry in
the internal structure of the atom, thus preventing the magnetic
forces arising from the motion of the electrons from forming a
system of closed lines of force running wholly within the atom.
While it has been assumed that the ions of the elements in the
previous periods, whether positively or negatively charged, contain
configurations of marked symmetrical character, we must, however,
be prepared to encounter a definite lack of symmetry in the
electronic configurations in ions of those elements within the fourth
period which contain a group of electrons in $3$-quanta orbits in the
transition stage between symmetrical configurations of $8$~and $18$
electrons respectively. As pointed out by Kossel, the experimental
results exhibit an extreme simplicity, the magnetic moment of the
ions depending only on the number of electrons in the ion. Ferric
ions, for example, exhibit the same atomic magnetism as manganous
ions, while manganic ions exhibit the same atomic magnetism as
chromous ions. It is in beautiful agreement with what we have
assumed about the structure of the atoms of copper and zinc, that
the magnetism disappears with those ions containing $28$~electrons
which, as I stated, must be assumed to contain a complete group
of $3$-quanta orbits. On the whole a consideration of the magnetic
properties of the elements within the fourth period gives us a vivid
impression of how a wound in the otherwise symmetrical inner
structure is first developed and then healed as we pass from element
to element. It is to be hoped that a further investigation of the
magnetic properties will give us a clue to the way in which the
group of electrons in $3$-quanta orbits is developed step by step.

Also the colours of the ions directly support our view of atomic
structure. According to the postulates of the quantum theory
absorption as well as emission of radiation is regarded as taking
place during transitions between stationary states. The occurrence
of colours, that is to say the absorption of light in the visible region
of the spectrum, is evidence of transitions involving energy changes
of the same order of magnitude as those giving the usual optical
spectra of the elements. In contrast to the ions of the elements of
the preceding periods where all the electrons are assumed to be very
firmly bound, the occurrence of such processes in the fourth period
\PageSep{108}
is exactly what we should expect. For the development and completion
of the electronic groups with $3$-quanta orbits will proceed,
so to say, in competition with the binding of electrons in orbits of
higher quanta, since the binding of electrons in $3$-quanta orbits
occurs when the electrons in these orbits are bound more firmly
than electrons in $4_{1}$~orbits. The development of the group will
therefore proceed to the point where we may say there is equilibrium
between the two kinds of orbits. This condition may be
assumed to be intimately connected not only with the colour of the
ions, but also with the tendency of the elements to form ions with
different valencies. This is in contrast to the elements of the first
periods where the charge of the ions in aqueous solutions is always
the same for one and the same element.

\Section{Fifth Period. Rubidium---Xenon.} The structure of the atoms
in the remaining periods may be followed up in complete analogy
with what has already been said. Thus we shall assume that the
$37$th and $38$th electrons in the elements of the fifth period are
bound in $5_{1}$~orbits. This is supported by the measurements of the
arc spectrum of rubidium and the spark spectrum of strontium.
The latter spectrum indicates at the same time that $4_{3}$~orbits will
soon appear, and therefore in this period, which like the $4$th
contains $18$~elements, we must assume that we are witnessing a
\emph{further stage in the development of the electronic group of $4$-quanta
orbits}. The first stage in the formation of this group may be said
to have been attained in krypton with the appearance of a symmetrical
configuration of eight electrons consisting of two subgroups
each of four electrons in $4_{1}$~and $4_{2}$~orbits. A second preliminary
completion must be regarded as having been reached with the
appearance of a symmetrical configuration of $18$~electrons in the
case of silver, consisting of three subgroups with six electrons each
in orbits of the types $4_{1}$,~$4_{2}$ and~$4_{3}$. Everything that has been said
about the successive formation of the group of electrons with $3$-quanta
orbits applies unchanged to this stage in the transformation
of the group with $4$-quanta orbits. For in no case have we made
use of the absolute values of the quantum numbers nor of assumptions
concerning the form of the orbits but only of the number of
possible types of orbits which might come into consideration. At
\PageSep{109}
the same time it may be of interest to mention that the properties
of these elements compared with those of the foregoing period
nevertheless show a difference corresponding exactly to what would
be expected from the difference in the types of orbits. For instance,
the divergencies from the characteristic valency conditions of the
elements in the second and third periods appear later in the fifth
period than for elements in the fourth period. While an element
like titanium in the fourth period already shows a marked tendency
to occur with various valencies, on the other hand an element like
zirconium is still quadri-valent like carbon in the second period
and silicon in the third. A simple investigation of the kinematic
properties of the orbits of the electrons shows in fact that an
electron in an \Chg{excentric}{eccentric} $4_{3}$~orbit of an element in the fifth
period will be considerably more loosely bound than an electron in
a circular $3_{3}$~orbit of the corresponding element in the fourth
period, while electrons which are bound in \Chg{excentric}{eccentric} orbits of the
types $5_{1}$~and $4_{1}$ respectively will correspond to a binding of about
the same firmness.

At the end of the fifth period we may assume that xenon, the
atomic number of which is~$54$, has a structure which in addition to
the two $1$-quantum, eight $2$-quanta, eighteen $3$-quanta and eighteen
$4$-quanta orbits already mentioned contains a symmetrical
configuration of eight electrons in $5$-quanta orbits consisting of two
subgroups with four electrons each in $5_{1}$~and $5_{2}$ orbits respectively.

\Section{Sixth Period. Caesium---Niton.} If we now consider the atoms
of elements of still higher atomic number, we must first of all
assume that the $55$th and $56$th electrons in the atoms of caesium
and barium are bound in $6_{1}$~orbits. This is confirmed by the spectra
of these elements. It is clear, however, that we must be prepared
shortly to meet entirely new conditions. With increasing nuclear
charge we shall have to expect not only that an electron in a $5_{3}$~orbit
will be bound more firmly than in a $6_{1}$~orbit, but we must also
expect that a moment will arrive when during the formation of the
atom a $4_{4}$~orbit will represent a firmer binding of the electron than
an orbit of $5$~or $6$-quanta, in much the same way as in the elements
of the fourth period a new stage in the development of the $3$-quanta
group was started when a point was reached where for the first
\PageSep{110}
time the $19$th electron was bound in a $3_{3}$~orbit instead of in a $4_{1}$~orbit.
We shall thus expect in the sixth period to meet with a new
stage in the development of the group with $4$-quanta orbits. Once
this point has been reached we must be prepared to find with increasing
atomic number a number of elements following one another,
which as in the family of the iron metals have very nearly the same
properties. The similarity will, however, be still more pronounced,
since in this case we are concerned with the successive transformation
of a configuration of electrons which lies deeper in the interior
of the atom. You will have already guessed that what I have in view
is a simple explanation of the occurrence of the \emph{family of rare earths}
at the beginning of the sixth period. As in the case of the transformation
and completion of the group of $3$-quanta orbits in the fourth
period and the partial completion of groups of $4$-quanta orbits in
the fifth period, we may immediately deduce from the length of the
sixth period the number of electrons, namely~$32$, which are finally
contained in the $4$-quanta group of orbits. Analogous to what
applied to the group of $3$-quanta orbits it is probable that, when
the group is completed, it will contain eight electrons in each of the
four subgroups. Even though it has not yet been possible to follow
the development of the group step by step, we can even here give
some theoretical evidence in favour of the occurrence of a symmetrical
configuration of exactly this number of electrons. I shall
simply mention that it is not possible without coincidence of the
planes of the orbits to arrive at an interaction between four subgroups
of six electrons each in a configuration of simple trigonal
symmetry, which is equally simple as that shown by three subgroups.
The difficulties which we meet make it probable that a harmonic
interaction can be attained precisely by four groups each containing
eight electrons the orbital configurations of which exhibit axial
symmetry.

Just as in the case of the family of the iron metals in the fourth
period, the proposed explanation of the occurrence of the family of
rare earths in the sixth period is supported in an interesting
manner by an investigation of the magnetic properties of these
elements. In spite of the great chemical similarity the members
of this family exhibit very different magnetic properties, so that
while some of them exhibit but very little magnetism others exhibit
\PageSep{111}
a greater magnetic moment per atom than any other element which
has been investigated. It is also possible to give a simple interpretation
of the peculiar colours exhibited by the compounds of these
elements in much the same way as in the case of the family of iron
metals in the fourth period. The idea that the appearance of the
group of the rare earths is connected with the development of inner
groups in the atom is not in itself new and has for instance been
considered by Vegard in connection with his work on X-ray spectra.
The new feature of the present considerations lies, however, in the
emphasis laid on the peculiar way in which the relative strength of
the binding for two orbits of the same principal quantum number
but of different shapes varies with the nuclear charge and with the
number of electrons previously bound. Due to this fact the presence
of a group like that of the rare earths in the sixth period may be
considered as a direct consequence of the theory and might actually
have been predicted on a quantum theory, adapted to the explanation
of the properties of the elements within the preceding periods
in the way I have shown.

Besides \emph{the final development of the group of $4$-quanta orbits} we
observe in the sixth period in the family of the platinum metals \emph{the
second stage in the development of the group of $5$-quanta orbits}.
Also in the radioactive, chemically inactive gas niton, which completes
this period, we observe the first preliminary step in the
development of a group of electrons with $6$-quanta orbits. In the
atom of this element, in addition to the groups of electrons of two
$1$-quantum, eight $2$-quanta, eighteen $3$-quanta, thirty-two $4$-quanta
and eighteen $5$-quanta orbits respectively, there is also an outer
symmetrical configuration of eight electrons in $6$-quanta orbits,
which we shall assume to consist of two subgroups with four electrons
each in $6_{1}$~and $6_{2}$ orbits respectively.

\Section{Seventh Period.} In the seventh and last period of the periodic
system we may expect the appearance of $7$-quanta orbits in the
normal state of the atom. Thus in the neutral atom of radium in
addition to the electronic structure of niton there will be two
electrons in $7_{1}$~orbits which will penetrate during their revolution
not only into the region of the orbits of electrons possessing lower
values for the principal quantum number, but even to distances
\PageSep{112}
from the nucleus which are less than the radii of the orbits of the
innermost $1$-quantum orbits. The properties of the elements in the
seventh period are very similar to the properties of the elements in the
fifth period. Thus, in contrast to the conditions in the sixth period,
there are no elements whose properties resemble one another like
those of the rare earths. In exact analogy with what has already
been said about the relations between the properties of the elements
in the fourth and fifth periods this may be very simply explained by
the fact that an \Chg{excentric}{eccentric} $5_{4}$~orbit will correspond to a considerably
looser binding of an electron in the atom of an element of the
seventh period than the binding of an electron in a circular $4_{4}$~orbit
in the corresponding element of the sixth period, while there will be
a much smaller difference in the firmness of the binding of these
electrons in orbits of the types $7_{1}$~and $6_{1}$ respectively.

It is well known that the seventh period is not complete, for no atom
has been found having an atomic number greater than~$92$. This is
probably connected with the fact that the last elements in the
system are radioactive and that nuclei of atoms with a total charge
greater than~$92$ will not be sufficiently stable to exist under conditions
where the elements can be observed. It is tempting to
sketch a picture of the atoms formed by the capture and binding
of electrons around nuclei having higher charges, and thus to
obtain some idea of the properties which the corresponding hypothetical
elements might be expected to exhibit. I shall not develop
this matter further, however, since the general results we should
get will be evident to you from the views I have developed to
explain the properties of the elements actually observed. A survey
of these results is given in the following table, which gives a symbolical
representation of the atomic structure of the inactive gases
which complete the first six periods in the periodic system. In
order to emphasize the progressive change the table includes the
probable arrangement of electrons in the next atom which would
possess properties like the inactive gases.

The view of atomic constitution underlying this table, which
involves configurations of electrons moving with large velocities
between each other, so that the electrons in the ``outer'' groups
penetrate into the region of the orbits of the electrons of the ``inner''
groups, is of course completely different from such statical models
\PageSep{113}
of the atom as are proposed by Langmuir. But quite apart from this
it will be seen that the arrangement of the electronic groups in
the atom, to which we have been lead by tracing the way in which
each single electron has been bound, is essentially different from
the arrangement of the groups in Langmuir's theory. In order to
explain the properties of the elements of the sixth period Langmuir
assumes for instance that, in addition to the inner layers of cells
containing $2$,~$8$, $8$, $18$ and $18$ electrons respectively, which are
employed to account for the properties of the elements in the
earlier periods, the atom also possesses a layer of cells with room
for $32$~electrons which is just completed in the case of niton.

\Figure{}{113}[The inactive gases which complete the first six periods in the periodic system]

In this connection it may be of interest to mention a recent
paper by Bury, to which my attention was first drawn after the
deliverance of this address, and which contains an interesting
survey of the chemical properties of the elements based on similar
conceptions of atomic structure as those applied by Lewis and
Langmuir. From purely chemical considerations Bury arrives at
conclusions which as regards the arrangement and completion of
the groups in the main coincide with those of the present theory,
the outlines of which were given in my letters to Nature mentioned
in the introduction.

\Section{Survey of the periodic table.} The results given in this address
are also illustrated by means of the representation of the periodic
system given in \Fig{1}. In this figure the frames are meant to
indicate such elements in which one of the ``inner'' groups is
in a stage of development. Thus there will be found in the
\PageSep{114}
fourth and fifth periods a single frame indicating the final completion
of the electronic group with $3$-quanta orbits, and the
last stage but one in the development of the group with $4$-quanta
orbits respectively. In the sixth period it has been necessary to
introduce two frames, of which the inner one indicates the last
stage of the evolution of the group with $4$-quanta orbits, giving rise
to the rare earths. This occurs at a place in the periodic system
where the third stage in the development of an electronic group
with $5$-quanta orbits, indicated by the outer frame, has already
begun. In this connection it will be seen that the inner frame
encloses a smaller number of elements than is usually attributed
to the family of the rare earths. At the end of this group an
uncertainty exists, due to the fact that no element of atomic
number~$72$ is known with certainty. However, as indicated in
\Fig{1}, we must conclude from the theory that the group with
$4$-quanta orbits is finally completed in lutetium~($71$). This element
therefore ought to be the last in the sequence of consecutive
elements with similar properties in the first half of the sixth period,
and at the place~$72$ an element must be expected which in its
chemical and physical properties is homologous with zirconium and
thorium. This, which is already indited on Julius Thomsen's old
table, has also been pointed out by Bury. [Quite recently Dauvillier
has in an investigation of the X-ray spectrum excited in preparations
containing rare earths, observed certain faint lines which he ascribes
to an element of atomic number~$72$. This element is identified by
him as the element celtium, belonging to the family of rare earths,
the existence of which had previously been suspected by Urbain.
Quite apart from the difficulties which this result, if correct, might
entail for atomic theories, it would, since the rare earths according
to chemical view possess three valencies, imply a rise in positive
valency of two units when passing from the element~$72$ to the
next element~$73$, tantalum. This would mean an exception from
the otherwise general rule, that the valency never increases by
more than one unit when passing from one element to the next in
the periodic table\Add{.}] In the case of the incomplete seventh period
the full drawn frame indicates the third stage in the development
of the electronic group with $6$-quanta orbits, which must begin in
actinium. The dotted frame indicates the last stage but one in
\PageSep{115}
the development of the group with $5$-quanta orbits, which hitherto
has not been observed, but which ought to begin shortly after
uranium, if it has not already begun in this element.

With reference to the homology of the elements the exceptional
position of the elements enclosed by frames in \Fig{1} is further
emphasized by taking care that, in spite of the large similarity
many elements exhibit, no connecting lines are drawn between
two elements which occupy different positions in the system with
respect to framing. In fact, the large chemical similarity between,
for instance, aluminium and scandium, both of which are trivalent
and pronounced electropositive elements, is directly or indirectly
emphasized in the current representations of the periodic table.
While this procedure is justified by the analogous structure of the
trivalent ions of these elements, our more detailed ideas of atomic
structure suggest, however, marked differences in the physical
properties of aluminium and scandium, originating in the essentially
different character of the way in which the last three electrons
in the neutral atom are bound. This fact gives probably a direct
explanation of the marked difference existing between the spectra
of aluminium and scandium. Even if the spectrum of scandium is
not yet sufficiently cleared up, this difference seems to be of a much
more fundamental character than for instance the difference between
the arc spectra of sodium and copper, which apart from the large
difference in the absolute values of the spectral terms possess a
completely analogous structure, as previously mentioned in this
essay. On the whole we must expect that the spectra of elements
in the later periods lying inside a frame will show new features
compared with the spectra of the elements in the first three periods.
This expectation seems supported by recent work on the spectrum
of manganese by Catalan, which appeared just before the printing
of this essay.

Before I leave the interpretation of the chemical properties by
means of this atomic model I should like to remind you once again
of the fundamental principles which we have used. The whole
theory has evolved from an investigation of the way in which
electrons can be captured by an atom. The formation of an atom
was held to consist in the successive binding of electrons, this
binding resulting in radiation according to the quantum theory.
\PageSep{116}
According to the fundamental postulates of the theory this binding
takes place in stages by transitions between stationary states
accompanied by emission of radiation. For the problem of the
stability of the atom the essential problem is at what stage such a
process comes to an end. As regards this point the postulates give
no direct information, but here the correspondence principle is
brought in. Even though it has been possible to penetrate considerably
further at many points than the time has permitted me
to indicate to you, still it has not yet been possible to follow in
detail all stages in the formation of the atoms. We cannot say, for
instance, that the above table of the atomic constitution of the
inert gases may in every detail be considered as the unambiguous
result of applying the correspondence principle. On the other hand
it appears that our considerations already place the empirical data
in a light which scarcely permits of an essentially different interpretation
of the properties of the elements based upon the postulates of
the quantum theory. This applies not only to the series spectra
and the close relationship of these to the chemical properties of the
elements, but also to the X-ray spectra, the consideration of which
leads us into an investigation of interatomic processes of an entirely
different character. As we have already mentioned, it is necessary
to assume that the emission of the latter spectra is connected with
processes which may be described as a reorganization of the completely
formed atom after a disturbance produced in the interior
of the atom by the action of external forces.


\Chapter{IV.}{Reorganization of Atoms and X-Ray Spectra}

As in the case of the series spectra it has also been possible to represent
the frequency of each line in the X-ray spectrum of an element
as the difference of two of a set of spectral terms. We shall therefore
assume that each X-ray line is due to a transition between
two stationary states of the atom. The values of the atomic energy
corresponding to these states are frequently referred to as the
``energy levels'' of the X-ray spectra. The great difference between
the origin of the X-ray and the series spectra is clearly seen, however,
in the difference of the laws applying to the absorption of
radiation in the X-ray and the optical regions of the spectra. The
absorption by non-excited atoms in the latter case is connected
\PageSep{117}
with those lines in the series spectrum which correspond to combinations
of the various spectral terms with the largest of these
terms. As has been shown, especially by the investigations of
Wagner and de~Broglie, the absorption in the X-ray region, on
the other hand, is connected not with the X-ray lines but with
certain spectral regions commencing at the so-called ``absorption
edges.'' The frequencies of these edges agree very closely with the
spectral terms used to account for the X-ray lines. We shall now
see how the conception of atomic structure developed in the preceding
pages offers a simple interpretation of these facts. Let us
consider the following question: What changes in the state of the
atom can be produced by the absorption of radiation, and which
processes of emission can be initiated by such changes?

\Section{Absorption and emission of X-rays and correspondence
principle.} The possibility of producing a change at all in the
motion of an electron in the interior of an atom by means of radiation
must in the first place be regarded as intimately connected
with the character of the interaction between the electrons within
the separate groups. In contrast to the forms of motion where at
every moment the position of the electrons exhibits polygonal or
polyhedral symmetry, the conception of this interaction evolved from
a consideration of the possible formation of atoms by successive
binding of electrons has such a character that the harmonic components
in the motion of an electron are in general represented in
the resulting electric moment of the atom. As a result of this it
will be possible to release a single electron from the interaction
with the other electrons in the same group by a process which
possesses the necessary analogy with an absorption process on
the ordinary electrodynamic view claimed by the correspondence
principle. The points of view on which we based the interpretation
of the development and completion of the groups during the
formation of an atom imply, on the other hand, that just as no
additional electron can be taken up into a previously completed
group in the atom by a change involving emission of radiation,
similarly it will not be possible for a new electron to be added
to such a group, when the state of the atom is changed by
absorption of radiation. This means that an electron which belongs
\PageSep{118}
to one of the inner groups of the atom, as a consequence of an
absorption process---besides the case where it leaves the atom
completely---can only go over either to an incompleted group, or
to an orbit where the electron during the greater part of its revolution
moves at a distance from the nucleus large compared to the
distance of the other electrons. On account of the peculiar conditions
of stability which control the occurrence of incomplete groups in
the interior of the atom, the energy which is necessary to bring
about a transition to such a group will in general differ very little
from that required to remove the particular electron completely
from the atom. We must therefore assume that the energy levels
corresponding to the absorption edges indicate to a first approximation
the amount of work that is required to remove an electron
in one of the inner groups completely from the atom. The
correspondence principle also provides a basis for understanding
the experimental evidence about the appearance of the emission
lines of the X-ray spectra due to transitions between the stationary
states corresponding to these energy levels. Thus the nature of the
interaction between the electrons in the groups of the atom implies
that each electron in the atom is so to say prepared, independently
of the other electrons in the same group, to seize any opportunity
which is offered to become more firmly bound by being taken up
into a group of electrons with orbits corresponding to smaller values
of the principal quantum number. It is evident, however, that on
the basis of our views of atomic structure, such an opportunity is
always at hand as soon as an electron has been removed from one
of these groups.

At the same time that our view of the atom leads to a natural
conception of the phenomena of emission and absorption of X-rays,
agreeing closely with that by which Kossel has attempted to give
a formal explanation of the experimental observations, it also suggests
a simple explanation of those quantitative relations holding for the
frequencies of the lines which have been discovered by Moseley and
Sommerfeld. These researches brought to light a remarkable and
far-reaching similarity between the Röntgen spectrum of a given
element and the spectrum which would be expected to appear upon
the binding of a single electron by the nucleus. This similarity we
immediately understand if we recall that in the normal state of the
\PageSep{119}
atom there are electrons moving in orbits which, with certain
limitations, correspond to all stages of such a binding process and
that, when an electron is removed from its original place in the
atom, processes may be started within the atom which will correspond
to all transitions between these stages permitted by the
correspondence principle. This brings us at once out of those
difficulties which apparently arise, when one attempts to account
for the origin of the X-ray spectra by means of an atomic structure,
suited to explain the periodic system. This difficulty has been felt
to such an extent that it has led Sommerfeld for example in his
recent work to assume that the configurations of the electrons in
the various atoms of one and the same element may be different
even under usual conditions. Since, in contrast to our ideas, he
supposed all electrons in the principal groups of the atom to move
in equivalent orbits, he is compelled to assume that these groups
are different in the different atoms, corresponding to different
possible types of orbital shapes. Such an assumption, however, seems
inconsistent with an interpretation of the definite character of the
physical and chemical properties of the elements, and stands in marked
contradiction with the points of view about the stability of the atoms
which form the basis of the view of atomic structure here proposed.

\Section{X-ray spectra and atomic structure.} In this connection it is
of interest to emphasize that the group distribution of the electrons
in the atom, on which we have based both the explanation of the
periodic system and the classification of the lines in the X-ray
spectra, shows itself in an entirely different manner in these two
phenomena. While the characteristic change of the chemical
properties with atomic number is due to the gradual development
and completion of the groups of the loosest bound electrons, the
characteristic absence of almost every trace of a periodic change in
the X-ray spectra is due to two causes. Firstly the electronic
configuration of the completed groups is repeated unchanged for
increasing atomic number, and secondly the gradual way in which
the incompleted groups are developed implies that a type of orbit,
from the moment when it for the first time appears in the normal
state of the neutral atom, always will occur in this state and will
correspond to a steadily increasing firmness of binding. The development
\PageSep{120}
of the groups in the atom with increasing atomic number,
which governs the chemical properties of the elements shows itself
in the X-ray spectra mainly in the appearance of new lines. Swinne
has already referred to a connection of this kind between the periodic
system and the X-ray spectra in connection with Kossel's theory.
We can only expect a closer connection between the X-ray phenomena
and the chemical properties of the elements, when the conditions
on the surface of the atom are concerned. In agreement
with what has been brought to light by investigations on absorption
of X-rays in elements of lower atomic number, such as have
been performed in recent years in the physical laboratory at Lund,
we understand immediately that the position and eventual structure
of the absorption edges will to a certain degree depend upon
the physical and chemical conditions under which the element
investigated exists, while such a dependence does not appear in
the characteristic emission lines.

If we attempt to obtain a more detailed explanation of the
experimental observations, we meet the question of the influence
of the presence of the other electrons in the atom upon the firmness
of the binding of an electron in a given type of orbit. This influence
will, as we at once see, be least for the inner parts of the atom,
where for each electron the attraction of the nucleus is large in
proportion to the repulsion of the other electrons. It should also
be recalled, that while the relative influence of the presence of the
other electrons upon the firmness of the binding will decrease with
increasing charge of the nucleus, the effect of the variation in the
mass of the electron with the velocity upon the firmness of the
binding will increase strongly. This may be seen from Sommerfeld's
formula~\Eq{(11)}. While we obtain a fairly good agreement for the
levels corresponding to the removal of one of the innermost electrons
in the atom by using the simple formula~\Eq{(11)}, it is, however, already
necessary to take the influence of the other electrons into consideration
in making an approximate calculation of the levels corresponding
to a removal of an electron from one of the outer groups in the
atom. Just this circumstance offers us, however, a possibility of
obtaining information about the configurations of the electrons in
the interior of the atoms from the X-ray spectra. Numerous
investigations have been directed at this question both by
\PageSep{121}
Sommerfeld and his pupils and by Debye, Vegard and others. It
may also be remarked that de~Broglie and Dauvillier in a recent
paper have thought it possible to find support in the experimental
material for certain assumptions about the numbers of electrons in
the groups of the atom to which Dauvillier had been led by considerations
about the periodic system similar to those proposed by
Langmuir and Ladenburg. In calculations made in connection with
these investigations it is assumed that the electrons in the various
groups move in separate concentric regions of the atom, so that
the effect of the presence of the electrons in inner groups upon the
motion of the electrons in outer groups as a first approximation
may be expected to consist in a simple screening of the nucleus.
On our view, however, the conditions are essentially different, since
for the calculation of the firmness of the binding of the electrons
it is necessary to take into consideration that the electrons in the
more lightly bound groups in general during a certain fraction of
their revolution will penetrate into the region of the orbits of
electrons in the more firmly bound groups. On account of this
fact, many examples of which we saw in the series spectra, we \Chg{can not}{cannot}
expect to give an account of the firmness of the binding of the
separate electrons, simply by means of a ``screening correction''
consisting in the subtraction of a constant quantity from the value
for~$N$ in such formulae as \Eq{(5)} and~\Eq{(11)}. Furthermore in the calculation
of the work corresponding to the energy levels we must take
account not only of the interaction between, the electrons in the
normal state of the atom, but also of the changes in the configuration
and interaction of the remaining electrons, which establish
themselves automatically without emission of radiation during the
removal of the electron. Even though such calculations have not
yet been made very accurately, a preliminary investigation has
already shown that it is possible approximately to account for the
experimental results.

\Section{Classification of X-ray spectra.} Independently of a definite
view of atomic structure it has been possible by means of a formal
application of Kossel's and Sommerfeld's theories to disentangle
the large amount of experimental material on X-ray spectra. This
material is drawn mainly from the accurate measurements of
\PageSep{122}
Siegbahn and his collaborators. From this disentanglement of the
experimental observations, in which besides Sommerfeld and his
students especially Smekal and Coster have taken part, we have
obtained a nearly complete classification of the energy levels corresponding
to the X-ray spectra. These levels are formally referred
to types of orbits characterized by two quantum numbers $n$ and~$k$,
and certain definite rules for the possibilities of combination
between the various levels have also been found. In this way a
number of results of great interest for the further elucidation of
the origin of the X-ray spectra have been attained. First it has
not only been possible to find levels, which within certain limits
correspond to all possible pairs of numbers for $n$ and~$k$, but it has
been found that in general to each such pair more than one level
must be assigned. This result, which at first may appear very
surprising, upon further consideration can be given a simple
interpretation. We must remember that the levels depend not
only upon the constitution of the atom in the normal state, but
also upon the configurations which appear after the removal
of one of the inner electrons and which in contrast to the normal
state do not possess a uniquely completed character. If we thus
consider a process in which one of the electrons in a group
(subgroup) is removed we must be prepared to find that after the
process the orbits of the remaining electrons in this group may be
orientated in more than one way in relation to one another, and
still fulfil the conditions required of the stationary states by the
quantum theory. Such a view of the ``complexity'' of the levels, as
further consideration shows, just accounts for the manner in which
the energy difference of the two levels varies with the atomic
number. Without attempting to develop a more detailed picture
of atomic structure, Smekal has already discussed the possibility
of accounting for the multiplicity of levels. Besides referring to
the possibility that the separate electrons in the principal groups
do not move in equivalent orbits, Smekal suggests the introduction
of three quantum numbers for the description of the various groups,
but does not further indicate to what extent these quantum
numbers shall be regarded as characterizing a complexity in the
structure of the groups in the normal state itself or on the
contrary characterizing the incompleted groups which appear
when an electron is removed.
\PageSep{123}

It will be seen that the complexity of the X-ray levels exhibits a
close analogy with the explanation of the complexity of the terms
of the series spectra. There exists, however, this difference between
the complex structure of the X-ray spectra and the complex
structure of the lines in the series spectra, that in the X-ray
spectra there occur not only combinations between spectral terms,
for which $k$~varies by unity, but also between terms corresponding
to the same value of~$k$. This may be assumed to be
due to the fact, that in the X-ray spectra in contrast to the series
\Figure{5}{123}[Niton]
spectra we have to do with transitions between stationary states
where, both before and after the transition, the electron concerned
takes part in an intimate interaction with other electrons in orbits
with the same principal quantum number. Even though this
interaction may be assumed to be of such a nature that the
harmonic components which would appear in the motion of an
electron in the absence of the others will in general also appear
in the resulting moment of the atom, we must expect that the
interaction between the electrons will give rise to the appearance
in this moment of new types of harmonic components.
\PageSep{124}

It may be of interest to insert here a few words about a new
paper of Coster which appeared after this address was given,
and in which he has succeeded in obtaining an extended and
detailed connection between the X-ray spectra and the ideas
of atomic structure given in this essay. The classification mentioned
above was based on measurements of the spectra of the
heaviest elements, and the results in their complete form, which
were principally due to independent work of Coster and Wentzel,
may be represented by the diagram in \Fig{5}, which refers to
elements in the neighbourhood of niton. The vertical arrows
\Figure{6}{124}[xenon and krypton]
represent the observed lines arising from combinations between
the different energy levels which are represented by horizontal lines.
In each group the levels are arranged in the same succession as
their energy values, but their distances do not give a quantitative
picture of the actual energy-differences, since this would require a
much larger figure. The numbers~$n_{k}$ attached to the different levels
indicate the type of the corresponding orbit. The letters $a$ and~$b$
refer to the rules of combination which I mentioned. According
to these rules the possibility of combination is limited (1)~by the
exclusion of combinations, for which $k$~changes by more than one
unit, (2)~by the condition that only combinations between an $a$-
and a $b$-level can take place. The latter rule was given in this
\PageSep{125}
form by Coster; Wentzel formulated it in a somewhat different
way by the formal introduction of a third quantum number. In
his new paper Coster has established a similar classification for the
lighter elements. For the elements in the neighbourhood of xenon
and krypton he has obtained results illustrated by the diagrams
given in \Fig{6}. Just as in \Fig{5} the levels correspond exactly to
those types of orbits which, as seen from the table on \PageRef[page]{113},
according to the theory will be present in the atoms of these elements.
In xenon several of the levels present in niton have disappeared,
and in krypton still more levels have fallen away. Coster
has also investigated in which elements these particular levels
appear for the last time, when passing from higher to lower atomic
number. His results concerning this point confirm in detail the
predictions of the theory. Further he proves that the change in
the firmness of binding of the electrons in the outer groups in
the elements of the family of the rare earths shows a dependence
on the atomic number which strongly supports the assumption that
in these elements a completion of an inner group of $4$-quanta
orbits takes place. For details the reader is referred to Coster's
paper in the \Title{Philosophical Magazine}. Another important contribution
to our systematic knowledge of the X-ray spectra is
contained in a recent paper by Wentzel. He shows that various
lines, which find no place in the classification hitherto considered,
can be ascribed in a natural manner to processes of reorganization,
initiated by the removal of more than one electron from the
atom; these lines are therefore in a certain sense analogous to
the enhanced lines in the optical spectra.

\Chapter{}{Conclusion}

Before bringing this address to a close I wish once more to
emphasize the complete analogy in the application of the
quantum theory to the stability of the atom, used in explaining
two so different phenomena as the periodic system and X-ray
spectra. This point is of the greatest importance in judging the
reality of the theory, since the justification for employing considerations,
relating to the formation of atoms by successive capture
of electrons, as a guiding principle for the investigation of atomic
\PageSep{126}
structure might appear doubtful if such considerations could not
be brought into natural agreement with views on the reorganization
of the atom after a disturbance in the normal electronic
arrangement. Even though a certain inner consistency in this
view of atomic structure will be recognized, it is, however, hardly
necessary for me to emphasize the incomplete character of the
theory, not only as regards the elaboration of details, but also so
far as the foundation of the general points of view is concerned.
There seems, however, to be no other way of advance in atomic
problems than that which hitherto has been followed, namely to let
the work in these two directions go hand in hand.

%%%%%%%%%%%%%%%%%%%%%%%%% GUTENBERG LICENSE %%%%%%%%%%%%%%%%%%%%%%%%%%
\PGLicense
\begin{PGtext}
End of the Project Gutenberg EBook of The Theory of Spectra and Atomic
Constitution, by Niels (Niels Henrik David) Bohr

*** END OF THIS PROJECT GUTENBERG EBOOK THEORY OF SPECTRA ***

***** This file should be named 47464-pdf.pdf or 47464-pdf.zip *****
This and all associated files of various formats will be found in:
        http://www.gutenberg.org/4/7/4/6/47464/

Produced by Andrew D. Hwang

Updated editions will replace the previous one--the old editions
will be renamed.

Creating the works from public domain print editions means that no
one owns a United States copyright in these works, so the Foundation
(and you!) can copy and distribute it in the United States without
permission and without paying copyright royalties.  Special rules,
set forth in the General Terms of Use part of this license, apply to
copying and distributing Project Gutenberg-tm electronic works to
protect the PROJECT GUTENBERG-tm concept and trademark.  Project
Gutenberg is a registered trademark, and may not be used if you
charge for the eBooks, unless you receive specific permission.  If you
do not charge anything for copies of this eBook, complying with the
rules is very easy.  You may use this eBook for nearly any purpose
such as creation of derivative works, reports, performances and
research.  They may be modified and printed and given away--you may do
practically ANYTHING with public domain eBooks.  Redistribution is
subject to the trademark license, especially commercial
redistribution.



*** START: FULL LICENSE ***

THE FULL PROJECT GUTENBERG LICENSE
PLEASE READ THIS BEFORE YOU DISTRIBUTE OR USE THIS WORK

To protect the Project Gutenberg-tm mission of promoting the free
distribution of electronic works, by using or distributing this work
(or any other work associated in any way with the phrase "Project
Gutenberg"), you agree to comply with all the terms of the Full Project
Gutenberg-tm License available with this file or online at
  www.gutenberg.org/license.


Section 1.  General Terms of Use and Redistributing Project Gutenberg-tm
electronic works

1.A.  By reading or using any part of this Project Gutenberg-tm
electronic work, you indicate that you have read, understand, agree to
and accept all the terms of this license and intellectual property
(trademark/copyright) agreement.  If you do not agree to abide by all
the terms of this agreement, you must cease using and return or destroy
all copies of Project Gutenberg-tm electronic works in your possession.
If you paid a fee for obtaining a copy of or access to a Project
Gutenberg-tm electronic work and you do not agree to be bound by the
terms of this agreement, you may obtain a refund from the person or
entity to whom you paid the fee as set forth in paragraph 1.E.8.

1.B.  "Project Gutenberg" is a registered trademark.  It may only be
used on or associated in any way with an electronic work by people who
agree to be bound by the terms of this agreement.  There are a few
things that you can do with most Project Gutenberg-tm electronic works
even without complying with the full terms of this agreement.  See
paragraph 1.C below.  There are a lot of things you can do with Project
Gutenberg-tm electronic works if you follow the terms of this agreement
and help preserve free future access to Project Gutenberg-tm electronic
works.  See paragraph 1.E below.

1.C.  The Project Gutenberg Literary Archive Foundation ("the Foundation"
or PGLAF), owns a compilation copyright in the collection of Project
Gutenberg-tm electronic works.  Nearly all the individual works in the
collection are in the public domain in the United States.  If an
individual work is in the public domain in the United States and you are
located in the United States, we do not claim a right to prevent you from
copying, distributing, performing, displaying or creating derivative
works based on the work as long as all references to Project Gutenberg
are removed.  Of course, we hope that you will support the Project
Gutenberg-tm mission of promoting free access to electronic works by
freely sharing Project Gutenberg-tm works in compliance with the terms of
this agreement for keeping the Project Gutenberg-tm name associated with
the work.  You can easily comply with the terms of this agreement by
keeping this work in the same format with its attached full Project
Gutenberg-tm License when you share it without charge with others.

1.D.  The copyright laws of the place where you are located also govern
what you can do with this work.  Copyright laws in most countries are in
a constant state of change.  If you are outside the United States, check
the laws of your country in addition to the terms of this agreement
before downloading, copying, displaying, performing, distributing or
creating derivative works based on this work or any other Project
Gutenberg-tm work.  The Foundation makes no representations concerning
the copyright status of any work in any country outside the United
States.

1.E.  Unless you have removed all references to Project Gutenberg:

1.E.1.  The following sentence, with active links to, or other immediate
access to, the full Project Gutenberg-tm License must appear prominently
whenever any copy of a Project Gutenberg-tm work (any work on which the
phrase "Project Gutenberg" appears, or with which the phrase "Project
Gutenberg" is associated) is accessed, displayed, performed, viewed,
copied or distributed:

This eBook is for the use of anyone anywhere at no cost and with
almost no restrictions whatsoever.  You may copy it, give it away or
re-use it under the terms of the Project Gutenberg License included
with this eBook or online at www.gutenberg.org

1.E.2.  If an individual Project Gutenberg-tm electronic work is derived
from the public domain (does not contain a notice indicating that it is
posted with permission of the copyright holder), the work can be copied
and distributed to anyone in the United States without paying any fees
or charges.  If you are redistributing or providing access to a work
with the phrase "Project Gutenberg" associated with or appearing on the
work, you must comply either with the requirements of paragraphs 1.E.1
through 1.E.7 or obtain permission for the use of the work and the
Project Gutenberg-tm trademark as set forth in paragraphs 1.E.8 or
1.E.9.

1.E.3.  If an individual Project Gutenberg-tm electronic work is posted
with the permission of the copyright holder, your use and distribution
must comply with both paragraphs 1.E.1 through 1.E.7 and any additional
terms imposed by the copyright holder.  Additional terms will be linked
to the Project Gutenberg-tm License for all works posted with the
permission of the copyright holder found at the beginning of this work.

1.E.4.  Do not unlink or detach or remove the full Project Gutenberg-tm
License terms from this work, or any files containing a part of this
work or any other work associated with Project Gutenberg-tm.

1.E.5.  Do not copy, display, perform, distribute or redistribute this
electronic work, or any part of this electronic work, without
prominently displaying the sentence set forth in paragraph 1.E.1 with
active links or immediate access to the full terms of the Project
Gutenberg-tm License.

1.E.6.  You may convert to and distribute this work in any binary,
compressed, marked up, nonproprietary or proprietary form, including any
word processing or hypertext form.  However, if you provide access to or
distribute copies of a Project Gutenberg-tm work in a format other than
"Plain Vanilla ASCII" or other format used in the official version
posted on the official Project Gutenberg-tm web site (www.gutenberg.org),
you must, at no additional cost, fee or expense to the user, provide a
copy, a means of exporting a copy, or a means of obtaining a copy upon
request, of the work in its original "Plain Vanilla ASCII" or other
form.  Any alternate format must include the full Project Gutenberg-tm
License as specified in paragraph 1.E.1.

1.E.7.  Do not charge a fee for access to, viewing, displaying,
performing, copying or distributing any Project Gutenberg-tm works
unless you comply with paragraph 1.E.8 or 1.E.9.

1.E.8.  You may charge a reasonable fee for copies of or providing
access to or distributing Project Gutenberg-tm electronic works provided
that

- You pay a royalty fee of 20% of the gross profits you derive from
     the use of Project Gutenberg-tm works calculated using the method
     you already use to calculate your applicable taxes.  The fee is
     owed to the owner of the Project Gutenberg-tm trademark, but he
     has agreed to donate royalties under this paragraph to the
     Project Gutenberg Literary Archive Foundation.  Royalty payments
     must be paid within 60 days following each date on which you
     prepare (or are legally required to prepare) your periodic tax
     returns.  Royalty payments should be clearly marked as such and
     sent to the Project Gutenberg Literary Archive Foundation at the
     address specified in Section 4, "Information about donations to
     the Project Gutenberg Literary Archive Foundation."

- You provide a full refund of any money paid by a user who notifies
     you in writing (or by e-mail) within 30 days of receipt that s/he
     does not agree to the terms of the full Project Gutenberg-tm
     License.  You must require such a user to return or
     destroy all copies of the works possessed in a physical medium
     and discontinue all use of and all access to other copies of
     Project Gutenberg-tm works.

- You provide, in accordance with paragraph 1.F.3, a full refund of any
     money paid for a work or a replacement copy, if a defect in the
     electronic work is discovered and reported to you within 90 days
     of receipt of the work.

- You comply with all other terms of this agreement for free
     distribution of Project Gutenberg-tm works.

1.E.9.  If you wish to charge a fee or distribute a Project Gutenberg-tm
electronic work or group of works on different terms than are set
forth in this agreement, you must obtain permission in writing from
both the Project Gutenberg Literary Archive Foundation and Michael
Hart, the owner of the Project Gutenberg-tm trademark.  Contact the
Foundation as set forth in Section 3 below.

1.F.

1.F.1.  Project Gutenberg volunteers and employees expend considerable
effort to identify, do copyright research on, transcribe and proofread
public domain works in creating the Project Gutenberg-tm
collection.  Despite these efforts, Project Gutenberg-tm electronic
works, and the medium on which they may be stored, may contain
"Defects," such as, but not limited to, incomplete, inaccurate or
corrupt data, transcription errors, a copyright or other intellectual
property infringement, a defective or damaged disk or other medium, a
computer virus, or computer codes that damage or cannot be read by
your equipment.

1.F.2.  LIMITED WARRANTY, DISCLAIMER OF DAMAGES - Except for the "Right
of Replacement or Refund" described in paragraph 1.F.3, the Project
Gutenberg Literary Archive Foundation, the owner of the Project
Gutenberg-tm trademark, and any other party distributing a Project
Gutenberg-tm electronic work under this agreement, disclaim all
liability to you for damages, costs and expenses, including legal
fees.  YOU AGREE THAT YOU HAVE NO REMEDIES FOR NEGLIGENCE, STRICT
LIABILITY, BREACH OF WARRANTY OR BREACH OF CONTRACT EXCEPT THOSE
PROVIDED IN PARAGRAPH 1.F.3.  YOU AGREE THAT THE FOUNDATION, THE
TRADEMARK OWNER, AND ANY DISTRIBUTOR UNDER THIS AGREEMENT WILL NOT BE
LIABLE TO YOU FOR ACTUAL, DIRECT, INDIRECT, CONSEQUENTIAL, PUNITIVE OR
INCIDENTAL DAMAGES EVEN IF YOU GIVE NOTICE OF THE POSSIBILITY OF SUCH
DAMAGE.

1.F.3.  LIMITED RIGHT OF REPLACEMENT OR REFUND - If you discover a
defect in this electronic work within 90 days of receiving it, you can
receive a refund of the money (if any) you paid for it by sending a
written explanation to the person you received the work from.  If you
received the work on a physical medium, you must return the medium with
your written explanation.  The person or entity that provided you with
the defective work may elect to provide a replacement copy in lieu of a
refund.  If you received the work electronically, the person or entity
providing it to you may choose to give you a second opportunity to
receive the work electronically in lieu of a refund.  If the second copy
is also defective, you may demand a refund in writing without further
opportunities to fix the problem.

1.F.4.  Except for the limited right of replacement or refund set forth
in paragraph 1.F.3, this work is provided to you 'AS-IS', WITH NO OTHER
WARRANTIES OF ANY KIND, EXPRESS OR IMPLIED, INCLUDING BUT NOT LIMITED TO
WARRANTIES OF MERCHANTABILITY OR FITNESS FOR ANY PURPOSE.

1.F.5.  Some states do not allow disclaimers of certain implied
warranties or the exclusion or limitation of certain types of damages.
If any disclaimer or limitation set forth in this agreement violates the
law of the state applicable to this agreement, the agreement shall be
interpreted to make the maximum disclaimer or limitation permitted by
the applicable state law.  The invalidity or unenforceability of any
provision of this agreement shall not void the remaining provisions.

1.F.6.  INDEMNITY - You agree to indemnify and hold the Foundation, the
trademark owner, any agent or employee of the Foundation, anyone
providing copies of Project Gutenberg-tm electronic works in accordance
with this agreement, and any volunteers associated with the production,
promotion and distribution of Project Gutenberg-tm electronic works,
harmless from all liability, costs and expenses, including legal fees,
that arise directly or indirectly from any of the following which you do
or cause to occur: (a) distribution of this or any Project Gutenberg-tm
work, (b) alteration, modification, or additions or deletions to any
Project Gutenberg-tm work, and (c) any Defect you cause.


Section  2.  Information about the Mission of Project Gutenberg-tm

Project Gutenberg-tm is synonymous with the free distribution of
electronic works in formats readable by the widest variety of computers
including obsolete, old, middle-aged and new computers.  It exists
because of the efforts of hundreds of volunteers and donations from
people in all walks of life.

Volunteers and financial support to provide volunteers with the
assistance they need are critical to reaching Project Gutenberg-tm's
goals and ensuring that the Project Gutenberg-tm collection will
remain freely available for generations to come.  In 2001, the Project
Gutenberg Literary Archive Foundation was created to provide a secure
and permanent future for Project Gutenberg-tm and future generations.
To learn more about the Project Gutenberg Literary Archive Foundation
and how your efforts and donations can help, see Sections 3 and 4
and the Foundation information page at www.gutenberg.org


Section 3.  Information about the Project Gutenberg Literary Archive
Foundation

The Project Gutenberg Literary Archive Foundation is a non profit
501(c)(3) educational corporation organized under the laws of the
state of Mississippi and granted tax exempt status by the Internal
Revenue Service.  The Foundation's EIN or federal tax identification
number is 64-6221541.  Contributions to the Project Gutenberg
Literary Archive Foundation are tax deductible to the full extent
permitted by U.S. federal laws and your state's laws.

The Foundation's principal office is located at 4557 Melan Dr. S.
Fairbanks, AK, 99712., but its volunteers and employees are scattered
throughout numerous locations.  Its business office is located at 809
North 1500 West, Salt Lake City, UT 84116, (801) 596-1887.  Email
contact links and up to date contact information can be found at the
Foundation's web site and official page at www.gutenberg.org/contact

For additional contact information:
     Dr. Gregory B. Newby
     Chief Executive and Director
     gbnewby@pglaf.org

Section 4.  Information about Donations to the Project Gutenberg
Literary Archive Foundation

Project Gutenberg-tm depends upon and cannot survive without wide
spread public support and donations to carry out its mission of
increasing the number of public domain and licensed works that can be
freely distributed in machine readable form accessible by the widest
array of equipment including outdated equipment.  Many small donations
($1 to $5,000) are particularly important to maintaining tax exempt
status with the IRS.

The Foundation is committed to complying with the laws regulating
charities and charitable donations in all 50 states of the United
States.  Compliance requirements are not uniform and it takes a
considerable effort, much paperwork and many fees to meet and keep up
with these requirements.  We do not solicit donations in locations
where we have not received written confirmation of compliance.  To
SEND DONATIONS or determine the status of compliance for any
particular state visit www.gutenberg.org/donate

While we cannot and do not solicit contributions from states where we
have not met the solicitation requirements, we know of no prohibition
against accepting unsolicited donations from donors in such states who
approach us with offers to donate.

International donations are gratefully accepted, but we cannot make
any statements concerning tax treatment of donations received from
outside the United States.  U.S. laws alone swamp our small staff.

Please check the Project Gutenberg Web pages for current donation
methods and addresses.  Donations are accepted in a number of other
ways including checks, online payments and credit card donations.
To donate, please visit:  www.gutenberg.org/donate


Section 5.  General Information About Project Gutenberg-tm electronic
works.

Professor Michael S. Hart was the originator of the Project Gutenberg-tm
concept of a library of electronic works that could be freely shared
with anyone.  For forty years, he produced and distributed Project
Gutenberg-tm eBooks with only a loose network of volunteer support.

Project Gutenberg-tm eBooks are often created from several printed
editions, all of which are confirmed as Public Domain in the U.S.
unless a copyright notice is included.  Thus, we do not necessarily
keep eBooks in compliance with any particular paper edition.

Most people start at our Web site which has the main PG search facility:

     www.gutenberg.org

This Web site includes information about Project Gutenberg-tm,
including how to make donations to the Project Gutenberg Literary
Archive Foundation, how to help produce our new eBooks, and how to
subscribe to our email newsletter to hear about new eBooks.
\end{PGtext}

% %%%%%%%%%%%%%%%%%%%%%%%%%%%%%%%%%%%%%%%%%%%%%%%%%%%%%%%%%%%%%%%%%%%%%%% %
%                                                                         %
% End of the Project Gutenberg EBook of The Theory of Spectra and Atomic  %
% Constitution, by Niels (Niels Henrik David) Bohr                        %
%                                                                         %
% *** END OF THIS PROJECT GUTENBERG EBOOK THEORY OF SPECTRA ***           %
%                                                                         %
% ***** This file should be named 47464-t.tex or 47464-t.zip *****        %
% This and all associated files of various formats will be found in:      %
%         http://www.gutenberg.org/4/7/4/6/47464/                         %
%                                                                         %
% %%%%%%%%%%%%%%%%%%%%%%%%%%%%%%%%%%%%%%%%%%%%%%%%%%%%%%%%%%%%%%%%%%%%%%% %

